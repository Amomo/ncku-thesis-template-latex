%
% This file is part of the project of
% National Cheng Kung University (NCKU) Thesis/Dissertation Template in LaTex.
% This project is hold at
%     <https://github.com/wengan-li/ncku-thesis-template-latex>
% by Wen-Gan Li.
%
% This project is distributed in the hope of usefuling to someone,
% you can redistribute it and/or modify it under the terms of the
% Attribution-NonCommercial-ShareAlike 4.0 International.
%
% You should have received a copy of the
% Attribution-NonCommercial-ShareAlike 4.0 International
% along with this project.
% If not, see <http://creativecommons.org/licenses/by-nc-sa/4.0/legalcode.txt>.
%
% Please feel free to fork it, modify it, and try it.
% Have fun !!!
%

% ----------------------------------------------------------------------------
%
% http://tex.stackexchange.com/questions/34312/how-to-create-a-command-with-key-values
%
% 用\begin{figure} .. \end{figure}
% 可能會出現問題
% http://www.tex.ac.uk/cgi-bin/texfaq2html?label=ouparmd
%
% ----------------------------------------------------------------------------

\pgfkeys
{
  /InsertFigures/.is family, /InsertFigures,
  default/.style = 
  {
    perrow = 1,
    caption = \empty,
    label = \empty,
    align = \empty,      % Useless, for backporting
    opacity = 0.75,
  },
  perrow/.estore in = \TmpMIValueImagePerRow,
  caption/.estore in = \TmpMIValueCaption,
  label/.estore in = \TmpMIValueLabel,
  align/.estore in = \TmpMIValueAlign,      % Useless, for backporting
  opacity/.estore in = \TmpValueOpacity,
} % End of \pgfkeys{}

% Insert multi-figure
% Arg: 1st: Table configure
%      2~9th: Figure (Max 8 Figures)
\DeclareDocumentCommand{\InsertFigures}{
  +O{\empty} +m +G{\empty} +G{\empty} +G{\empty}
  +G{\empty} +G{\empty} +G{\empty} +G{\empty}}
{
  % Parse the input
  \pgfkeys{/InsertFigures, default, #1}%
  %
  \begin{figure}[H]%
  \begin{minipage}[c]{\textwidth}%
  \begin{mdframed}[skipabove=0pt, skipbelow=0pt, leftmargin=0pt, rightmargin=0pt,
    innerleftmargin=0pt, innerrightmargin=0pt, innertopmargin=0pt,
    innerbottommargin=0pt, linewidth=0pt, apptotikzsetting={%
    \tikzset{mdfbackground/.append style={opacity=\TmpValueOpacity}}}]%
      \if \TmpMIValueImagePerRow 1
        \InsertFiguresOnePerRow{#2}{#3}{#4}{#5}{#6}{#7}{#8}{#9}%
      \fi
      \if \TmpMIValueImagePerRow 2
        \InsertFiguresTwoPerRow{#2}{#3}{#4}{#5}{#6}{#7}{#8}{#9}%
      \fi
      \if \TmpMIValueImagePerRow 3
        \InsertFiguresThreePerRow{#2}{#3}{#4}{#5}{#6}{#7}{#8}{#9}%
      \fi
      \if \TmpMIValueImagePerRow 4
        \InsertFiguresFourPerRow{#2}{#3}{#4}{#5}{#6}{#7}{#8}{#9}%
      \fi
      %
  \end{mdframed}%
  \end{minipage}%
  % Set Caption and Label
  \SetFigureCaptionAndLabel{\TmpMIValueCaption}{\TmpMIValueLabel}
  \end{figure}%
} % End of \newcommand{}

% Low-level insert image
\newcommand{\InsertSubfigureBox}[2]
{
  \begin{subfigure}{#1\textwidth}%
  \centering
  %
  \InsertFiguresSubFigure#2
  % Set Caption and Label
  \SetFigureCaptionAndLabel{%
    \TmpMISubValueCaption}{\TmpMISubValueLabel}
  \end{subfigure}
} % End of \newcommand{}

%----------------------------------------------------------------

\newcommand{\InsertSubfigureOneFigure}[1]
{
  \InsertSubfigureBox{1.0}{#1}
} % End of \newcommand{}

\newcommand{\InsertSubfigureTwoFigure}[2]
{
  \InsertSubfigureBox{0.5}{#1}%
  ~
  \InsertSubfigureBox{0.5}{#2}%
} % End of \newcommand{}

\newcommand{\InsertSubfigureThreeFigure}[3]
{
  \InsertSubfigureBox{0.315}{#1}%
  ~
  \InsertSubfigureBox{0.315}{#2}%
  ~
  \InsertSubfigureBox{0.315}{#3}%
} % End of \newcommand{}

\newcommand{\InsertSubfigureFourFigure}[4]
{
  \InsertSubfigureBox{0.225}{#1}%
  ~
  \InsertSubfigureBox{0.225}{#2}%
  ~
  \InsertSubfigureBox{0.225}{#3}%
  ~
  \InsertSubfigureBox{0.225}{#4}%
} % End of \newcommand{}

%----------------------------------------------------------------

\DeclareDocumentCommand{\InsertFiguresOnePerRow}{
  +m                   +G{\empty} +G{\empty} +G{\empty}
  +G{\empty} +G{\empty} +G{\empty} +G{\empty}}
{
  \InsertSubfigureOneFigure{#1}
  %
  \ifthenelse{\equal{#2}{\empty}}{}%
  {

    \InsertSubfigureOneFigure{#2}
  }%
  %
  \ifthenelse{\equal{#3}{\empty}}{}%
  {

    \InsertSubfigureOneFigure{#3}
  }%  %
  \ifthenelse{\equal{#4}{\empty}}{}%
  {

    \InsertSubfigureOneFigure{#4}
  }%  %
  \ifthenelse{\equal{#5}{\empty}}{}%
  {

    \InsertSubfigureOneFigure{#5}
  }%  %
  \ifthenelse{\equal{#6}{\empty}}{}%
  {

    \InsertSubfigureOneFigure{#6}
  }%  %
  \ifthenelse{\equal{#7}{\empty}}{}%
  {

    \InsertSubfigureOneFigure{#7}
  }%  %
  \ifthenelse{\equal{#8}{\empty}}{}%
  {

    \InsertSubfigureOneFigure{#8}
  }%
} % End of \newcommand{}

\DeclareDocumentCommand{\InsertFiguresTwoPerRow}{
  +m                   +G{\empty} +G{\empty} +G{\empty}
  +G{\empty} +G{\empty} +G{\empty} +G{\empty}}
{
  %
  \ifthenelse{\equal{#2}{\empty}}%
  {
    \InsertSubfigureOneFigure{#1}%
  }%
  {
    \InsertSubfigureTwoFigure{#1}{#2}%
  }%
  %
  \ifthenelse{\equal{#4}{\empty}}%
  {
    \ifthenelse{\equal{#3}{\empty}}{}%
    {

      \InsertSubfigureOneFigure{#3}%
    }%
  }%
  {

    \InsertSubfigureTwoFigure{#3}{#4}%
  }%
  %
  \ifthenelse{\equal{#6}{\empty}}%
  {
    \ifthenelse{\equal{#5}{\empty}}{}%
    {

      \InsertSubfigureOneFigure{#5}%
    }%
  }%
  {

    \InsertSubfigureTwoFigure{#5}{#6}%
  }%
  %
  \ifthenelse{\equal{#8}{\empty}}%
  {
    \ifthenelse{\equal{#7}{\empty}}{}%
    {

      \InsertSubfigureOneFigure{#7}%
    }%
  }%
  {

    \InsertSubfigureTwoFigure{#7}{#8}%
  }%
} % End of \newcommand{}

\DeclareDocumentCommand{\InsertFiguresThreePerRow}{
  +m                   +G{\empty} +G{\empty} +G{\empty}
  +G{\empty} +G{\empty} +G{\empty} +G{\empty}}
{
  \ifthenelse{\equal{#3}{\empty}}%
  {
    \ifthenelse{\equal{#2}{\empty}}%
    {
      \InsertSubfigureOneFigure{#1}%
    }%
    {
      \InsertSubfigureTwoFigure{#1}{#2}%
    }%
  }%
  {
    \InsertSubfigureThreeFigure{#1}{#2}{#3}%
  }%
  %
  \ifthenelse{\equal{#6}{\empty}}%
  {
    \ifthenelse{\equal{#5}{\empty}}%
    {

      \InsertSubfigureOneFigure{#4}%
    }%
    {

      \InsertSubfigureTwoFigure{#4}{#5}%
    }%
  }%
  {

    \InsertSubfigureThreeFigure{#4}{#5}{#6}%
  }%
  %
  \ifthenelse{\equal{#8}{\empty}}%
  {
    \ifthenelse{\equal{#7}{\empty}}{}%
    {

      \InsertSubfigureOneFigure{#7}%
    }%
  }%
  {

    \InsertSubfigureTwoFigure{#7}{#8}%
  }%
} % End of \newcommand{}

\DeclareDocumentCommand{\InsertFiguresFourPerRow}{
  +m                   +G{\empty} +G{\empty} +G{\empty}
  +G{\empty} +G{\empty} +G{\empty} +G{\empty}}
{
  %
  \ifthenelse{\equal{#4}{\empty}}%
  {
    \ifthenelse{\equal{#3}{\empty}}%
    {
      \ifthenelse{\equal{#2}{\empty}}%
      {
        \InsertSubfigureOneFigure{#1}%
      }%
      {
        \InsertSubfigureTwoFigure{#1}{#2}%
      }%
    }%
    {
      \InsertSubfigureThreeFigure{#1}{#2}{#3}%
    }%
  }%
  {
    \InsertSubfigureFourFigure{#1}{#2}{#3}{#4}%
  }%
  %
  \ifthenelse{\equal{#8}{\empty}}%
  {
    \ifthenelse{\equal{#7}{\empty}}%
    {
      \ifthenelse{\equal{#6}{\empty}}%
      {

        \InsertSubfigureOneFigure{#5}%
      }%
      {

        \InsertSubfigureTwoFigure{#5}{#6}%
      }%
    }%
    {

      \InsertSubfigureThreeFigure{#5}{#6}{#7}%
    }%
  }%
  {

    \InsertSubfigureFourFigure{#5}{#6}{#7}{#8}%
  }%
} % End of \newcommand{}

% ----------------------------------------------------------------------------

\pgfkeys
{
  /InsertFiguresSubFigure/.is family, /InsertFiguresSubFigure,
  default/.style = 
  {
    scale = 1.0,
    angle = 0,
    caption = \empty,
    label = \empty,
    align = \empty,      % Useless, for backporting
  },
  scale/.estore in = \TmpMISubValueScale,
  angle/.estore in = \TmpMISubValueAngle,
  caption/.estore in = \TmpMISubValueCaption,
  label/.estore in = \TmpMISubValueLabel,
  align/.estore in = \TmpMISubValueAlign,      % Useless, for backporting
} % End of \pgfkeys{}

% Low-level insert image
\newcommand{\InsertFiguresSubFigure}[2][\empty]
{
  % Parse the input
  \pgfkeys{/InsertFiguresSubFigure, default, #1}
  %
  \includegraphics[
    scale=\TmpMISubValueScale,
    angle=\TmpMISubValueAngle]{#2}
  %
} % End of \newcommand{}

% ----------------------------------------------------------------------------

% 過去的API, 以 Error提醒不能再使用
\newcommand{\InsertMultiImages}{\errmessage{模版: 由v1.4.1開始, InsertCenterImage已不能再使用, 請改使用InsertFigures.}\stop}

% -----------------------------------------------------------------
