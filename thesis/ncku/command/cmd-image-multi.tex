
% Some helper function about insert image
\usepackage{ifthen}
\usepackage{xparse}

% ----------------------------------------------------------------------------
%
% http://tex.stackexchange.com/questions/34312/how-to-create-a-command-with-key-values
%
% 用\begin{figure} .. \end{figure}
% 可能會出現問題
% http://www.tex.ac.uk/cgi-bin/texfaq2html?label=ouparmd
%
% ----------------------------------------------------------------------------

% -----------------------------------------------------------------

% For multi images
\newcommand\MultiImagesPerRow{1}
\newcommand{\SetMultiImagesPerRow}[1]
  {\renewcommand{\MultiImagesPerRow}{#1}}
\newcommand{\GetMultiImagesPerRow}{\MultiImagesPerRow}

\newcommand\MultiImageTotalValue{0}
\newcommand{\SetMultiImageTotalValue}[1]
  {\renewcommand{\MultiImageTotalValue}{#1}}
\newcommand{\GetMultiImageTotalValue}{\MultiImageTotalValue}

\newcommand\MultiImageId{0}
\newcommand{\SetMultiImageId}[1]
  {\renewcommand{\MultiImageId}{#1}}
\newcommand{\GetMultiImageId}{\MultiImageId}

\newcommand\WidthOfImagePerRow{0}

%add desired spacing between images, e. g. ~, \quad, \qquad etc.
%(or a blank line to force the subfigure onto a new line)
\newcommand{\SubfigureBreakSpaceLine}[1]
{
  \ifthenelse{\equal{\intcalcMod{
    \GetMultiImageId - 1}{\GetMultiImagesPerRow}}{0}}%
  {%
    % Echo blank line
    %

  } % End of if{}
  {%
    ~
  } % End of else{}
} % End of \newcommand{}

\newcommand{\SetWidthOfImagePerRow}
{
  %
%  GetMultiImagesPerRow: \GetMultiImagesPerRow\\
%  GetMultiImageTotalValue: \GetMultiImageTotalValue\\
  %
  \ifthenelse{\equal{\GetMultiImagesPerRow}{1}}
  {
    %perrow = 1
    \FPeval{\WidthOfImagePerRow}{1.0}
  } % End of if{}
  {
    %perrow > 1
    %
    % Flag
    \provideboolean{WidthOfImagePerRowTmpIsDone}
    \setboolean{WidthOfImagePerRowTmpIsDone}{false}
    %
    % Calc remain
    \FPeval{\TmpRemain}{
      clip((\GetMultiImageTotalValue) - (\GetMultiImageId))}
    %TmpRemain: \TmpRemain\\
    %
    % Get mod
    \modulo{\GetMultiImageTotalValue}{\GetMultiImagesPerRow}
    %mod: \result\\
    %
    \FPiflt{\TmpRemain}{\GetMultiImagesPerRow}
    \else
      % Gt
      \setboolean{WidthOfImagePerRowTmpIsDone}{true}
      %
      % Default
      \FPeval{
        \WidthOfImagePerRow}{
          clip(clip(1 / (\GetMultiImagesPerRow)))}
    \fi % End of \FPiflt
    %
    \ifthenelse{\boolean{WidthOfImagePerRowTmpIsDone}}
    {}
    {
      % Lt
      \ifthenelse{\result = 0}
      {
        % Last row
        %
        \FPeval{
          \WidthOfImagePerRow}{
            clip(clip(1 / (\GetMultiImagesPerRow)))}
      } % End of if{}
      {
        % Diff row
        %
        \FPifeq{\TmpRemain}{0}
          \FPeval{\WidthOfImagePerRow}{1.0}
        \else
          \FPeval{
            \WidthOfImagePerRow}{
              clip(clip(1 / (\GetMultiImagesPerRow)))}
        \fi % End of \FPiflt
      } % End of else{}
    } % End of else{}
  } % End of else{}
  %
%  \WidthOfImagePerRow
  %
} % End of \newcommand{}

\pgfkeys
{
  /InsertMultiImages/.is family, /InsertMultiImages,
  default/.style = 
  {
    perrow = 1,
    caption = \empty,
    label = \empty,
    align = \empty,
    pos = H,
    col = 1,
  },
  perrow/.estore in = \TmpMIValueImagePerRow,
  caption/.estore in = \TmpMIValueCaption,
  label/.estore in = \TmpMIValueLabel,
  align/.estore in = \TmpValueAlign,
  pos/.estore in = \TmpMIValuePosition,
  col/.estore in = \TmpMIValueColumn,
} % End of \pgfkeys{}

% Insert multi-image
% Arg: 1st: Table configure
%      2~9th: Images (Max 8 Images)
%
% ---------------------------------------
% 使用 \InsertMultiImageLL#2
% 是跟 \InsertMultiImageLL{#2} 不一樣的
% 直接連接#2是指把#2整個當成function的所有參數
% 而{#2}是指把#2當成單一個參數傳給function
% ---------------------------------------
%
\DeclareDocumentCommand{\InsertMultiImages}{
  +O{\empty} +m +G{\empty} +G{\empty} +G{\empty}
  +G{\empty} +G{\empty} +G{\empty} +G{\empty}}
{
  % Parse the input
  \pgfkeys{/InsertMultiImages, default, #1}
  %
  %Set Multi Image Total
  \ifthenelse{\equal{#2}{\empty}}{}{\SetMultiImageTotalValue{1}}
  \ifthenelse{\equal{#3}{\empty}}{}{\SetMultiImageTotalValue{2}}
  \ifthenelse{\equal{#4}{\empty}}{}{\SetMultiImageTotalValue{3}}
  \ifthenelse{\equal{#5}{\empty}}{}{\SetMultiImageTotalValue{4}}
  \ifthenelse{\equal{#6}{\empty}}{}{\SetMultiImageTotalValue{5}}
  \ifthenelse{\equal{#7}{\empty}}{}{\SetMultiImageTotalValue{6}}
  \ifthenelse{\equal{#8}{\empty}}{}{\SetMultiImageTotalValue{7}}
  \ifthenelse{\equal{#9}{\empty}}{}{\SetMultiImageTotalValue{8}}
  %
  % Set per row
  \FPeval{\MultiImagesPerRow}{clip(
    \TmpMIValueImagePerRow)} %
  %
  %
  % Print begin block
  \ifthenelse{\equal{\TmpMIValueColumn}{1}}
  { %
    \begin{figure}[\TmpMIValuePosition]
  } % End of if{}
  { %
    \begin{figure*}[\TmpMIValuePosition]
  } % End of else{}
  %
  % Do center if needed
  \ifthenelse{\equal{\TmpValueAlign}{center}} %
    {\center}{}
    %
    % Image 1
    %\#2: P #2 P\\
    \ifthenelse{\equal{#2}{\empty}}
      {}
      {
        \SetMultiImageId{1}
        \SubfigureBreakSpaceLine{1}
        \InsertSubfigure#2
      } % End of else{}
    %
    % Image 2
    %\#3: P #3 P\\
    \ifthenelse{\equal{#3}{\empty}}
      {}
      {
        \SetMultiImageId{2}
        \SubfigureBreakSpaceLine{2}
        \InsertSubfigure#3
      } % End of else{}
    %
    % Image 3
    %\#4: P #4 P\\
    \ifthenelse{\equal{#4}{\empty}}
      {}
      {
        \SetMultiImageId{3}
        \SubfigureBreakSpaceLine{3}
        \InsertSubfigure#4
      } % End of else{}
    %
    % Image 4
    %\#5: P #5 P\\
    \ifthenelse{\equal{#5}{\empty}}
      {}
      {
        \SetMultiImageId{4}
        \SubfigureBreakSpaceLine{4}
        \InsertSubfigure#5
      } % End of else{}
    %
    % Image 5
    %\#6: P #6 P\\
    \ifthenelse{\equal{#6}{\empty}}
      {}
      {
        \SetMultiImageId{5}
        \SubfigureBreakSpaceLine{5}
        \InsertSubfigure#6
      } % End of else{}
    %
    % Image 6
    %\#7: P #7 P\\
    \ifthenelse{\equal{#7}{\empty}}
      {}
      {
        \SetMultiImageId{6}
        \SubfigureBreakSpaceLine{6}
        \InsertSubfigure#7
      } % End of else{}
    %
    % Image 7
    %\#8: P #8 P\\
    \ifthenelse{\equal{#8}{\empty}}
      {}
      {
        \SetMultiImageId{7}
        \SubfigureBreakSpaceLine{7}
        \InsertSubfigure#8
      } % End of else{}
    %
    % Image 8
    %\#9: P #9 P\\
    \ifthenelse{\equal{#9}{\empty}}
      {}
      {
        \SetMultiImageId{8}
        \SubfigureBreakSpaceLine{8}
        \InsertSubfigure#9
      } % End of else{}
    %
    % Set Caption
    \SetImageCaption{\TmpMIValueCaption}
    % Set Label
    \ifthenelse{\equal{\TmpMIValueCaption}{\empty}}
      {}{\SetImageLabel{\TmpMIValueLabel}}
    %
  % Print end block
  \ifthenelse{\equal{\TmpMIValueColumn}{1}}
  { %
    \end{figure}
  } % End of if{}
  { %
    \end{figure*}
  } % End of else{}
} % End of \newcommand{}

\pgfkeys
{
  /InsertSubfigure/.is family, /InsertSubfigure,
  default/.style = 
  {
    scale = 1.0,
    angle = 0,
    caption = \empty,
    label = \empty,
    align = \empty,
  },
  scale/.estore in = \TmpMISubValueScale,
  angle/.estore in = \TmpMISubValueAngle,
  caption/.estore in = \TmpMISubValueCaption,
  label/.estore in = \TmpMISubValueLabel,
  align/.estore in = \TmpMISubValueAlign,
} % End of \pgfkeys{}

% Low-level insert image
\newcommand{\InsertSubfigure}[2][\empty]
{
  % Set width
  \SetWidthOfImagePerRow
%  \WidthOfImagePerRow
  %
  % Parse the input
  \pgfkeys{/InsertSubfigure, default, #1}
  %
  \ifthenelse{\equal{\TmpMIValueColumn}{2}}
  { %
    \begin{subfigure}{\WidthOfImagePerRow\textwidth}
  } % End of if{}
  { %
    \begin{subfigure}{\WidthOfImagePerRow\columnwidth}
  } % End of else{}
    %
  %
  % Do center if needed
  \ifthenelse{\equal{\TmpValueAlign}{center}} %
    {\center}{}
    %
    \includegraphics[
      scale=\TmpMISubValueScale,
      angle=\TmpMISubValueAngle]{#2}
    %
    % Set Caption
    \SetImageCaption{\TmpMISubValueCaption}
    % Set Label
    \ifthenelse{\equal{
      \TmpMISubValueCaption}{\empty}}
      {}{\SetImageLabel{\TmpMISubValueLabel}}
  \end{subfigure}%
} % End of \newcommand{}

% ----------------------------------------------------------------------------
