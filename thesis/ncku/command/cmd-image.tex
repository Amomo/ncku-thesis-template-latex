
% Some helper function about insert image

% 用\begin{figure} .. \end{figure}
% 可能會出現問題
% http://www.tex.ac.uk/cgi-bin/texfaq2html?label=ouparmd

\newcommand{\SetImageCaption}[1]
{
  \IfNoValueF{#1}
  {
    \ifthenelse{\equal{#1}{\empty}}
    {}
    {\caption{#1}}
  } % End of IfNoValueF{}
} % End of \newcommand{}

\newcommand{\SetImageLabel}[1]
{
  \IfNoValueF{#1}
  {
    \ifthenelse{\equal{#1}{\empty}}
    {}
    {\label{#1}}
  } % End of IfNoValueF{}
} % End of \newcommand{}

% Insert a single image
\DeclareDocumentCommand{\InsertImage}{m m G{\empty} G{\empty}}
{
  \begin{figure}[hbtp]

    \ifthenelse{\equal{#1}{\empty}}
      {\includegraphics[scale=1.0]{#2}}
      {\includegraphics[scale=#1]{#2}}

    \SetImageCaption{#3}
    \SetImageLabel{#4}
  \end{figure}
} % End of \DeclareDocumentCommand{}

% Insert a single image, put place it in center
\DeclareDocumentCommand{\InsertCenterImage}{m m G{\empty} G{\empty}}
{
  \begin{figure}[hbtp]
    \center

    \ifthenelse{\equal{#1}{\empty}}
      {\includegraphics[scale=1.0]{#2}}
      {\includegraphics[scale=#1]{#2}}

    \SetImageCaption{#3}
    \SetImageLabel{#4}
  \end{figure}
} % End of \DeclareDocumentCommand{}

% For multi images
\newcommand\MultiImagesPerRow{1}
\newcommand{\SetMultiImagesPerRow}[1]
  {\renewcommand{\MultiImagesPerRow}{#1}}
\newcommand{\GetMultiImagesPerRow}[0]{\MultiImagesPerRow}

%add desired spacing between images, e. g. ~, \quad, \qquad etc.
%(or a blank line to force the subfigure onto a new line)
\newcommand{\SubfigureBreakSpaceLine}[1]
{
  \ifthenelse{\equal{\intcalcMod{\GetMultiImagesPerRow}{#1}}{0}}%
  {%
    % Echo blank line
    %
  } % End of if{}
  {%
    \hfill
  } % End of else{}
} % End of \newcommand{}

% Confingure the main image
\newcommand{\SetMultiImageRoot}[3]
{
  % Set per row
  \FPeval{\MultiImagesPerRow}{clip(#1)} %

  \ifthenelse{\equal{\GetMultiImagesPerRow}{1}}%
  {%
    \FPeval{\WidthOfImagePerRow}{1} %
  } % End of if{}
  {%
    \FPeval{\WidthOfImagePerRow}{clip(clip(1 / \GetMultiImagesPerRow) - 0.1)} %
  } % End of else{}
} % End of \newcommand{}

\newcommand{\SetMultiImageRootCaptionAndLabel}[3]
{
  \SetImageCaption{#2} %
  \SetImageLabel{#3} %
} % End of \newcommand{}

\newcommand{\InsertSubfigureImage}[3]
{%
  \begin{subfigure}[b]{\WidthOfImagePerRow\textwidth}%
    \center%
    \includegraphics[scale=1.0]{#1}%
    \SetImageCaption{#2}%
    \SetImageLabel{#3}%
  \end{subfigure}%
} % End of \newcommand{}

% Low-level insert image
\newcommand{\InsertMultiImageLL}[3]
{
  \InsertSubfigureImage{#1}{#2}{#3}%
} % End of \newcommand{}

% Insert multi-image
% Arg: 1st: Configure
% Max 8 Image
%
% ---------------------------------------
% 使用 \InsertMultiImageLL#2
% 是跟 \InsertMultiImageLL{#2} 不一樣的
% 直接連接#2是指把#2整個當成function的所有參數
% 而{#2}是指把#2當成單一個參數傳給function
% ---------------------------------------
%
\DeclareDocumentCommand{\InsertMultiImages}{m g g g g g g g g}
{
  \begin{figure}[hbtp]
    \centering

    \SetMultiImageRoot#1
    % Image 1
    \InsertMultiImageLL#2
    % Image 2
    \IfNoValueF{#3}%
    {%
      \SubfigureBreakSpaceLine{2}%
      \InsertMultiImageLL#3%
    } % End of if{}
    % Image 3
    \IfNoValueF{#4}%
    {%
      \SubfigureBreakSpaceLine{3}%
      \InsertMultiImageLL#4%
    } % End of if{}
    % Image 4
    \IfNoValueF{#5}
    {
      \SubfigureBreakSpaceLine{4}
      \InsertMultiImageLL#5
    } % End of if{}
    % Image 5
    \IfNoValueF{#6}
    {
      \SubfigureBreakSpaceLine{5}
      \InsertMultiImageLL#6
    } % End of if{}
    % Image 6
    \IfNoValueF{#7}
    {
      \SubfigureBreakSpaceLine{6}
      \InsertMultiImageLL#7
    } % End of if{}
    % Image 7
    \IfNoValueF{#8}
    {
      \SubfigureBreakSpaceLine{7}
      \InsertMultiImageLL#8
    } % End of if{}
    % Image 8
    \IfNoValueF{#9}
    {
      \SubfigureBreakSpaceLine{8}
      \InsertMultiImageLL#9
    } % End of if{}
    \SetMultiImageRootCaptionAndLabel#1
  \end{figure}
} % End of \DeclareDocumentCommand{}

% ----------------------------------------------------------------------------


