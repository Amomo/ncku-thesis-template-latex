%
% This file is part of the project of
% National Cheng Kung University (NCKU) Thesis/Dissertation Template in LaTex.
% This project is hold at
%     <https://github.com/wengan-li/ncku-thesis-template-latex>
% by Wen-Gan Li.
%
% This project is distributed in the hope of usefuling to someone,
% you can redistribute it and/or modify it under the terms of the
% Attribution-NonCommercial-ShareAlike 4.0 International.
%
% You should have received a copy of the
% Attribution-NonCommercial-ShareAlike 4.0 International
% along with this project.
% If not, see <http://creativecommons.org/licenses/by-nc-sa/4.0/legalcode.txt>.
%
% Please feel free to fork it, modify it, and try it.
% Have fun !!!
%

% Some helper function for reference

% ----------------------------------------------------------------------------

% 為了能連同顯示的內容都能控制, 故做多一層command來包
% Implatment a custom '\label' to have more control
% \LabelThisAs{ < label_name >}{ < format/display_value > }

\makeatletter
\newcommand{\LabelThisAs}[2]
{%
  \def\@currentlabel{#2}%
  \label{#1}%
} % End of \newcommand{}
\makeatother

% ----------------------------------------------------------------------------

% For equation
\newcommand{\RefEquation}[1]{\ref{#1}}

% For equation
\newcommand{\RefEquationB}[1]{\eqref{#1}}

% For bib
\newcommand{\RefBib}[1]{\cite{#1}}

% For figure, table, chapter, section, subsection, .etc
\newcommand{\RefTo}[1]
{
%  \ifthenelse{\equal{\GetStartAppendixChapter}{\ValueEnableAppendixChapter}}%
%  {%
%    \refstepcounter{appendixchapter}
%    \refstepcounter{appendixsection}
%    \refstepcounter{appendixsubsection}
%    \refstepcounter{appendixsubsubsection}
%  }%
%  {%
%    \refstepcounter{chapter}
%    \refstepcounter{section}
%    \refstepcounter{subsection}
%    \refstepcounter{subsubsection}
%  }%
  \ref{#1}%
} % End of \newcommand{}

\newcommand{\RefFigure}[1]{\RefTo{#1}}

\newcommand{\RefTable}[1]{\RefTo{#1}}

% For page
\newcommand{\RefPage}[1]{\pageref{#1}}


% ----------------------------------------------------------------------------

% 過去的API, 以 Error提醒不能再使用
%\newcommand{\RefTo}{\errmessage{模版: 由v1.4.5開始, RefTo已不再推薦使用. 請改用.}\stop}

%\makeatletter
%\newcommand{\todo}[1][]{\@latex@warning{TODO #1}\fbox{TODO\dots}}
%\makeatother

% ----------------------------------------------------------------------------
