%
% This file is part of the project of
% National Cheng Kung University (NCKU) Thesis/Dissertation Template in LaTex.
% This project is hold at
%     <https://github.com/wengan-li/ncku-thesis-template-latex>
% by Wen-Gan Li.
%
% This project is distributed in the hope of usefuling to someone,
% you can redistribute it and/or modify it under the terms of the
% Attribution-NonCommercial-ShareAlike 4.0 International.
%
% You should have received a copy of the
% Attribution-NonCommercial-ShareAlike 4.0 International
% along with this project.
% If not, see <http://creativecommons.org/licenses/by-nc-sa/4.0/legalcode.txt>.
%
% Please feel free to fork it, modify it, and try it.
% Have fun !!!
%

% Some common helper function

% ----------------------------------------------------------------------------

% Some helper functions

\newcommand{\GetMonthInEng}[1]
{%
  \ifthenelse{\equal{#1}{1}}{January}{}%
  \ifthenelse{\equal{#1}{2}}{February}{}%
  \ifthenelse{\equal{#1}{3}}{March}{}%
  \ifthenelse{\equal{#1}{4}}{April}{}%
  \ifthenelse{\equal{#1}{5}}{May}{}%
  \ifthenelse{\equal{#1}{6}}{June}{}%
  \ifthenelse{\equal{#1}{7}}{July}{}%
  \ifthenelse{\equal{#1}{8}}{August}{}%
  \ifthenelse{\equal{#1}{9}}{September}{}%
  \ifthenelse{\equal{#1}{10}}{October}{}%
  \ifthenelse{\equal{#1}{11}}{November}{}%
  \ifthenelse{\equal{#1}{12}}{December}{}%
} % End of \newcommand{}

% 計算出台灣民國幾年
% Get the year using Taiwans' year
\newcommand{\SetOralTaiwanYear}[1]%
{%
  \FPeval{\OralTaiwanYearResult}{clip(#1 - 1911)}%
} % End of \newcommand{}

% 計算出台灣民國幾年
% Get the year using Taiwans' year
\newcommand{\SetThesisTaiwanYear}[1]%
{%
  \FPeval{\ThesisTaiwanYearResult}{clip(#1 - 1911)}%
} % End of \newcommand{}

% ----------------------------------------------------------------------------

% In the minimal example below the macro \modulo{<a>}{<b>} stores the result of <a> mod <b> in the macro \result
\newcommand{\modulo}[2]{%
  \FPeval{\result}{trunc(#1-(#2*trunc(#1/#2,0)),0)}%
}

% ----------------------------------------------------------------------------

% 定義了 fmpage: 一個加框的展示區 framed minipage
% http://brunoj.wordpress.com/2009/10/08/latex-the-framed-minipage/
\newsavebox{\fmbox}
\newenvironment{fmpage}[1]
{\begin{lrbox}{\fmbox}\begin{minipage}{#1}}
{\end{minipage}\end{lrbox}\fbox{\usebox{\fmbox}}}

% ----------------------------------------------------------------------------

\newcommand{\EmptyLine}{\ \\ \par}

% ----------------------------------------------------------------------------

\global\mdfdefinestyle{DescriptionFrameStyle}{%
  linewidth=1pt, apptotikzsetting={%
    \tikzset{mdfbackground/.append style={opacity=0.75}}}%
} % End of \mdfdefinestyle{}
\newenvironment{DescriptionFrame}%
{\begin{mdframed}[style=DescriptionFrameStyle]}%
{\end{mdframed}}

% ----------------------------------------------------------------------------
