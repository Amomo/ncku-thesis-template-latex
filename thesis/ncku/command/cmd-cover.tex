%
% This file is part of the project of
% National Cheng Kung University (NCKU) Thesis/Dissertation Template in LaTex.
% This project is hold at
%     <https://github.com/wengan-li/ncku-thesis-template-latex>
% by Wen-Gan Li.
%
% This project is distributed in the hope of usefuling to someone,
% you can redistribute it and/or modify it under the terms of the
% Attribution-NonCommercial-ShareAlike 4.0 International.
%
% You should have received a copy of the
% Attribution-NonCommercial-ShareAlike 4.0 International
% along with this project.
% If not, see <http://creativecommons.org/licenses/by-nc-sa/4.0/legalcode.txt>.
%
% Please feel free to fork it, modify it, and try it.
% Have fun !!!
%

% Some helper function use in cover

% ----------------------------------------------------------------------------
\newcommand{\StartCover}
{
  %
  \UseEngLinesSpacing
  %
  \StartNewPage
  %
  % 設定使用 無頁碼
  \thispagestyle{empty}
  %
  \EnableCoverPageStyle
  %
  % Aligned to the center of the page
  \begin{center}
} % End of \newcommand{}

\newcommand{\EndCover}
{
  % End of alignment
  \end{center}
  \DisableCoverPageStyle
  \EndOfPage
  \UseDefaultLinesSpacing
} % End of \newcommand{}
% ----------------------------------------------------------------------------

% --- University name 學校名字 ---
% 基本上是寫死, 但是如果是別校的人, 可直接使用\SetSchoolName來修改
\newcommand\VarUniversityChiName{國立成功大學}           % Default
\newcommand\VarUniversityEngName{National Cheng Kung University} % Default
\newcommand{\SetSchoolChiName}[1]{\renewcommand{\VarUniversityChiName}{#1}}
\newcommand{\SetSchoolEngName}[1]{\renewcommand{\VarUniversityEngName}{#1}}
\newcommand{\SetSchoolName}[2]
{
  \SetSchoolChiName{#1}
  \SetSchoolEngName{#2}
} % End of \newcommand{}

\newcommand{\GetSchoolChiName}{\VarUniversityChiName}
\newcommand{\GetSchoolEngName}{\VarUniversityEngName}
% ----------------------------------------------------------------------------

% --- Chinese / English title 中英文論文題目 ---
\newcommand{\VarThesisChiName}{Chinese Title Here} % Default
\newcommand{\VarThesisEngName}{English Title Here} % Default
\newcommand{\SetChiTitle}[1]{\renewcommand{\VarThesisChiName}{#1}}
\newcommand{\SetEngTitle}[1]{\renewcommand{\VarThesisEngName}{#1}}
\newcommand{\SetTitle}[2]
{
  \SetChiTitle{#1}
  \SetEngTitle{#2}
} % End of \newcommand{}

\newcommand{\GetChiTitle}{\VarThesisChiName}
\newcommand{\GetEngTitle}{\VarThesisEngName}

% ----------------------------------------------------------------------------

% --- User's name 使用者名字 ---
\newcommand{\VarMyChiName}{你的名字}     % Default
\newcommand{\VarMyEngName}{Your name}   % Default
\newcommand{\SetMyChiName}[1]{\renewcommand{\VarMyChiName}{#1}}
\newcommand{\SetMyEngName}[1]{\renewcommand{\VarMyEngName}{#1}}
\newcommand{\SetMyName}[2]
{
  \SetMyChiName{#1}
  \SetMyEngName{#2}
} % End of \newcommand{}

\newcommand{\GetAuthorChiName}{\VarMyChiName}
\newcommand{\GetAuthorEngName}{\VarMyEngName}

% ----------------------------------------------------------------------------

% --- Degree name 學位 ---
% thesis 是指論文的通稱
% dissertation 指的是博士的論文

% 碩士論文  Master's thesis
% 博士論文  Doctoral dissertation

\newcommand{\ValueDegreeMaster}{0}
\newcommand{\ValueDegreePhd}{1}
\newcommand{\FlagDegreeType}{\ValueDegreePhd} % Default
\newcommand{\GetFlagDegreeType}{\FlagDegreeType}
\newcommand{\SetFlagDegreeType}[1]{\renewcommand{\FlagDegreeType}{#1}}

\newcommand{\VarDegreeChiName}{碩士/博士} % Default
\newcommand{\VarDegreeEngName}{Master / Doctor} % Default
\newcommand{\degreeThesisEname}{Master's Thesis / Doctoral Dissertation} % Default

\newcommand{\GetChiDegree}{\VarDegreeChiName}
\newcommand{\GetEngDegree}{\VarDegreeEngName}
\newcommand{\GetEngDegreeThesis}{\degreeThesisEname}
\newcommand{\SetChiDegree}[1]{\renewcommand{\VarDegreeChiName}{#1}}
\newcommand{\SetEngDegree}[1]{\renewcommand{\VarDegreeEngName}{#1}}
\newcommand{\SetEngDegreeThesis}[1]{\renewcommand{\degreeThesisEname}{#1}}

\newcommand{\PhdDegree}
{
  \SetFlagDegreeType{\ValueDegreePhd}
  \SetChiDegree{博士}
  \SetEngDegree{Doctor}
  \SetEngDegreeThesis{Doctoral Dissertation}
} % End of \newcommand{}

\newcommand{\MasterDegree}
{
  \SetFlagDegreeType{\ValueDegreeMaster}
  \SetChiDegree{碩士}
  \SetEngDegree{Master}
  \SetEngDegreeThesis{Master's Thesis}
} % End of \newcommand{}

% ----------------------------------------------------------------------------

% --- Date 日期 ---

% \CoverDateNumInChi: 日期使用中文數字,
% 而不是阿拉伯數字,
% 故使用'\CoverDateNumInChi'可以顯示
% '第一章' 而不是 '中華民國 103 年 12 月 31 日'.
% \CoverDateNumInChi必須配合\DisplayCoverInChi來使用, 否則會無效.
%\CoverDateNumInChi

% --- 論文的日期 ---
\newcommand{\ThesisYear}{2014}  % Default
\newcommand{\ThesisMonth}{1}    % Default

\newcommand{\SetThesisDate}[2]{\SetThesisDate{#1}{#2}} % For backporting
\newcommand{\SetCoverDate}[2]
{
  \SetThesisTaiwanYear{#1}
  \renewcommand{\ThesisYear}{#1}
  \renewcommand{\ThesisMonth}{#2}
} % End of \newcommand{}

\newcommand{\GetThesisYear}{\ThesisYear}
\newcommand{\GetThesisYearInTaiwanYear}{\ThesisTaiwanYearResult}
\newcommand{\GetThesisMonth}{\ThesisMonth}
\newcommand{\GetThesisMonthNumInChi}{\zhnumber{\ThesisMonth}}
\newcommand{\GetThesisMonthInEng}{\GetMonthInEng{\ThesisMonth}}

% ---  口試的日期 ---
\newcommand{\OralChiYear}{101}      % Default
\newcommand{\OralChiMonth}{1}       % Default
\newcommand{\OralChiDay}{1}         % Default
\newcommand{\OralEngYear}{2014}     % Default
\newcommand{\OralEngMonth}{January} % Default
\newcommand{\OralEngDay}{1}         % Default

\newcommand{\GetOralChiYear}{\OralChiYear}
\newcommand{\GetOralYearInTaiwanYear}
{\SetThesisTaiwanYear{\OralEngYear}\ThesisTaiwanYearResult}
\newcommand{\GetOralYearInTaiwanYearNumInChi}
{\SetThesisTaiwanYear{\OralEngYear}\zhdigits{\ThesisTaiwanYearResult}}
\newcommand{\GetOralChiMonth}{\OralChiMonth}
\newcommand{\GetOralChiDay}{\OralChiDay}
\newcommand{\GetOralEngYear}{\OralEngYear}
\newcommand{\GetOralEngMonth}{\OralEngMonth}
\newcommand{\GetOralEngDay}{\OralEngDay}
\newcommand{\GetOralEngDayNumInChi}{\zhnumber{\OralEngDay}}

\newcommand{\SetOralChiDate}[3]
{
  \SetOralTaiwanYear{#1}
  \renewcommand{\OralChiYear}{\OralTaiwanYearResult}
  \renewcommand{\OralChiMonth}{#2}
  \renewcommand{\OralChiDay}{#3}
} % End of \newcommand{}

\newcommand{\SetOralEngDate}[3]
{
  \renewcommand{\OralEngYear}{#1}
  \renewcommand{\OralEngMonth}{\GetMonthInEng{#2}}
  \renewcommand{\OralEngDay}{#3}
} % End of \newcommand{}

\newcommand{\SetOralDate}[3]
{
  \SetOralChiDate{#1}{#2}{#3}
  \SetOralEngDate{#1}{#2}{#3}
} % End of \newcommand{}

% ----------------------------------------------------------------------------

% --- 學院 College, 系所 Department and Institute ---

% --------------------------- College ---------------------------
\newcommand{\VarCollegeChiName}{學院 C}
\newcommand{\VarCollegeEngName}{College of C}
\newcommand{\SetCollChiName}[1]{\renewcommand{\VarCollegeChiName}{#1}}
\newcommand{\SetCollEngName}[1]{\renewcommand{\VarCollegeEngName}{#1}}
\newcommand{\SetCollName}[2]
{
  \SetCollChiName{#1}
  \SetCollEngName{#2}
} % End of \newcommand{}

\newcommand{\GetCollChiName}{\VarCollegeChiName}
\newcommand{\GetCollEngName}{\VarCollegeEngName}

% --------------------------- Department ---------------------------
\newcommand{\VarDepartmentChiName}{A 系 / 所}
%\newcommand{\VarDepartmentEngName}{DeptA} % Short form of department
\newcommand{\VarDepartmentEngFullName}{Department / Insitute A} % Full name of department
\newcommand{\SetDeptChiName}[1]{\renewcommand{\VarDepartmentChiName}{#1}}
%\newcommand{\SetDeptEngShortName}[1]{\renewcommand{\VarDepartmentEngName}{#1}}
\newcommand{\SetDeptEngFullName}[1]{\renewcommand{\VarDepartmentEngFullName}{#1}}
\newcommand{\SetDeptName}[3]
{
  \SetDeptChiName{#1}
%  \SetDeptEngShortName{#2}
  \SetDeptEngFullName{#3}
} % End of \newcommand{}

\newcommand{\GetDeptChiName}{\VarDepartmentChiName}
\newcommand{\GetDeptEngName}{\VarDepartmentEngFullName}

% ----------------------------------------------------------------------------

% --- 指導老師 Advisor(s) ---
% 在封面上預算了最多3位的空間
% 中文名字固定以 博士 結尾
% 英文名字固定以 Dr. 開頭

\newcommand{\VarAdvisorChiNameA}{X}
\newcommand{\VarAdvisorEngNameA}{X}
\newcommand{\VarAdvisorChiNameB}{}
\newcommand{\VarAdvisorEngNameB}{}
\newcommand{\VarAdvisorChiNameC}{}
\newcommand{\VarAdvisorEngNameC}{}

\newcommand{\GetAdvisorChiNameA}{\VarAdvisorChiNameA}
\newcommand{\GetAdvisorEngNameA}{\VarAdvisorEngNameA}
\newcommand{\GetAdvisorChiNameB}{\VarAdvisorChiNameB}
\newcommand{\GetAdvisorEngNameB}{\VarAdvisorEngNameB}
\newcommand{\GetAdvisorChiNameC}{\VarAdvisorChiNameC}
\newcommand{\GetAdvisorEngNameC}{\VarAdvisorEngNameC}

\newcommand{\SetAdvisorChiNameA}[1]{\renewcommand{\VarAdvisorChiNameA}{#1}}
\newcommand{\SetAdvisorEngNameA}[1]{\renewcommand{\VarAdvisorEngNameA}{#1}}
\newcommand{\SetAdvisorChiNameB}[1]{\renewcommand{\VarAdvisorChiNameB}{#1}}
\newcommand{\SetAdvisorEngNameB}[1]{\renewcommand{\VarAdvisorEngNameB}{#1}}
\newcommand{\SetAdvisorChiNameC}[1]{\renewcommand{\VarAdvisorChiNameC}{#1}}
\newcommand{\SetAdvisorEngNameC}[1]{\renewcommand{\VarAdvisorEngNameC}{#1}}

\newcommand{\SetAdvisorNameA}[2]
{
  \SetAdvisorChiNameA{#1}
  \SetAdvisorEngNameA{#2}
} % End of \newcommand{}

\newcommand{\SetAdvisorNameB}[2]
{
  \SetAdvisorChiNameB{#1}
  \SetAdvisorEngNameB{#2}
} % End of \newcommand{}

\newcommand{\SetAdvisorNameC}[2]
{
  \SetAdvisorChiNameC{#1}
  \SetAdvisorEngNameC{#2}
} % End of \newcommand{}

% ----------------------------------------------------------------------------

% Use to create cover
\newcommand{\CreateCover}%
{
  \begin{document}
  %
% This file is part of ncku-thesis-template.
%
% ncku-thesis-template is distributed in the hope of usefuling to someone,
% you can redistribute it and/or modify
% it under the terms of the Attribution-NonCommercial-ShareAlike
% 4.0 International.
%
% You should have received a copy of the
% Attribution-NonCommercial-ShareAlike 4.0 International
% along with ncku-thesis-template.
%
% If not, see <http://creativecommons.org/licenses/by-nc-sa/4.0/legalcode.txt>.
%

% ------------------------------------------------

% 根據user的需求去選用中文或英文封面
%
% 封面: 顯示所有封面內容, 但沒有學校Logo
% 內頁: 顯示所有封面內容, 但有學校Logo
% 不論是精裝版或平裝版都是 封面 (殼/皮) + 內頁
%

% 封面沒有學校Logo
\ClearWatermark

\ifthenelse{\GetLangEnableChi = 1}
{
  %
% This file is part of ncku-thesis-templete.
%
% ncku-thesis-templete is distributed in the hope that it will be useful,
% you can redistribute it and/or modify
% it under the terms of the Attribution-NonCommercial-ShareAlike
% 4.0 International.
%
% You should have received a copy of the
% Attribution-NonCommercial-ShareAlike 4.0 International
% along with ncku-thesis-templete.
%
% If not, see <http://creativecommons.org/licenses/by-nc-sa/4.0/legalcode.txt>.
%

% ----------------------------------------------------------------------------
%                Chinese cover
%                   中文封面
% ----------------------------------------------------------------------------

% Set the line spacing to single for the titles (to compress the lines)
\renewcommand{\baselinestretch}{1}   %行距 1 倍

% ------------------------------------------------

\StartNewPage

% Add to "Table of Contents"
\addcontentsline{toc}{chapter}{Cover}

% 設定使用 無頁碼, 有浮水印
\thispagestyle{empty}

% Aligned to the center of the page
\begin{center}

% ------------------------------------------------

% 顯示 校名, 系所名, 論文種類
\begin{minipage}[c][5cm][t]{\textwidth}
  \begin{center}
    \makebox[10cm][s]{\Huge \GetSchoolChiName} \\

    \vspace{1cm}
    \makebox[8cm][s]{\Huge \GetDeptChiName} \\

    \vspace{1cm}
    \makebox[5cm][s]{\Huge \GetChiDegree 論文} \\
  \end{center}
\end{minipage}
% ------------------------------------------------

\vspace{5cm}

% ------------------------------------------------

% Chinese and English title 中英文題目
\begin{minipage}[c][5cm][t]{\textwidth}
  \begin{center}
%    \parbox{\textwidth}{\center \Large \GetChiTitle}
    \makebox[\textwidth][c]{\parbox{\paperwidth }{\center \Large \GetChiTitle}}

    \vspace{0.5cm}
%    \parbox{\textwidth}{\center \Large \GetEngTitle}
    \makebox[\textwidth][c]{\parbox{\paperwidth }{\center \Large \GetEngTitle}}
  \end{center}
\end{minipage}

% ------------------------------------------------

\vspace{1.5cm}

% ------------------------------------------------

% 顯示學生和老師的名字

\begin{minipage}[c][4.5cm][t]{\textwidth}
  \begin{center}

  % --------------------------

  % 顯示 學生 的名字
  \hspace{2.0em}
  \makebox[4.0em][r]{\Large 研究生:}
  \makebox[6.0em][l]{\Large \GetAuthorChiName}
  \makebox[8.0em][c]{}
  \makebox[4.0em][r]{\Large Student:}
  \makebox[8.0em][l]{\Large \GetAuthorEngName}\\
  % --------------------------

  \vspace{0.5cm}

  % --------------------------

  % 顯示 指導老師 A 的名字
  \hspace{2.0em}
  \makebox[4.0em][r]{\Large 指導老師:}
  \makebox[6.0em][l]{\Large \GetAdvisorChiNameA \thinspace 博士}
  \makebox[8.0em][c]{}
  \makebox[4.0em][r]{\Large Advisor:}
  \makebox[8.0em][l]{\Large Dr. \thinspace \GetAdvisorEngNameA}\\
  % --------------------------

  \vspace{0.1cm}

  % --------------------------

  % 顯示 指導老師 B 的名字
  \hspace{2.0em}
  \makebox[4.0em][r]
  {%
    \ifthenelse{\equal{\GetAdvisorChiNameB}{\empty}}%
      {}%
      {\Large 共同指導:}%
  }
  \makebox[6.0em][l]%
  {%
    \ifthenelse{\equal{\GetAdvisorChiNameB}{\empty}}%
      {}%
      {\Large \GetAdvisorChiNameB \thinspace 博士}%
  }
  \makebox[8.0em][c]{}
  \makebox[4.0em][r]
  {%
    \ifthenelse{\equal{\GetAdvisorEngNameB}{\empty}}%
      {}%
      {\Large Co-Advisor:}%
  }
  \makebox[8.0em][l]%
  {%
    \ifthenelse{\equal{\GetAdvisorEngNameB}{\empty}}%
      {}%
      {\Large Dr. \thinspace \GetAdvisorEngNameB}%
  } \\

  % --------------------------

  \vspace{0.1cm}

  % --------------------------

  % 顯示 指導老師 C 的名字
  \hspace{2.0em}
  \makebox[4.0em][r]{}
  \makebox[6.0em][l]
  {%
    \ifthenelse{\equal{\GetAdvisorChiNameC}{\empty}}%
      {}%
      {\Large \GetAdvisorChiNameC \thinspace 博士}%
  }
  \makebox[8.0em][c]{}
  \makebox[4.0em][r]{}
  \makebox[8.0em][l]
  {%
    \ifthenelse{\equal{\GetAdvisorEngNameC}{\empty}}%
      {}%
      {\Large Dr. \thinspace \GetAdvisorEngNameC}%
  } \\

  % --------------------------

  \end{center}
\end{minipage}
% ------------------------------------------------

% Date 日期
\vspace{0.5cm}
\makebox[8cm][s]{\Large 中華民國 \GetThesisYearInTaiwanYear 年 \GetThesisMonth 月}

% ------------------------------------------------

% End of alignment
\end{center}

\EndOfPage
% ------------------------------------------------

}{}

\ifthenelse{\GetLangEnableEng = 1}
{
  %
% This file is part of ncku-thesis-template.
%
% ncku-thesis-template is distributed in the hope of usefuling to someone,
% you can redistribute it and/or modify
% it under the terms of the Attribution-NonCommercial-ShareAlike
% 4.0 International.
%
% You should have received a copy of the
% Attribution-NonCommercial-ShareAlike 4.0 International
% along with ncku-thesis-template.
%
% If not, see <http://creativecommons.org/licenses/by-nc-sa/4.0/legalcode.txt>.
%

% ----------------------------------------------------------------------------
%                English cover
%                   英文封面
% ----------------------------------------------------------------------------

% ------------------------------------------------
\StartCover
% ------------------------------------------------

% 顯示 校名, 系所名, 論文種類
\begin{minipage}[c][5cm][t]{\textwidth}
  \begin{center}
    \vspace{0.8cm}
    \makebox[\textwidth][c]{\Huge \GetSchoolEngName} \\

    \vspace{0.5cm}
    \makebox[\textwidth][c]{\LARGE \GetDeptEngName} \\

    \vspace{0.5cm}
    \makebox[\textwidth][c]{\LARGE \GetEngDegreeThesis} \\

    \ifthenelse{\equal{\GetFlagDisplayDraft}{1}}%
      {\vspace{0.5cm}\makebox[5cm][c]{\LARGE \GetTextDraftEng}}{}
  \end{center}
\end{minipage}

% ------------------------------------------------

\vspace{5.5cm}

% ------------------------------------------------

% English title 英文題目
\begin{minipage}[c][5cm][t]{\textwidth}
  \begin{center}
%    \parbox{\textwidth}{\center \LARGE \GetEngTitle}

%    \parbox{\textwidth}{\center \Large \GetChiTitle}
    \makebox[\textwidth][c]{\parbox{\paperwidth }{\center \Large \GetChiTitle}}

    \vspace{0.5cm}
%    \parbox{\textwidth}{\center \Large \GetEngTitle}
    \makebox[\textwidth][c]{\parbox{\paperwidth }{\center \Large \GetEngTitle}}
  \end{center}
\end{minipage}

% ------------------------------------------------

\vspace{1.0cm}

% ------------------------------------------------

% 顯示學生和老師的名字

\begin{minipage}[c][4.5cm][t]{\textwidth}
  \begin{center}

    \ifthenelse{\equal{\GetCDBothName}{1}}%
      {%
        % ----- 中英文同時顯示 -----

        % --------------------------

        % 顯示 學生 的名字
        \hspace{2.0em}
        \makebox[4.0em][r]{\Large 學生:}
        \makebox[6.0em][l]{\Large \GetAuthorChiName}
        \makebox[8.0em][c]{}
        \makebox[4.0em][r]{\Large Student:}
        \makebox[8.0em][l]{\Large \GetAuthorEngName}\\
        % --------------------------

        \vspace{0.5cm}

        % --------------------------

        % 顯示 指導老師 A 的名字
        \hspace{2.0em}
        \makebox[4.0em][r]{\Large 指導老師:}
        \makebox[6.0em][l]{\Large \GetAdvisorChiNameA \thinspace 教授}
        \makebox[8.0em][c]{}
        \makebox[4.0em][r]{\Large Advisor:}
        \makebox[8.0em][l]{\Large Prof. \thinspace \GetAdvisorEngNameA}\\
        % --------------------------

        \vspace{0.1cm}

        % --------------------------

        % 顯示 指導老師 B 的名字
        \hspace{2.0em}
        \makebox[4.0em][r]
        {%
          \ifthenelse{\equal{\GetAdvisorChiNameB}{\empty}}%
            {}%
            {\Large 共同指導:}%
        }
        \makebox[6.0em][l]%
        {%
          \ifthenelse{\equal{\GetAdvisorChiNameB}{\empty}}%
            {}%
            {\Large \GetAdvisorChiNameB \thinspace 教授}%
        }
        \makebox[8.0em][c]{}
        \makebox[4.0em][r]
        {%
          \ifthenelse{\equal{\GetAdvisorEngNameB}{\empty}}%
            {}%
            {\Large Co-Advisor:}%
        }
        \makebox[8.0em][l]%
        {%
          \ifthenelse{\equal{\GetAdvisorEngNameB}{\empty}}%
            {}%
            {\Large Prof. \thinspace \GetAdvisorEngNameB}%
        } \\

        % --------------------------

        \vspace{0.1cm}

        % --------------------------

        % 顯示 指導老師 C 的名字
        \hspace{2.0em}
        \makebox[4.0em][r]{}
        \makebox[6.0em][l]
        {%
          \ifthenelse{\equal{\GetAdvisorChiNameC}{\empty}}%
            {}%
            {\Large \GetAdvisorChiNameC \thinspace 教授}%
        }
        \makebox[8.0em][c]{}
        \makebox[4.0em][r]{}
        \makebox[8.0em][l]
        {%
          \ifthenelse{\equal{\GetAdvisorEngNameC}{\empty}}%
            {}%
            {\Large Prof. \thinspace \GetAdvisorEngNameC}%
        } \\

        % --------------------------
      }%
      {%
        % ----- 只顯示英文 -----

        % --------------------------

        % 顯示 學生 的名字
        \makebox[7em][r]{\Large Student:}
        \makebox[10.0em][l]{\Large \GetAuthorEngName}\\

        % --------------------------

        \vspace{0.5cm}

        % --------------------------

        % 顯示 指導老師 A 的名字
        \makebox[7em][r]{\Large Advisor:}
        \makebox[10.0em][l]{\Large Prof. \thinspace \GetAdvisorEngNameA} \\

        % --------------------------

        \vspace{0.1cm}

        % --------------------------

        % 顯示 指導老師 B 的名字
        \makebox[7em][r]
        {%
          \ifthenelse{\equal{\GetAdvisorEngNameB}{\empty}}%
            {}%
            {\Large Co-Advisor:}%
        }
        \makebox[10.0em][l]
        {%
          \ifthenelse{\equal{\GetAdvisorEngNameB}{\empty}}%
            {}%
            {\Large Prof. \thinspace \GetAdvisorEngNameB}%
        } \\

        % --------------------------

        \vspace{0.1cm}

        % --------------------------

        % 顯示 指導老師 C 的名字
        \makebox[7em][r]{}
        \makebox[10.0em][l]
        {%
          \ifthenelse{\equal{\GetAdvisorEngNameC}{\empty}}%
            {}%
            {\Large Prof. \thinspace \GetAdvisorEngNameC}%
        } \\
        % --------------------------

      }%

  \end{center}
\end{minipage}
% ------------------------------------------------

% Date 日期
\vspace{0.5cm}
\begin{minipage}{\textwidth}
  \begin{center}\Large %
    \ifthenelse{\equal{\GetFlagDegreeType}{\ValueDegreeMaster}}%
      {\GetThesisMonthInEng \thinspace \GetThesisYear}%
      {\GetOralEngDay \thinspace \thinspace \GetThesisMonthInEng \thinspace \thinspace \GetThesisYear}
  \end{center}
\end{minipage}

% ------------------------------------------------
\EndCover
% ------------------------------------------------

}{}

% 重新使用學校浮水印 Watermark
\UseSchoolWatermark

% ------------------------------------------------

  \end{document}
} % End of \newcommand{}

% Use to include and display inner cover
\newcommand{\DisplayInnerCover}{%
% This file is part of the project of
% National Cheng Kung University (NCKU) Thesis/Dissertation Template in LaTex.
% This project is hold at
%     <https://github.com/wengan-li/ncku-thesis-template-latex>
% by Wen-Gan Li.
%
% This project is distributed in the hope of usefuling to someone,
% you can redistribute it and/or modify it under the terms of the
% Attribution-NonCommercial-ShareAlike 4.0 International.
%
% You should have received a copy of the
% Attribution-NonCommercial-ShareAlike 4.0 International
% along with this project.
% If not, see <http://creativecommons.org/licenses/by-nc-sa/4.0/legalcode.txt>.
%
% Please feel free to fork it, modify it, and try it.
% Have fun !!!
%

% ------------------------------------------------

% 根據user的需求去選用中文或英文封面內頁
%
% 封面: 顯示所有封面內容, 但沒有學校Logo
% 內頁: 顯示所有封面內容, 但有學校Logo
% 不論是精裝版或平裝版都是 封面 (殼/皮) + 內頁
%
\if \GetDisplayCoverLang \ValueDisplayCoverLangEng
  %
% This file is part of ncku-thesis-template.
%
% ncku-thesis-template is distributed in the hope of usefuling to someone,
% you can redistribute it and/or modify
% it under the terms of the Attribution-NonCommercial-ShareAlike
% 4.0 International.
%
% You should have received a copy of the
% Attribution-NonCommercial-ShareAlike 4.0 International
% along with ncku-thesis-template.
%
% If not, see <http://creativecommons.org/licenses/by-nc-sa/4.0/legalcode.txt>.
%

% ----------------------------------------------------------------------------
%                English cover
%                   英文封面
% ----------------------------------------------------------------------------

% ------------------------------------------------
\StartCover
% ------------------------------------------------

% 顯示 校名, 系所名, 論文種類
\begin{minipage}[c][5cm][t]{\textwidth}
  \begin{center}
    \vspace{0.8cm}
    \makebox[\textwidth][c]{\Huge \GetSchoolEngName} \\

    \vspace{0.5cm}
    \makebox[\textwidth][c]{\LARGE \GetDeptEngName} \\

    \vspace{0.5cm}
    \makebox[\textwidth][c]{\LARGE \GetEngDegreeThesis} \\

    \ifthenelse{\equal{\GetFlagDisplayDraft}{1}}%
      {\vspace{0.5cm}\makebox[5cm][c]{\LARGE \GetTextDraftEng}}{}
  \end{center}
\end{minipage}

% ------------------------------------------------

\vspace{5.5cm}

% ------------------------------------------------

% English title 英文題目
\begin{minipage}[c][5cm][t]{\textwidth}
  \begin{center}
%    \parbox{\textwidth}{\center \LARGE \GetEngTitle}

%    \parbox{\textwidth}{\center \Large \GetChiTitle}
    \makebox[\textwidth][c]{\parbox{\paperwidth }{\center \Large \GetChiTitle}}

    \vspace{0.5cm}
%    \parbox{\textwidth}{\center \Large \GetEngTitle}
    \makebox[\textwidth][c]{\parbox{\paperwidth }{\center \Large \GetEngTitle}}
  \end{center}
\end{minipage}

% ------------------------------------------------

\vspace{1.0cm}

% ------------------------------------------------

% 顯示學生和老師的名字

\begin{minipage}[c][4.5cm][t]{\textwidth}
  \begin{center}

    \ifthenelse{\equal{\GetCDBothName}{1}}%
      {%
        % ----- 中英文同時顯示 -----

        % --------------------------

        % 顯示 學生 的名字
        \hspace{2.0em}
        \makebox[4.0em][r]{\Large 學生:}
        \makebox[6.0em][l]{\Large \GetAuthorChiName}
        \makebox[8.0em][c]{}
        \makebox[4.0em][r]{\Large Student:}
        \makebox[8.0em][l]{\Large \GetAuthorEngName}\\
        % --------------------------

        \vspace{0.5cm}

        % --------------------------

        % 顯示 指導老師 A 的名字
        \hspace{2.0em}
        \makebox[4.0em][r]{\Large 指導老師:}
        \makebox[6.0em][l]{\Large \GetAdvisorChiNameA \thinspace 教授}
        \makebox[8.0em][c]{}
        \makebox[4.0em][r]{\Large Advisor:}
        \makebox[8.0em][l]{\Large Prof. \thinspace \GetAdvisorEngNameA}\\
        % --------------------------

        \vspace{0.1cm}

        % --------------------------

        % 顯示 指導老師 B 的名字
        \hspace{2.0em}
        \makebox[4.0em][r]
        {%
          \ifthenelse{\equal{\GetAdvisorChiNameB}{\empty}}%
            {}%
            {\Large 共同指導:}%
        }
        \makebox[6.0em][l]%
        {%
          \ifthenelse{\equal{\GetAdvisorChiNameB}{\empty}}%
            {}%
            {\Large \GetAdvisorChiNameB \thinspace 教授}%
        }
        \makebox[8.0em][c]{}
        \makebox[4.0em][r]
        {%
          \ifthenelse{\equal{\GetAdvisorEngNameB}{\empty}}%
            {}%
            {\Large Co-Advisor:}%
        }
        \makebox[8.0em][l]%
        {%
          \ifthenelse{\equal{\GetAdvisorEngNameB}{\empty}}%
            {}%
            {\Large Prof. \thinspace \GetAdvisorEngNameB}%
        } \\

        % --------------------------

        \vspace{0.1cm}

        % --------------------------

        % 顯示 指導老師 C 的名字
        \hspace{2.0em}
        \makebox[4.0em][r]{}
        \makebox[6.0em][l]
        {%
          \ifthenelse{\equal{\GetAdvisorChiNameC}{\empty}}%
            {}%
            {\Large \GetAdvisorChiNameC \thinspace 教授}%
        }
        \makebox[8.0em][c]{}
        \makebox[4.0em][r]{}
        \makebox[8.0em][l]
        {%
          \ifthenelse{\equal{\GetAdvisorEngNameC}{\empty}}%
            {}%
            {\Large Prof. \thinspace \GetAdvisorEngNameC}%
        } \\

        % --------------------------
      }%
      {%
        % ----- 只顯示英文 -----

        % --------------------------

        % 顯示 學生 的名字
        \makebox[7em][r]{\Large Student:}
        \makebox[10.0em][l]{\Large \GetAuthorEngName}\\

        % --------------------------

        \vspace{0.5cm}

        % --------------------------

        % 顯示 指導老師 A 的名字
        \makebox[7em][r]{\Large Advisor:}
        \makebox[10.0em][l]{\Large Prof. \thinspace \GetAdvisorEngNameA} \\

        % --------------------------

        \vspace{0.1cm}

        % --------------------------

        % 顯示 指導老師 B 的名字
        \makebox[7em][r]
        {%
          \ifthenelse{\equal{\GetAdvisorEngNameB}{\empty}}%
            {}%
            {\Large Co-Advisor:}%
        }
        \makebox[10.0em][l]
        {%
          \ifthenelse{\equal{\GetAdvisorEngNameB}{\empty}}%
            {}%
            {\Large Prof. \thinspace \GetAdvisorEngNameB}%
        } \\

        % --------------------------

        \vspace{0.1cm}

        % --------------------------

        % 顯示 指導老師 C 的名字
        \makebox[7em][r]{}
        \makebox[10.0em][l]
        {%
          \ifthenelse{\equal{\GetAdvisorEngNameC}{\empty}}%
            {}%
            {\Large Prof. \thinspace \GetAdvisorEngNameC}%
        } \\
        % --------------------------

      }%

  \end{center}
\end{minipage}
% ------------------------------------------------

% Date 日期
\vspace{0.5cm}
\begin{minipage}{\textwidth}
  \begin{center}\Large %
    \ifthenelse{\equal{\GetFlagDegreeType}{\ValueDegreeMaster}}%
      {\GetThesisMonthInEng \thinspace \GetThesisYear}%
      {\GetOralEngDay \thinspace \thinspace \GetThesisMonthInEng \thinspace \thinspace \GetThesisYear}
  \end{center}
\end{minipage}

% ------------------------------------------------
\EndCover
% ------------------------------------------------

\else
  %
% This file is part of ncku-thesis-templete.
%
% ncku-thesis-templete is distributed in the hope that it will be useful,
% you can redistribute it and/or modify
% it under the terms of the Attribution-NonCommercial-ShareAlike
% 4.0 International.
%
% You should have received a copy of the
% Attribution-NonCommercial-ShareAlike 4.0 International
% along with ncku-thesis-templete.
%
% If not, see <http://creativecommons.org/licenses/by-nc-sa/4.0/legalcode.txt>.
%

% ----------------------------------------------------------------------------
%                Chinese cover
%                   中文封面
% ----------------------------------------------------------------------------

% Set the line spacing to single for the titles (to compress the lines)
\renewcommand{\baselinestretch}{1}   %行距 1 倍

% ------------------------------------------------

\StartNewPage

% Add to "Table of Contents"
\addcontentsline{toc}{chapter}{Cover}

% 設定使用 無頁碼, 有浮水印
\thispagestyle{empty}

% Aligned to the center of the page
\begin{center}

% ------------------------------------------------

% 顯示 校名, 系所名, 論文種類
\begin{minipage}[c][5cm][t]{\textwidth}
  \begin{center}
    \makebox[10cm][s]{\Huge \GetSchoolChiName} \\

    \vspace{1cm}
    \makebox[8cm][s]{\Huge \GetDeptChiName} \\

    \vspace{1cm}
    \makebox[5cm][s]{\Huge \GetChiDegree 論文} \\
  \end{center}
\end{minipage}
% ------------------------------------------------

\vspace{5cm}

% ------------------------------------------------

% Chinese and English title 中英文題目
\begin{minipage}[c][5cm][t]{\textwidth}
  \begin{center}
%    \parbox{\textwidth}{\center \Large \GetChiTitle}
    \makebox[\textwidth][c]{\parbox{\paperwidth }{\center \Large \GetChiTitle}}

    \vspace{0.5cm}
%    \parbox{\textwidth}{\center \Large \GetEngTitle}
    \makebox[\textwidth][c]{\parbox{\paperwidth }{\center \Large \GetEngTitle}}
  \end{center}
\end{minipage}

% ------------------------------------------------

\vspace{1.5cm}

% ------------------------------------------------

% 顯示學生和老師的名字

\begin{minipage}[c][4.5cm][t]{\textwidth}
  \begin{center}

  % --------------------------

  % 顯示 學生 的名字
  \hspace{2.0em}
  \makebox[4.0em][r]{\Large 研究生:}
  \makebox[6.0em][l]{\Large \GetAuthorChiName}
  \makebox[8.0em][c]{}
  \makebox[4.0em][r]{\Large Student:}
  \makebox[8.0em][l]{\Large \GetAuthorEngName}\\
  % --------------------------

  \vspace{0.5cm}

  % --------------------------

  % 顯示 指導老師 A 的名字
  \hspace{2.0em}
  \makebox[4.0em][r]{\Large 指導老師:}
  \makebox[6.0em][l]{\Large \GetAdvisorChiNameA \thinspace 博士}
  \makebox[8.0em][c]{}
  \makebox[4.0em][r]{\Large Advisor:}
  \makebox[8.0em][l]{\Large Dr. \thinspace \GetAdvisorEngNameA}\\
  % --------------------------

  \vspace{0.1cm}

  % --------------------------

  % 顯示 指導老師 B 的名字
  \hspace{2.0em}
  \makebox[4.0em][r]
  {%
    \ifthenelse{\equal{\GetAdvisorChiNameB}{\empty}}%
      {}%
      {\Large 共同指導:}%
  }
  \makebox[6.0em][l]%
  {%
    \ifthenelse{\equal{\GetAdvisorChiNameB}{\empty}}%
      {}%
      {\Large \GetAdvisorChiNameB \thinspace 博士}%
  }
  \makebox[8.0em][c]{}
  \makebox[4.0em][r]
  {%
    \ifthenelse{\equal{\GetAdvisorEngNameB}{\empty}}%
      {}%
      {\Large Co-Advisor:}%
  }
  \makebox[8.0em][l]%
  {%
    \ifthenelse{\equal{\GetAdvisorEngNameB}{\empty}}%
      {}%
      {\Large Dr. \thinspace \GetAdvisorEngNameB}%
  } \\

  % --------------------------

  \vspace{0.1cm}

  % --------------------------

  % 顯示 指導老師 C 的名字
  \hspace{2.0em}
  \makebox[4.0em][r]{}
  \makebox[6.0em][l]
  {%
    \ifthenelse{\equal{\GetAdvisorChiNameC}{\empty}}%
      {}%
      {\Large \GetAdvisorChiNameC \thinspace 博士}%
  }
  \makebox[8.0em][c]{}
  \makebox[4.0em][r]{}
  \makebox[8.0em][l]
  {%
    \ifthenelse{\equal{\GetAdvisorEngNameC}{\empty}}%
      {}%
      {\Large Dr. \thinspace \GetAdvisorEngNameC}%
  } \\

  % --------------------------

  \end{center}
\end{minipage}
% ------------------------------------------------

% Date 日期
\vspace{0.5cm}
\makebox[8cm][s]{\Large 中華民國 \GetThesisYearInTaiwanYear 年 \GetThesisMonth 月}

% ------------------------------------------------

% End of alignment
\end{center}

\EndOfPage
% ------------------------------------------------

\fi

% ------------------------------------------------
}

% ----------------------------------------------------------------------------

\newcommand{\ValueDisplayCoverLangEng}{0}
\newcommand{\ValueDisplayCoverLangChi}{1}
\newcommand{\VarDisplayCoverLang}{\ValueDisplayCoverLangEng}
\newcommand{\GetDisplayCoverLang}{\VarDisplayCoverLang}
\newcommand{\DisplayCoverInChi}{\renewcommand{\VarDisplayCoverLang}{\ValueDisplayCoverLangChi}}
\newcommand{\DisplayCoverInEng}{\renewcommand{\VarDisplayCoverLang}{\ValueDisplayCoverLangEng}}

% 日期顯示中文數字
\newcommand{\ValueDisplayCoverDateNumInNum}{0}
\newcommand{\ValueDisplayCoverDateNumInChi}{1}
\newcommand{\VarDisplayCoverDateNum}{\ValueDisplayCoverDateNumInNum}
\newcommand{\GetDisplayCoverDateNum}{\VarDisplayCoverDateNum}
\newcommand{\CoverDateNumInChi}{\renewcommand{\VarDisplayCoverDateNum}{\ValueDisplayCoverDateNumInChi}}

% ----------------------------------------------------------------------------

% Display Chinese and English name in english cover
\newcommand{\CoverDisplayNameChiEng}{0} % Default not display both
\newcommand{\SetCDBothName}{\renewcommand{\CoverDisplayNameChiEng}{1}}
\newcommand{\GetCDBothName}{\CoverDisplayNameChiEng}
\newcommand{\DisplayCoverPeoplesBothNames}{\SetCDBothName}

% A wrapper to handle \CDBothName{}
\newcommand{\CDBothName}{\DisplayCoverPeoplesBothNames}

% ----------------------------------------------------------------------------

% 顯示 '(初稿)' (中文版) 和 '(Draft)' (英文版) 在封面
\newcommand{\GetTextDraftChi}{(初稿)}
\newcommand{\GetTextDraftEng}{(Draft)}
\newcommand{\VarCoverDisplayDraft}{0} % Don't display in default
\newcommand{\EnableFlagDisplayDraft}{\renewcommand{\VarCoverDisplayDraft}{1}}
\newcommand{\DisplayDraft}{\EnableFlagDisplayDraft}
\newcommand{\GetFlagDisplayDraft}{\VarCoverDisplayDraft}

% ----------------------------------------------------------------------------
