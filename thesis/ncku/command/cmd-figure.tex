%
% This file is part of the project of
% National Cheng Kung University (NCKU) Thesis/Dissertation Template in LaTex.
% This project is hold at
%     <https://github.com/wengan-li/ncku-thesis-template-latex>
% by Wen-Gan Li.
%
% This project is distributed in the hope of usefuling to someone,
% you can redistribute it and/or modify it under the terms of the
% Attribution-NonCommercial-ShareAlike 4.0 International.
%
% You should have received a copy of the
% Attribution-NonCommercial-ShareAlike 4.0 International
% along with this project.
% If not, see <http://creativecommons.org/licenses/by-nc-sa/4.0/legalcode.txt>.
%
% Please feel free to fork it, modify it, and try it.
% Have fun !!!
%

% ----------------------------------------------------------------------------
%
% http://tex.stackexchange.com/questions/34312/how-to-create-a-command-with-key-values
%
% 用\begin{figure} .. \end{figure}
% 可能會出現問題
% http://www.tex.ac.uk/cgi-bin/texfaq2html?label=ouparmd
%
% ----------------------------------------------------------------------------

\DeclareDocumentCommand{\SetFigureCaptionAndLabel}{+m +m}
{
  \ifthenelse{\equal{#1}{\empty}}{}%
  {%
    \ifthenelse{\equal{%
      \GetStartExtendedAbstractFigureTableControl}{%
      \ValueDisableExtendedAbstractFigureTableControl}}%
      {\caption{#1}}{\caption[]{#1}}
    \ifthenelse{\equal{#2}{\empty}}{}{\label{#2}}%
  }%
} % End of \DeclareDocumentCommand{}

\DeclareDocumentCommand{\SetFigureCaption}{+m}
{
  \ifthenelse{\equal{#1}{\empty}}{}{\IfNoValueF{#1}{\caption{#1}}}
} % End of \DeclareDocumentCommand{}

\DeclareDocumentCommand{\SetImageLabel}{+m}
{
  \ifthenelse{\equal{#1}{\empty}}{}{\IfNoValueF{#1}{\label{#1}}}
} % End of \DeclareDocumentCommand{}

% -----------------------------------------------------------------

\pgfkeys
{
  /InsertFigure/.is family, /InsertFigure,
  default/.style =
  {
    scale = 1.0,
    angle = 0,
    caption = \empty,
    label = \empty,
    pos = {H},      % Useless, for backporting
    align = \empty, % Useless, for backporting
    opacity = 0.75,
  },
  scale/.estore in = \TmpValueScale,
  angle/.estore in = \TmpValueAngle,
  caption/.estore in = \TmpValueCaption,
  label/.estore in = \TmpValueLabel,
  pos/.estore in = \TmpValuePosition,   % Useless, for backporting
  align/.estore in = \TmpValueAlign,    % Useless, for backporting
  opacity/.estore in = \TmpValueOpacity,
} % End of \pgfkeys{}

% Insert a single column image
\newcommand{\InsertFigure}[2][\empty]
{%
  % Parse the input
  \pgfkeys{/InsertFigure, default, #1}%
  %
  \begin{figure}[H]%
  \begin{minipage}[c]{\textwidth}%
  \begin{mdframed}[skipabove=0pt, skipbelow=0pt, leftmargin=0pt, rightmargin=0pt,
    innerleftmargin=0pt, innerrightmargin=0pt, innertopmargin=0pt,
    innerbottommargin=0pt, linewidth=0pt, apptotikzsetting={%
    \tikzset{mdfbackground/.append style={opacity=\TmpValueOpacity}}}]%
    \makebox[\textwidth]{%
      \includegraphics[
        scale=\TmpValueScale,
        angle=\TmpValueAngle]{#2}%
    }%
  \end{mdframed}%
  \end{minipage}%
  % Set Caption and Label
  \SetFigureCaptionAndLabel{\TmpValueCaption}{\TmpValueLabel}
  \end{figure}%
} % End of \newcommand{}

% -----------------------------------------------------------------

\def\ValueFigureNameDefault{Figure}
\def\ValueFigureNameCustom{Figure} % Default
\def\UseFigureNameDefault{%
  \renewcommand{\figurename}{\ValueFigureNameDefault}}
\def\UseFigureNameCustom{%
  \renewcommand{\figurename}{\ValueFigureNameCustom}}
\newcommand{\SetCustomFigureName}[1]{%
  \renewcommand{\ValueFigureNameCustom}{#1}}

\UseFigureNameCustom % Default

% -----------------------------------------------------------------

% 過去的API, 以 Error提醒不能再使用
\newcommand{\InsertCenterImage}{\errmessage{模版: 由v1.4.1開始, InsertCenterImage已不能再使用, 請改使用InsertFigure.}\stop}
\newcommand{\InsertImage}{\errmessage{模版: 由v1.4.1開始, InsertCenterImage已不能再使用, 請改使用InsertFigure.}\stop}

% -----------------------------------------------------------------
\begin{comment}
\def\ValueFigureNameBoldOn{1}
\def\ValueFigureNameBoldOff{0}
\def\VarFigureNameBoldOption{\ValueFigureNameBoldOn} %Default
\def\GetFigureNameBoldOption{\VarFigureNameBoldOption}
\newcommand{\EnableFigureNameBold}{%
  \renewcommand{\VarFigureNameBoldOption}{\ValueFigureNameBoldOn}}
\newcommand{\DisableFigureNameBold}{%
  \renewcommand{\VarFigureNameBoldOption}{\ValueFigureNameBoldOff}}

% ------------------------------------------

\def\ValueFigureTextBoldOn{3}
\def\ValueFigureTextBoldOff{2}
\def\VarFigureTextBoldOption{\ValueFigureTextBoldOff} %Default
\def\GetFigureTextBoldOption{\VarFigureTextBoldOption}
\newcommand{\EnableFigureTextBold}{%
  \renewcommand{\VarFigureTextBoldOption}{\ValueFigureTextBoldOn}}
\newcommand{\DisableFigureTextBold}{%
  \renewcommand{\VarFigureTextBoldOption}{\ValueFigureTextBoldOff}}
\end{comment}
% ------------------------------------------

% Default style
\newcommand{\UseFigureCaptionDefaultStyle}
{%
%  \ifthenelse{\equal{\GetFigureNameBoldOption}{\ValueFigureNameBoldOn}}
%  {%
%    \captionsetup[figure]{labelfont=bf}
%  }%
%  {%
%    \captionsetup[figure]{labelfont=normalfont}
%  }%
  %
%  \ifthenelse{\equal{\GetFigureTextBoldOption}{\ValueFigureTextBoldOn}}
%  {%
%    \captionsetup[figure]{textfont=bf}%
%  }%
%  {%
%    \captionsetup[figure]{textfont=normalfont}%
%  }%
  \captionsetup[figure]{labelfont=bf, textfont=normalfont}%
} % End of \newcommand{}

% Style for Extended Abstract
\newcommand{\UseFigureCaptionExtendedAbstractStyle}
{%
  \captionsetup[figure]{font=bf}%
  \renewcommand{\thefigure}{\arabic{figure}}%
} % End of \newcommand{}

\UseFigureCaptionDefaultStyle % Default

% -----------------------------------------------------------------
