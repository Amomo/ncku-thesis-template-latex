%
% This file is part of the project of
% National Cheng Kung University (NCKU) Thesis/Dissertation Template in LaTex.
% This project is hold at
%     <https://github.com/wengan-li/ncku-thesis-template-latex>
% by Wen-Gan Li.
%
% This project is distributed in the hope of usefuling to someone,
% you can redistribute it and/or modify it under the terms of the
% Attribution-NonCommercial-ShareAlike 4.0 International.
%
% You should have received a copy of the
% Attribution-NonCommercial-ShareAlike 4.0 International
% along with this project.
% If not, see <http://creativecommons.org/licenses/by-nc-sa/4.0/legalcode.txt>.
%
% Please feel free to fork it, modify it, and try it.
% Have fun !!!
%

% Some helper function about chapter and section

% ----------------------------------------------------------------------------
\begin{comment}
\newcommand{\VarChapterHeaderSeparatorInToc}{.} % Default
\newcommand{\SetChapterTitleIndexSep}[1]{%
  \renewcommand{\VarChapterHeaderSeparatorInToc}{#1}}

\newcommand{\VarSectionHeaderSeparatorInToc}{.} % Default
\newcommand{\SetSectionTitleIndexSep}[1]{%
  \renewcommand{\VarSectionHeaderSeparatorInToc}{#1}}

\newcommand{\VarSubSectionHeaderSeparatorInToc}{.} % Default
\newcommand{\SetSubSectionTitleIndexSep}[1]{%
  \renewcommand{\VarSubSectionHeaderSeparatorInToc}{#1}}

\newcommand{\VarSubSubSectionHeaderSeparatorInToc}{.} % Default
\newcommand{\SetSubSubSectionTitleIndexSep}[1]{%
  \renewcommand{\VarSubSubSectionHeaderSeparatorInToc}{#1}}
\end{comment}
% ----------------------------------------------------------------------------

\newcommand{\DisplayChapterHeader}[2]%
{%
 \vspace*{0pt}%
  {%
    \parindent 0pt \centering%
    \Large\bf #1 \par%
    \vskip 15pt%
    \Large \bf #2 \par%
    \nobreak%
%    \vskip 35pt%
    \vskip 25pt%
  }%
  \addcontentsline{%
    toc}{chapter}{#1\GetCTitleNumberFormatSepAtIndex{\ \ \ }#2}%
  \addtocontents{lof}{\protect\addvspace{10 pt}}
  \addtocontents{lot}{\protect\addvspace{10 pt}}
} % End of \newcommand{}

\newcommand{\DisplayChapterHeaderStar}[1]%
{%
  \vspace*{0pt}%
  {%
    \parindent 0pt \centering \Large \bf #1\par%
    \nobreak%
%    \vskip 30pt%
    \vskip 20pt%
  }%
  \addcontentsline{toc}{chapter}{#1}%
} % End of \newcommand{}

% ----------------------------------------------------------------------------

% 使用 \frontmatter, \mainmatter 其實可簡化事情,
% 但這增加不需要的內容在context.tex, 使用bypass的方法已減少同學的煩惱.
\newcommand{\VarPageInMainmatter}{1}
\newcommand{\VarPageNotInMainmatter}{0}
\newcommand{\ValuePageInMainmatter}{\VarPageNotInMainmatter} % Default
\newcommand{\SetValuePageInMainmatter}{%
  \renewcommand{\ValuePageInMainmatter}{\VarPageInMainmatter}}
\newcommand{\GetValuePageInMainmatter}{\ValuePageInMainmatter}

% 這邊是overload latex原生的\chapter{}和\chapter*{}
\RenewDocumentCommand{\chapter}{s m}
{%
  %
  % 檢查是 \chapter*{} 或是 \chapter{}
  \IfBooleanTF{#1}%
  {%
    % \chapter*{}
    %
    % 用\chapter*{}都是前幾頁的內容, 用羅馬數字
    %
    %    Starred
    \DisplayChapterHeaderStar{#2}
  }%
  {%
    % \chapter{}
    %
    % 用\chapter{}都是主要內容, 故用數字
    % 如果第一次使用\chapter{}, 則把頁碼改使用成數字
    \ifthenelse{\equal{\GetValuePageInMainmatter}{\VarPageNotInMainmatter}}%
    {\pagenumbering{arabic}\SetValuePageInMainmatter}{}
    %
    %    Non-Starred
    %
    \ifthenelse{\equal{\GetStartAppendixChapter}{\ValueEnableAppendixChapter}}%
    {%
      % Appendix Chapter
      \refstepcounter{appendixchapter}
      %
      %
%      \SetAppendixTitleFormatFinalString
%      \DisplayChapterHeader{\GetAppendixTitleFormatFinalString}{#2}
      \SetupAppendixChapterTitleNumberFormatString%
      \DisplayChapterHeader{\GetAppendixChapterTitleNumberFormatString}{#2}
    }%
    {%
      % 一般Chapter
      \refstepcounter{chapter}
      %
      %
      \SetupChapterTitleNumberFormatString%
      \DisplayChapterHeader{\GetChapterTitleNumberFormatString}{#2}
%      \ifthenelse{\equal{\GetChapterTitleLang}{\VarChapterTitleLangEng}}%
%      {%
        % 題目為英文
%        \DisplayChapterHeader{第\thechapter 章}{#2}
%        \DisplayChapterHeader{\ValueChapterTitleNumberingFormat}{#2}
%      }%
%      {%
        % 題目為中文
%        \DisplayChapterHeader{第\thechapter 章}{#2}
%      }%
    }%
  }%
  %
  % Reset counters
  \setcounter{section}{0}%
  \setcounter{appendixsection}{0}%
  \setcounter{figure}{0}%
  \setcounter{table}{0}%
  \setcounter{equation}{0}%
} % End of \RenewDocumentCommand{}

% ----------------------------------------------------------------------------
\begin{comment}
\makeatletter

%%% Restored from latex.ltx--Glenn G. Lai

\def\@startsection#1#2#3#4#5#6{\if@noskipsec \leavevmode \fi
   \par \@tempskipa #4\relax
   \@afterindenttrue
   \ifdim \@tempskipa <\z@
     \@tempskipa -\@tempskipa \@afterindentfalse
   \fi
   \if@nobreak \everypar{}\else
     \addpenalty{\@secpenalty}\addvspace{\@tempskipa}\fi \@ifstar
     {\@ssect{#3}{#4}{#5}{#6}}%
     {\@dblarg{\@sect{#1}{#2}{#3}{#4}{#5}{#6}}}}

 \def\section{\@startsection{section}{1}{\z@}%
    {4.6ex plus -1ex minus -.2ex}{.3ex plus .2ex}{\bf}}
\def\subsection{\@startsection{subsection}{2}{\z@}%
    {3.9ex plus -1ex minus -.2ex}{.2ex plus .2ex}{\bf}}
\def\subsubsection{\@startsection{subsubsection}{3}{\z@}%
    {3.25ex plus -1ex minus -.2ex}{.1ex plus .2ex}{\bf}}

\makeatother
\end{comment}
% ----------------------------------------------------------------------------

\newcommand{\DisplaySectionHeader}[2]%
{%
  \vspace{0.5cm}%
  \ifthenelse{\equal{#1}{\empty}}%
  {%
    \parindent 0pt \textbf{#2} \par%
  }%
  {%
    \parindent 0pt \textbf{#1\hspace{0.5cm}#2} \par%
  }%
  \vspace{0.1cm}%
  \addcontentsline{%
    toc}{section}{#1\GetSTitleNumberFormatSepAtIndex{\ \ \ }#2}%
} % End of \newcommand{}

\RenewDocumentCommand{\section}{s m}
{%
  %
  % 檢查是 \section*{} 或是 \section{}
  \IfBooleanTF{#1}%
  {%
    % \section*{}
%        Starred
  }%
  {%
    % \section{}
%        Non-Starred
    %
    \ifthenelse{\equal{\GetStartAppendixChapter}{\ValueEnableAppendixChapter}}%
    {%
      \refstepcounter{appendixsection}%
      \SetupAppendixSectionTitleNumberFormatString%
      \DisplaySectionHeader{\GetAppendixSectionTitleNumberFormatString}{#2}
    }%
    {%
      \refstepcounter{section}%
      \SetupSectionTitleNumberFormatString%
      \DisplaySectionHeader{\GetSectionTitleNumberFormatString}{#2}
    }%
  }%
  % Reset counters
  \setcounter{subsection}{0}%
  \setcounter{appendixsubsection}{0}%
} % End of \RenewDocumentCommand{}

% ----------------------------------------------------------------------------

\newcommand{\DisplaySubSectionHeader}[2]%
{%
  \vspace{0.5cm}%
  \ifthenelse{\equal{#1}{\empty}}%
  {%
    \parindent 0pt \textbf{#2} \par%
  }%
  {%
    \parindent 0pt \textbf{#1\hspace{0.5cm}#2} \par%
  }%
  \vspace{0.1cm}%
  \addcontentsline{%
    toc}{subsection}{#1\GetSSTitleNumberFormatSepAtIndex{\ \ \ }#2}%
} % End of \newcommand{}

\RenewDocumentCommand{\subsection}{s m}
{%
  %
  % 檢查是 \subsection*{} 或是 \subsection{}
  \IfBooleanTF{#1}%
  {%
    % \subsection*{}
%        Starred
  }%
  {%
    % \subsection{}
%        Non-Starred
    %
    \ifthenelse{\equal{\GetStartAppendixChapter}{\ValueEnableAppendixChapter}}%
    {%
      \refstepcounter{appendixsubsection}%
      \SetupAppendixSubSectionTitleNumberFormatString%
      \DisplaySubSectionHeader{%
        \GetAppendixSubSectionTitleNumberFormatString}{#2}
    }%
    {%
      \refstepcounter{subsection}%
      \SetupSubSectionTitleNumberFormatString%
      \DisplaySubSectionHeader{%
        \GetSubSectionTitleNumberFormatString}{#2}
    }%
  }%
  % Reset counters
  \setcounter{subsubsection}{0}%
  \setcounter{appendixsubsubsection}{0}%
} % End of \RenewDocumentCommand{}

% ----------------------------------------------------------------------------

\newcommand{\DisplaySubSubSectionHeader}[2]%
{%
  \vspace{0.5cm}%
  \ifthenelse{\equal{#1}{\empty}}%
  {%
    \parindent 0pt \textbf{#2} \par%
  }%
  {%
    \parindent 0pt \textbf{#1\hspace{0.5cm}#2} \par%
  }%
  \vspace{0.1cm}%
  \addcontentsline{%
    toc}{subsubsection}{#1\GetSSSTitleNumberFormatSepAtIndex{\ \ \ }#2}%
} % End of \newcommand{}

\RenewDocumentCommand{\subsubsection}{s m}
{%
  %
  % 檢查是 \subsubsection*{} 或是 \subsubsection{}
  \IfBooleanTF{#1}%
  {%
    % \subsubsection*{}
%        Starred
  }%
  {%
    % \subsubsection{}
%        Non-Starred
    %
    \ifthenelse{\equal{\GetStartAppendixChapter}{\ValueEnableAppendixChapter}}%
    {%
      \refstepcounter{appendixsubsubsection}%
      \SetupAppendixSubSubSectionTitleNumberFormatString%
      \DisplaySubSubSectionHeader{%
        \GetAppendixSubSubSectionTitleNumberFormatString}{#2}
    }%
    {%
      \refstepcounter{subsubsection}%
      \SetupSubSubSectionTitleNumberFormatString%
      \DisplaySubSubSectionHeader{%
        \GetSubSubSectionTitleNumberFormatString}{#2}
    }%
  }%
} % End of \RenewDocumentCommand{}

% ----------------------------------------------------------------------------
% Chapters (對應\chapter{})
\DeclareDocumentCommand{\StartChapter}{+m +g}
{%
  \StartNewPage%
  \chapter{#1}%
  \pagestyle{plain}% Default page style
  \IfNoValueF{#2}{\label{#2}}%
} % End of \DeclareDocumentCommand{}

% Chapters Star (對應\chapter*{})
\DeclareDocumentCommand{\StartChapterStar}{+m +g}
{%
  \StartNewPage%
  \chapter*{#1}%
  \pagestyle{plain}% Default page style
  \IfNoValueF{#2}{\label{#2}}%
} % End of \DeclareDocumentCommand{}

\newcommand{\EndChapter}
{%
  \EndOfPage%
  \UseDefaultLineStretch%
} % End of \newcommand{}

% Section
\DeclareDocumentCommand{\StartSection}{+m +g}
{%
  \section{#1}%
  \IfNoValueF{#2}{\label{#2}}%
} % End of \DeclareDocumentCommand{}

% Sub-Section
\DeclareDocumentCommand{\StartSubSection}{+m +g}
{%
  \subsection{#1}%
  \IfNoValueF{#2}{\label{#2}}%
} % End of \DeclareDocumentCommand{}

% Sub-Sub-Section
\DeclareDocumentCommand{\StartSubSubSection}{+m +g}
{%
  \subsubsection{#1}%
  \IfNoValueF{#2}{\label{#2}}%
} % End of \DeclareDocumentCommand{}

% ----------------------------------------------------------------------------

% Chapter標題改使用 第1章 的方式來顯示
%\newcommand\VarChapterTitleLangChi{1}
%\newcommand\VarChapterTitleLangEng{0}
%\newcommand\VarChapterTitleLang{\VarChapterTitleLangEng}
%\newcommand\GetChapterTitleLang{\VarChapterTitleLang}
%\newcommand{\ChapterTitleInChi}{\renewcommand{\VarChapterTitleLang}{\VarChapterTitleLangChi}}

% A wrapper to handle \ChapterSectionTitleInChi
%\newcommand{\ChapterSectionTitleInChi}
%{
%  \ChapterTitleInChi
%} % End of \newcommand{}

% ----------------------------------------------------------------------------

% 過去的API, 以 Error提醒不能再使用
\newcommand{\ChapterTitleNumInChi}{\errmessage{模版: 由v1.4.1開始, ChapterTitleNumInChi已不能再使用. 請參考最新版的conf.tex使用方式.}\stop}

\newcommand{\ChapterTitleInChi}{\errmessage{模版: 由v1.4.4開始, ChapterTitleInChi已不能再使用. 請參考最新版的conf.tex使用方式.}\stop}

\newcommand{\ChapterSectionTitleInChi}{\errmessage{模版: 由v1.4.4開始, ChapterSectionTitleInChi已不能再使用. 請參考最新版的conf.tex使用方式.}\stop}

% ----------------------------------------------------------------------------
