% ------------------------------------------------
\StartChapter{Introduction}{chapter:introduction}
% ------------------------------------------------

\StartSection{介紹}

這是國立成功大學碩博士用畢業論文的LaTex模板. 本模板是使用學校最新的畢業論文要求來設計(參考: 附錄 - 撰寫論文須知 P.\RefPage{appendix:thesis-spec}).

雖然本模板的目標是為了提供學生可以使用LaTex來寫畢業論文. 但是各系所有各自的格式, 所以做了一個表列出已知的系所情況(參考: 附錄 - 可使用的系所 P.\RefPage{appendix:acceptable-dept}), 故請在使用前先留意自己的系所有沒有格式要求. 如果沒有, 則本模板應該是可以用來使用; 否則要看系所上的格式, 是否跟本模板有相同的寫法.

本模板分以下幾個主要部份來進行教學:
\begin{enumerate}
  \item 本模板的架構設計
  \item 設定本模板的一些資料以轉成你的論文
  \item 介紹Latex和本模板所提供的語法
  \item 最後有一個chapter為"老師們的話"(Chap. \RefTo{chapter:words-from-teacher})寫了一些老師對論文的想法和意見, 以供同學們留意
\end{enumerate}
同學們只要閱讀完後, 把部份的檔案直接copy和修改內容, 應該很快就能上手本模板去寫自己的論文.

另外在附錄(appendix)附上了一些重要的學校的文件, 由於本模板很接近完善, 故直接使用本模板後可不需再閱過學校相關規定之文件, 所以該類文件置於此僅為備考用.

% ------------------------------------------------
%\newpage

\begin{description}
  \item[版權 License] \hfill \\
  \InsertImage
    [scale=0.8, align=center,
      caption={CC Attribution-NonCommercial-ShareAlike License},
      label={fig:appendix:by-nc}]
    {./example/introduction/pic/by-nc-sa.png}

    本著作(ncku-thesis-templete-latex\RefBib{web:this-project:github})採用創用 CC 姓名標示-非商業性-相同方式分享 4.0 授權條款.

    This work(ncku-thesis-templete-latex\RefBib{web:this-project:github}) is licensed under Creative Commons Attribution-NonCommercial-ShareAlike 4.0 International License.

  詳細請看'LICENSE'這檔案中的條款說明.

  \item[版本修改 ChangeLog] \hfill \\
    \begin{description}
      \item[v1.1.6] 刪除'Induection'和'README.md'中的'版本 Version'

      \item[v1.1.5] 修正每個Chapter的第一頁的頁碼位置跟其他頁面不同的問題(\href{https://github.com/wengan-li/ncku-thesis-templete-latex/issues/1}{Issue \#1})

      \item[v1.1.4] 修正目錄自己沒有在目錄的Linking中出現

      \item[v1.1.3] 修正README.md中內容的位置錯誤

      \item[v1.1.2] \hfill \\
        \begin{enumerate}
          \item 重寫有關figure API的code, 增加和優化那些功能 (如增加align)
          \item 更新README.md的內容
          \item 增加ChangeLog
        \end{enumerate}

      \item[v1.1.1] \hfill \\
        \begin{enumerate}
          \item 把'Abstract'的中文版本是以'摘要'來顯示
          \item 修改和改良有關oral文件的一些path位置
        \end{enumerate}

      \item[v1.1.0] \hfill \\
        \begin{enumerate}
          \item 增加版權資料到一些核心檔案
          \item 修改和增加一些圖書館要求的內容
          \item 修改有關abstract的一些path位置
          \item 正式得到學校有關部門對這模板的接受
        \end{enumerate}

      \item[v1.0.1] 修改少量錯誤的內容和URL連接

      \item[<= v1.0.0] 正式完成版本
    \end{description}
\end{description}

% ------------------------------------------------
\EndChapter
% ------------------------------------------------
