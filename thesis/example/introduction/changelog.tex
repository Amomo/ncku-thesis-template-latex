% ------------------------------------------------
\begin{description}
  \item[v1.3.0] 重大改版 (如果是使用升級方式, 請注意以下所修改的部份有沒有影響自身的版本)\hfill
    \begin{description}
      \item[其他] \hfill
        \begin{enumerate}
          \item 更新CONTRIBUTE中的名單和使用的稱號
        \end{enumerate}
    \end{description}

  \item[v1.2.8] 修正日期在英文書脊中, 會因月份文字的長度而影響位置不一樣的問題

  \item[v1.2.7] \hfill
    \begin{enumerate}
      \item 增加可放置論文題目的長度. 修正在封面和Oral文件的樣板中, 會在題目沒有很長情況下, 被強迫斷行. 長度控制交由同學自己斷行, 以造出比較漂亮題目
      \item 修正書脊中題目跟學位不是同一個高度的問題
      \item 修正英文Oral文件的樣板會出現頁碼的問題
    \end{enumerate}

  \item[v1.2.5] 修正在'Objective'和'Acknowledgments'的錯誤內容

  \item[v1.2.4] 增加英文封面可同時顯示中英文 (\href{https://github.com/wengan-li/ncku-thesis-templete-latex/issues/3}{Issue \#3})

  \item[v1.2.3] \hfill
    \begin{enumerate}
      \item 修正統一使用'Fig'去取代'Fig.', 因為當使用'Fig.'時會產生更大的空格
      \item 修正在'表格 Table'中的圖片位置
      \item 移除在'圖片 Image'的'多張'中舊API的說明文字
      \item 修正在'圖片 Image'中插入多張的圖片時, 不管是主圖或子圖片都推薦使用'align = center'來進行置中, 除非是為了特殊的原因
    \end{enumerate}

  \item[v1.2.2] 修正在'Induection'中的'ChangeLog'和'License'中一些奇怪多餘的空白

  \item[v1.2.1] 修正中文書脊文字位置錯誤問題

  \item[v1.2.0] \hfill
    \begin{enumerate}
      \item Appendix新增'常見問題Q\&A'
      \item 把'Induection'中的'ChangeLog'改使用為單一'.tex'檔去存放
      \item 增加字眼'共同指導'或'Co-advisor'在封面上 (\href{https://github.com/wengan-li/ncku-thesis-templete-latex/issues/2}{Issue \#2})
      \item 重新調整中文封面中的中英文名字2邊的中間空間的大小, 以防止中文名字有4個字時, 出現overlap的問題
    \end{enumerate}

  \item[v1.1.6] 刪除'Induection'和'README.md'中的'版本 Version'

  \item[v1.1.5] 修正每個Chapter的第一頁的頁碼位置跟其他頁面不同的問題 (\href{https://github.com/wengan-li/ncku-thesis-templete-latex/issues/1}{Issue \#1})

  \item[v1.1.4] 修正目錄自己沒有在目錄的Linking中出現

  \item[v1.1.3] 修正README.md中內容的位置錯誤

  \item[v1.1.2] \hfill
    \begin{enumerate}
      \item 重寫有關figure API的code, 增加和優化那些功能 (如增加align)
      \item 更新README.md的內容
      \item 增加ChangeLog
    \end{enumerate}

  \item[v1.1.1] \hfill
    \begin{enumerate}
      \item 把'Abstract'的中文版本是以'摘要'來顯示
      \item 修改和改良有關oral文件的一些path位置
    \end{enumerate}

  \item[v1.1.0] \hfill
    \begin{enumerate}
      \item 增加版權資料到一些核心檔案
      \item 修改和增加一些圖書館要求的內容
      \item 修改有關abstract的一些path位置
      \item 正式得到學校有關部門對這模板的接受
    \end{enumerate}

  \item[v1.0.1] 修改少量錯誤的內容和URL連接

  \item[<= v1.0.0] 正式完成版本
\end{description}

% ------------------------------------------------
