% ------------------------------------------------
\begin{description}
  \item[v1.3.0] \textbf{重大改版}\\
    由於所修改的內容影響全部內容和排版, 故比較推薦以重新編寫的方式來升級.\hfill
    \begin{description}
      \item[排版] \hfill
        \begin{enumerate}
          \item 使用原本'utdiss.sty'來重構'ncku.sty'. 保留有用的內容, 其他都盡量刪去.
          \item 修正內頁邊界錯誤, 原本排版為大約上3.8cm、下4.3cm(含頁碼)、左3.5cm、右3.4cm. 現修正為上2.3cm、下3.5cm(含頁碼)、左2.5cm、右3cm, 以符合學校的格式.
          \item 更新封面邊界的使用方式, 產出效果跟舊版效果是一樣
          \item 調整Chapter和Section的字體大小, 除了Chapter字體比較大, Section跟內容的字體是一樣, 但以粗體來顯示
          \item 更新Acknowledgments的標題位置
          \item 更新Abstract的標題位置
        \end{enumerate}
      \item[封面] \hfill
        \begin{enumerate}
          \item 更改 '學生' -> '研究生'
          \item 更改 '教授' -> '博士', 'Prof.' -> 'Dr.' (因為要去除職稱上的差別)
          \item 修正錯字, 'Co-advisor' -> 'Co-Advisor'
          \item 更正 'Master's Dissertation' -> 'Master's Thesis'
          \item 增加內頁. 封面主要用在印刷版, 如精裝版 或 平裝版. 而內頁主要用在電子版 + 印刷版. 在'context.tex'中使用'\verb|\DisplayInsideCover|'來使用.
          \item 更名API 'SetThesisDate' -> 'SetCoverDate', 底層轉到'\verb|\SetCoverDate|'來保留這API
          \item 更新在conf.tex和編寫介紹中, 有關封面日期設定的說明.
          \item 更新在context.tex中, 有關要使用哪種封面的說明.
        \end{enumerate}
      \item[書脊] \hfill
        \begin{enumerate}
          \item 移除書脊功能, 移除任何相關檔案和說明. 基於有影印店說, 就算我們有提供書脊檔案給他們, 他們都會自己使用一些工具重新弄一個書脊出來以給影印機所印出來, 故模板不再需要提供書脊功能.
        \end{enumerate}
      \item[功能] \hfill
        \begin{enumerate}
          \item 增加可設定 '(初稿)' (中文版) 和 '(Draft)' (英文版) 在封面. 在conf.tex中使用'\verb|\DisplayDraft|'來使用.
        \end{enumerate}
      \item[Appendix] \hfill
        \begin{enumerate}
          \item 更新2015版的 '口試注意事項' 和 '學位論文上傳說明'
          \item 補上引用文件的URL
        \end{enumerate}
      \item[目錄] \hfill
        \begin{enumerate}
          \item 删除 '封面' 和 '口試証明文件' 出現在目錄
          \item 更正 '致謝' 在目錄顯示正確
          \item 更正 '摘要' 在目錄顯示正確
          \item 目錄使用新的style以壓縮內容
          \item 目錄可在conf.tex中使用'\verb|\IndexChiMode|'或'\verb|\IndexEngMode|'來控制所顯示的標題的文字語言.
          \item 更新'\verb|\DisplayIndex|', 並新增'\verb|\DisplayTablesIndex'| 和 '\verb|\DisplayFiguresIndex|' 在'context.tex'以控制需要顯示的索引內容, 以免得沒有相關的內容, 但多了一頁沒意義的索引頁.
          \item 提供'\verb|\SetIndexTitleText|', '\verb|\SetTablesIndexTitleText|' 和 '\verb|\SetFiguresIndexTitleText|'在conf.tex以讓同學們可以自行設定目錄中的標題文字.
        \end{enumerate}
      \item[摘要] \hfill
        \begin{enumerate}
          \item 修正英文顯示 'Key words' -> 'Keywords'
          \item 更名API 'StartChiAbstract' -> 'StartAbstractChi', 底層轉到'\verb|\StartAbstractChi|'來保留這API
          \item 更名API 'EndChiAbstract' -> 'EndAbstractChi', 底層轉到'\EndAbstractChi'來保留這API
        \end{enumerate}
      \item[其他] \hfill
        \begin{enumerate}
          \item 更新CONTRIBUTE中的名單和使用的稱號
          \item 修正檔名, 應該是'misc.bib', 而不是'msic.bib' (\href{https://github.com/wengan-li/ncku-thesis-template-latex/issues/4}{Issue \#4})
          \item 修正錯字 'Templete' -> 'Template', 受影響的API為 '\verb|\DisplayOralTemplate|' (原為 '\verb|\DisplayOralTemplete|', 底層轉到'\verb|\DisplayOralTemplate|'來保留這API) (\href{https://github.com/wengan-li/ncku-thesis-template-latex/issues/7}{Issue \#7})
          \item 修正封面和口試証明上的日期因轉換時造成的奇怪空格.
          \item 更新README.md中, CC Logo改使用HTML方式來對齊
          \item 更新README.md中, 畢業論文要求補上引用的URL
          \item 更正 '資訊工程系' -> '資訊工程研究所'
          \item 修正引用的API '\verb|\RefXXX|' 系列所引用的內容前面會有多餘的空白
        \end{enumerate}
    \end{description}

  \item[v1.2.8] 修正日期在英文書脊中, 會因月份文字的長度而影響位置不一樣的問題

  \item[v1.2.7] \hfill
    \begin{enumerate}
      \item 增加可放置論文題目的長度. 修正在封面和Oral文件的樣板中, 會在題目沒有很長情況下, 被強迫斷行. 長度控制交由同學自己斷行, 以造出比較漂亮題目
      \item 修正書脊中題目跟學位不是同一個高度的問題
      \item 修正英文Oral文件的樣板會出現頁碼的問題
    \end{enumerate}

  \item[v1.2.5] 修正在'Objective'和'Acknowledgments'的錯誤內容

  \item[v1.2.4] 增加英文封面可同時顯示中英文 (\href{https://github.com/wengan-li/ncku-thesis-template-latex/issues/3}{Issue \#3})

  \item[v1.2.3] \hfill
    \begin{enumerate}
      \item 修正統一使用'Fig'去取代'Fig.', 因為當使用'Fig.'時會產生更大的空格
      \item 修正在'表格 Table'中的圖片位置
      \item 移除在'圖片 Image'的'多張'中舊API的說明文字
      \item 修正在'圖片 Image'中插入多張的圖片時, 不管是主圖或子圖片都推薦使用'align = center'來進行置中, 除非是為了特殊的原因
    \end{enumerate}

  \item[v1.2.2] 修正在'Induection'中的'ChangeLog'和'License'中一些奇怪多餘的空白

  \item[v1.2.1] 修正中文書脊文字位置錯誤問題

  \item[v1.2.0] \hfill
    \begin{enumerate}
      \item Appendix新增'常見問題Q\&A'
      \item 把'Induection'中的'ChangeLog'改使用為單一'.tex'檔去存放
      \item 增加字眼'共同指導'或'Co-advisor'在封面上 (\href{https://github.com/wengan-li/ncku-thesis-template-latex/issues/2}{Issue \#2})
      \item 重新調整中文封面中的中英文名字2邊的中間空間的大小, 以防止中文名字有4個字時, 出現overlap的問題
    \end{enumerate}

  \item[v1.1.6] 刪除'Induection'和'README.md'中的'版本 Version'

  \item[v1.1.5] 修正每個Chapter的第一頁的頁碼位置跟其他頁面不同的問題 (\href{https://github.com/wengan-li/ncku-thesis-template-latex/issues/1}{Issue \#1})

  \item[v1.1.4] 修正目錄自己沒有在目錄的Linking中出現

  \item[v1.1.3] 修正README.md中內容的位置錯誤

  \item[v1.1.2] \hfill
    \begin{enumerate}
      \item 重寫有關figure API的code, 增加和優化那些功能 (如增加align)
      \item 更新README.md的內容
      \item 增加ChangeLog
    \end{enumerate}

  \item[v1.1.1] \hfill
    \begin{enumerate}
      \item 把'Abstract'的中文版本是以'摘要'來顯示
      \item 修改和改良有關oral文件的一些path位置
    \end{enumerate}

  \item[v1.1.0] \hfill
    \begin{enumerate}
      \item 增加版權資料到一些核心檔案
      \item 修改和增加一些圖書館要求的內容
      \item 修改有關abstract的一些path位置
      \item 正式得到學校有關部門對這模板的接受
    \end{enumerate}

  \item[v1.0.1] 修改少量錯誤的內容和URL連接

  \item[<= v1.0.0] 正式完成版本
\end{description}

% ------------------------------------------------
