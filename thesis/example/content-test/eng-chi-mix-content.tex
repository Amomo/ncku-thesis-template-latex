% ------------------------------------------------

\newpage
\StartSection{混合的中英文內容 (只用段落來分段)}
用來看內容, 符號, 段距, 字元之間的距離等東西\\

Google公司(英語:Google Inc.; 中文:穀歌[3]、穀歌[4]、科高[5]), 是一家美國的跨國科技企業, 業務範圍涵蓋互聯網搜索、雲計算、廣告技術等領域, 開發並提供大量基於互聯網的產品與服務[6], 其主要利潤來自於AdWords等廣告服務[7][8].

\begin{description}
\item [Google] Google由在斯坦福大學攻讀理工博士的拉裡•佩奇和謝爾蓋•布林共同創建, 因此兩人也被稱為``Google Guys''[9][10][11]. 1998年9月4日, Google以私營公司的形式創立, 目的是設計並管理互聯網搜尋引擎``Google搜索''. 2004年8月19日, Google公司在納斯達克上市, 後來被稱為``三駕馬車''的公司兩位共同創始人與出任首席執行官的埃裡克•施密特在此時承諾:共同在Google工作至少二十年, 即至2024年止[12].

\item [Google]\hfill\\ Google由在斯坦福大學攻讀理工博士的拉裡•佩奇和謝爾蓋•布林共同創建, 因此兩人也被稱為``Google Guys''[9][10][11]. 1998年9月4日, Google以私營公司的形式創立, 目的是設計並管理互聯網搜尋引擎``Google搜索''. 2004年8月19日, Google公司在納斯達克上市, 後來被稱為``三駕馬車''的公司兩位共同創始人與出任首席執行官的埃裡克•施密特在此時承諾:共同在Google工作至少二十年, 即至2024年止[12].

\item Google的宗旨是``整合全球範圍的資訊, 使人人皆可訪問並從中受益''(To organize the world's information and make it universally accessible and useful)[13]; 而非正式的口號則為``不作惡''(Don't be evil), 由工程師阿米特•派特爾(Amit Patel)所創[14], 並得到了保羅•布赫海特的支持[15][16]. Google公司的總部稱為``Googleplex'', 位於美國加州聖克拉拉縣的山景城. 2011年4月, 佩奇接替施密特擔任首席執行官[17].

\item 在2015年8月, Google進行宣佈資產重組. 重組後, Google劃歸新成立的Alphabet底下. 同時, 此舉把Google旗下的核心搜索和廣告業務與Google無人車等新興業務分離開來[18].
\end{description}

據估計, Google在全世界的資料中心內運營著上百萬台的伺服器, [19]每天處理數以億計的搜索請求[20]和約二十四PB使用者生成的資料. [21][22][23][24] Google自創立起開始的快速成長同時也帶動了一系列的產品研發、並購事項與合作關係, 而不僅僅是公司核心的網路搜索業務. Google公司提供豐富的線上軟體服務, 如雲端硬碟、Gmail電子郵件, 包括Orkut、Google Buzz以及Google+在內的社交網路服務. Google的產品同時也以應用軟體的形式進入使用者桌面, 例如Google Chrome網頁流覽器、Picasa圖片整理與編輯軟體、Google Talk即時通訊工具等. 另外, Google還進行了移動設備的Android作業系統以及Google Chrome OS作業系統的開發. [25]

資訊分析網站Alexa資料顯示, Google的主功能變數名稱google.com是全世界訪問量最高的網站, Google搜索在其他國家或地區域名下的多個網站(google.co.in、google.de、google.com.hk等等), 及旗下的YouTube、Blogger、Orkut等的訪問量都在前一百名之內. [26]其中, 社交網路服務Orkut於2014年9月關閉. [27]

Facebook(原本稱作thefacebook)是一家位於美國加州聖馬刁郡門洛派克市的線上社交網路服務網站. 其名稱的靈感來自美國高中提供給學生包含照片和聯絡資料的通訊錄(或稱花名冊)暱稱「face book」[6][7].

除了文字訊息之外, 使用者可傳送圖片、影片和聲音媒體訊息(現在也可以傳送其他檔案類型如.doc,.docx,.xls,.xlsx等, 但是.exe可能會被禁止傳送)給其他使用者, 以及透過整合的地圖功能分享使用者的所在位置. Facebook是在2004年2月4日由馬克•紮克伯格與他的哈佛大學室友們所創立[8]. Facebook的會員最初只限於哈佛學生加入, 但後來逐漸擴展到其他在波士頓區域的同學也能使用, 包括一些常春藤名校、MIT、紐約大學、史丹福大學等. 接著逐漸支援讓其他大學和高中學生加入, 並在最後開放給任何13歲或以上的人使用.  現在Facebook允許任何聲明自己年滿13歲的使用者註冊[9].

使用者必須註冊才能使用Facebook, 註冊後他們可以創建個人檔案、將其他使用者加為好友、傳遞訊息, 並在其他使用者更新個人檔案時獲得自動通知. 此外使用者也可以加入有相同興趣的群組, 這些群組依據工作地點、學校或其他特性分類. 使用者亦可將朋友分別加入不同的列表中管理, 例如「同事」或「摯友」等. 截至2012年9月, Facebook內已有超過十幾億個活躍使用者[10], 其中約有9\%的不實使用者[11]. 截至2012年, Facebook每年共產生180拍位元組(PB)的資料, 並以每24小時0.5拍位元元組的速度增加[12]. 統計顯示, Facebook上每天上傳3億5千萬張圖片. [13]

Facebook創始人馬克•紮克伯格是世界上最著名的CEO之一. 而馬克•紮克伯格曾經的朋友與商業合作夥伴愛德華多•薩維林在新加坡亦十分知名[14].

Google公司(英語:Google Inc.; 中文:穀歌[3]、穀歌[4]、科高[5]), 是一家美國的跨國科技企業, 業務範圍涵蓋互聯網搜索、雲計算、廣告技術等領域, 開發並提供大量基於互聯網的產品與服務[6], 其主要利潤來自於AdWords等廣告服務[7][8].

\begin{itemize}
\item Google由在斯坦福大學攻讀理工博士的拉裡•佩奇和謝爾蓋•布林共同創建, 因此兩人也被稱為``Google Guys''[9][10][11]. 1998年9月4日, Google以私營公司的形式創立, 目的是設計並管理互聯網搜尋引擎``Google搜索''. 2004年8月19日, Google公司在納斯達克上市, 後來被稱為``三駕馬車''的公司兩位共同創始人與出任首席執行官的埃裡克•施密特在此時承諾:共同在Google工作至少二十年, 即至2024年止[12].

\item Google的宗旨是``整合全球範圍的資訊, 使人人皆可訪問並從中受益''(To organize the world's information and make it universally accessible and useful)[13];
\begin{itemize}
\item Google的宗旨是``整合全球範圍的資訊, 使人人皆可訪問並從中受益''(To organize the world's information and make it universally accessible and useful)[13]; 而非正式的口號則為``不作惡''(Don't be evil), 由工程師阿米特•派特爾(Amit Patel)所創[14], 並得到了保羅•布赫海特的支持[15][16]. Google公司的總部稱為``Googleplex'', 位於美國加州聖克拉拉縣的山景城. 2011年4月, 佩奇接替施密特擔任首席執行官[17].
\item Google的宗旨是``整合全球範圍的資訊, 使人人皆可訪問並從中受益''(To organize the world's information and make it universally accessible and useful)[13].
\item Google公司的總部稱為``Googleplex'', 位於美國加州聖克拉拉縣的山景城. 2011年4月, 佩奇接替施密特擔任首席執行官[17].
\end{itemize}

\item 在2015年8月, Google進行宣佈資產重組. 重組後, Google劃歸新成立的Alphabet底下. 同時, 此舉把Google旗下的核心搜索和廣告業務與Google無人車等新興業務分離開來[18].
\end{itemize}

據估計, Google在全世界的資料中心內運營著上百萬台的伺服器, [19]每天處理數以億計的搜索請求[20]和約二十四PB使用者生成的資料. [21][22][23][24] Google自創立起開始的快速成長同時也帶動了一系列的產品研發、並購事項與合作關係, 而不僅僅是公司核心的網路搜索業務. Google公司提供豐富的線上軟體服務, 如雲端硬碟、Gmail電子郵件, 包括Orkut、Google Buzz以及Google+在內的社交網路服務. Google的產品同時也以應用軟體的形式進入使用者桌面, 例如Google Chrome網頁流覽器、Picasa圖片整理與編輯軟體、Google Talk即時通訊工具等. 另外, Google還進行了移動設備的Android作業系統以及Google Chrome OS作業系統的開發. [25]

資訊分析網站Alexa資料顯示, Google的主功能變數名稱google.com是全世界訪問量最高的網站, Google搜索在其他國家或地區域名下的多個網站(google.co.in、google.de、google.com.hk等等), 及旗下的YouTube、Blogger、Orkut等的訪問量都在前一百名之內. [26]其中, 社交網路服務Orkut於2014年9月關閉. [27]

Facebook(原本稱作thefacebook)是一家位於美國加州聖馬刁郡門洛派克市的線上社交網路服務網站. 其名稱的靈感來自美國高中提供給學生包含照片和聯絡資料的通訊錄(或稱花名冊)暱稱「face book」[6][7].
\begin{enumerate}
\item 除了文字訊息之外, 使用者可傳送圖片、影片和聲音媒體訊息(現在也可以傳送其他檔案類型如.doc,.docx,.xls,.xlsx等, 但是.exe可能會被禁止傳送)給其他使用者, 以及透過整合的地圖功能分享使用者的所在位置.

\item Facebook是在2004年2月4日由馬克•紮克伯格與他的哈佛大學室友們所創立[8]. Facebook的會員最初只限於哈佛學生加入, 但後來逐漸擴展到其他在波士頓區域的同學也能使用, 包括一些常春藤名校、MIT、紐約大學、史丹福大學等.

\item 接著逐漸支援讓其他大學和高中學生加入, 並在最後開放給任何13歲或以上的人使用.  現在Facebook允許任何聲明自己年滿13歲的使用者註冊[9].
\end{enumerate}
使用者必須註冊才能使用Facebook, 註冊後他們可以創建個人檔案、將其他使用者加為好友、傳遞訊息, 並在其他使用者更新個人檔案時獲得自動通知. 此外使用者也可以加入有相同興趣的群組, 這些群組依據工作地點、學校或其他特性分類. 使用者亦可將朋友分別加入不同的列表中管理, 例如「同事」或「摯友」等. 截至2012年9月, Facebook內已有超過十幾億個活躍使用者[10], 其中約有9\%的不實使用者[11]. 截至2012年, Facebook每年共產生180拍位元組(PB)的資料, 並以每24小時0.5拍位元元組的速度增加[12]. 統計顯示, Facebook上每天上傳3億5千萬張圖片. [13]

Facebook創始人馬克•紮克伯格是世界上最著名的CEO之一. 而馬克•紮克伯格曾經的朋友與商業合作夥伴愛德華多•薩維林在新加坡亦十分知名[14].

% ------------------------------------------------
