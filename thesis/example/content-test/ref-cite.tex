% ------------------------------------------------
\StartChapter{測試Reference/Cite}{testpage:ref-cite:chapter}
% ------------------------------------------------

本校創校於西元1931年(昭和6年,民國20年)1月15日,原名為「臺南高等工業學校」;1944年(昭和19年,民國33年)改稱為「臺南工業專門學校」。民國34年臺灣光復。本校於民國35年2月改制為「臺灣省立臺南工業專科學校」,由王石安博士擔任校長;35年10月改制為「臺灣省立工學院」,仍由王石安博士擔任校長。彼時僅有成功校區,39年增購勝利校區。41年2月,由秦大鈞博士接任校長。

\StartSection{Section}{testpage:ref-cite:section}
45年8月,本校改制為「臺灣省立成功大學」,仍由秦大鈞博士擔任校長;同時增設文理學院及商學院。46年8月,由閻振興博士接任校長。54年1月,由羅雲平博士接任校長。55年增購光復校區。58年10月,將文理學院分為文學院及理學院。60年8月,改制為「國立成功大學」,並由倪超博士接任校長;同年增購建國校區。67年8月,由王唯農博士接任校長。

\StartSubSection{SubSection}{testpage:ref-cite:subsection}
69年8月,夏漢民博士接任校長;同年將商學院更名為管理學院。72年8月,增設醫學院,並增購自強校區及敬業校區。74年增購力行校區部分校地。76年增購歸仁校區,設置航空太空實驗場。77年6月本校醫學院附設醫院正式營運。77年8月,馬哲儒博士接任校長。80年增購自強校區北半部。82年陸續增購台南市「文大五」用地,闢為本校安南校區。

\StartSubSubSection{SubSubSection}{testpage:ref-cite:subsubsection}
96年2月,賴明詔院士接任校長,97年2月教育部公佈「發展國際一流大學及頂尖研究中心計畫」第二梯次的審議結果,本校繼續獲得教育部的肯定與補助,積極朝國際一流大學的目標邁進。97年10月增購歸仁校區北側台糖土地(正式登記為本校管有)。100年2月,黃煌煇博士接任校長。100年4月本校獲得教育部第二期頂尖大學計畫補助,持續朝國際一流大學的目標邁進。104年2月,蘇慧貞博士接任校長。


\InsertFigure
  [scale=0.5,
    caption={Ref用figure}, label={testpage:ref-cite:figure}]
  {./example/abstract/pic/extended-abstract-2.jpg}

\InsertTable
  [caption={Ref用table}, label={testpage:ref-cite:table}]
  {
    \begin{tabular}{llll}
    \hline
    Engine &  &  & OPEL Astra C16SE \\ \hline
    Displacement (cc) &  &  & 1598 \\
    Bore x stroke(mm x mm) &  &  & 79 x 81.5 \\
    Value mechanism &  &  & SOHC \\
    Number of valves &  &  & Intake 4, exhaust 4 \\
    Compression ratio &  &  & 9.8:1 \\
    Torque &  &  & 135/3400 Nm/rpm \\
    Power &  &  & 74/5800 kW/rpm \\
    Ignition sequence &  &  & 1-3-4-2 \\
    Spark plug &  &  & BPR6ES \\
    Fuel &  &  & 95 unleaded gasoline \\
    Cylinder arrangment &  &  & In-line 4 cylinders \\ \hline
    \end{tabular}
  } % End of  \InsertTable{}

\EquationBegin{testpage:ref-cite:eq}E = mc^2\EquationEnd

% ------------------------------------------------
\EndChapter
% ------------------------------------------------

\StartChapter{測試 RefXXX}

chapter (\verb|\RefTo{}|): \RefTo{testpage:ref-cite:chapter}\\
chapter (\verb|\ref*{}|): \ref*{testpage:ref-cite:chapter}\\

section (\verb|\RefTo{}|): \RefTo{testpage:ref-cite:section}\\
section (\verb|\ref*{}|): \ref*{testpage:ref-cite:section}\\

subsection (\verb|\RefTo{}|): \RefTo{testpage:ref-cite:subsection}\\
subsection (\verb|\ref*{}|): \ref*{testpage:ref-cite:subsection}\\

subsubsection (\verb|\RefTo{}|): \RefTo{testpage:ref-cite:subsubsection}\\
subsubsection (\verb|\ref*{}|): \ref*{testpage:ref-cite:subsubsection}\\

figure (\verb|\RefFigure{}|): \RefFigure{testpage:ref-cite:figure}\\
figure (\verb|\ref*{}|): \ref*{testpage:ref-cite:figure}\\

table (\verb|\RefTable{}|): \RefTable{testpage:ref-cite:table}\\
table (\verb|\ref*{}|): \ref*{testpage:ref-cite:table}\\

equation (\verb|\RefEquation{}|): \RefEquation{testpage:ref-cite:eq}\\
equation (\verb|\ref*{}|): \ref*{testpage:ref-cite:eq}\\

equation (\verb|\RefEquationB{}|): \RefEquationB{testpage:ref-cite:eq}\\
equation (\verb|\eqref{}|): \eqref{testpage:ref-cite:eq}\\

\begin{comment}
\end{comment}
% ------------------------------------------------
\EndChapter
% ------------------------------------------------
