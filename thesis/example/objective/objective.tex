% ------------------------------------------------
\StartChapter{Objective}{chapter:objective}
% ------------------------------------------------

\section{起因}

做這個模板的原因其實很簡單:

\begin{enumerate}
  \item
  {
    去投國外paper時, 對方可能會要求使用LaTeX, 所以未來要懂LaTeX是不意外的.
  } % End of \item{}

  \item
  {
    想拿LaTeX來寫畢業論文, 卻發現學校只提供Mircosoft Word模板, 但卻沒有提供LaTeX的, 所以證明本模板對學校是有存在價值的.
  } % End of \item{}

  \item
  {
    因為看到發現台灣科技大學\cite{web:latex:template:ntust}, 台灣大學\cite{web:latex:template:ntu}, 元智大學\cite{web:latex:template:ntust}都能找到LaTeX的模板, 連大陸那邊都有一些學校有在提供.

    那些學校的畢業論文模板不只提供是Mircosoft Word版本(.doc), 是會連Latex(.tex)版本都有, 而我們學校卻沒有. 唯一我們學校在Google上找到的有提到的卻是數學系系網頁上的功能\cite{web:latex:ncku_math_introduction}和建在數學系上的一個討論區\cite{web:latex:ncku_math_forum}.
  } % End of \item{}

  \item
  {
    因為學校對Phd跟Master的畢業論文要求是同一個格式, 所以如果完成後對學校任何學生應該都有其好處.

    對大家都有多一個選擇來寫畢業論文, 而不是被限在使用Mircosoft Word來寫.
  } % End of \item{}

  \item
  {
    經過詢問我們資訊工程系(CSIE)的系上一些老師後, 意外發現原來某些實驗室其實已經有各自的版本存在(參考: Acknowledgment P.\pageref{chapter:acknowledgments-chi}), 但每個版本都有各自的優缺點.

    E.g :
    \begin{enumerate}

      \item
      {
        新的使用者或接手的人不容易修改或使用.
      } % End of \item{}

      \item
      {
        或是需要安裝的步驟十分麻煩 (e.g cwTeX\cite{web:latex:cwtex}).
      } % End of \item{}

      \item
      {
        另外有一些因為是只針對英文版本, 沒有考量在編寫或初稿時會有中英混雜的時候(同時因學校奇怪的要求, 例如英文內容的論文卻要寫中文論文名字等), 所以需要把整個論文分開成不同格式的檔案.
      } % End of \item{}

      \item
      {
        etc.
      } % End of \item{}
    \end{enumerate}
  } % End of \item{}
\end{enumerate}

% ------------------------------------------------

\section{目標}
所以為了解決以上的問題, 這個模板針對了好幾點來處理:

\begin{enumerate}

  \item
  {
    把本模板做到連笨蛋都可以很快懂得使用(所謂的Books for Dummies), 所以只留下使用者要填寫的部份外, 其他都交由模板去負責.
  } % End of \item{}

  \item
  {
    希望做到使用者只讀這份模板, 就會懂得去修改和寫自己所需的內容(所謂的Self-contained. 但其實是不太可能的, 因為Latex的使用手冊就算寫成一本幾百頁的書, 都可以缺少很多東西), 所以會同時提供很基本使用Latex的方式, 和填寫本模板步驟.
  } % End of \item{}

  \item
  {
    希望一份模板, 能同時應用在中文或是英文版本, 只要修改內容和一些的設定.
  } % End of \item{}

  \item
  {
    把本模板open source, 讓以後任何的同學們都可以使用和修改, 以合適當時的需求.
  } % End of \item{}

\end{enumerate}

而選擇使用XeLaTex的原因, 是我分析了cwTeX, CJK和XeLaTex.
cwTeX的寫法太糟, 要背多新一種語法, 而且安裝複雜\cite{web:latex:cwtex}; 而CJK有一定程度的設定才能在整個論文中自由使用, 感覺設定麻煩而不太能笨蛋化來用, 所以放棄選用; 故最後選用最簡單加一些包裝, 就可以簡單使用中英混合的XeLaTex.

% ------------------------------------------------

\section{缺點}
但是同樣任何東西都會有缺點, 故本模板都不意外:

\begin{enumerate}

  \item
  {
    本模板是以台灣國立成功大學所最新訂下的畢業論文要求(參考: 附錄 - 撰寫論文須知 P.\pageref{appendix:thesis-spec})來設計, 所以不一定能對非本校的人有用.
  } % End of \item{}

  \item
  {
    對沒有程式基礎, 只會用Mircosoft Word的人來講, 可能會在修改或使用上會十分吃力.
  } % End of \item{}

  \item
  {
    因為我針對某些使用者不用去接觸的部份, 進行了大量的包裝(Wrapper), 所以如果懂得Latex的人可能會覺得我破壞了Latex的語法. 但是本模板是針對笨蛋化和全自動, 我相信對不會的人來講, 才不管這問題 (如同一般理論派和應用派的差別).
  } % End of \item{}

  \item
  {
    某些包裝出來的語法, 可能會在一些情況下會產生衝突而令Latex不接受, 這時候有2種做法:
    \begin{enumerate}
      \item
      {
        不使用某些寫法, 例如已知的\begin{verbatim}\InsertImage和\InsertCenterImage\end{verbatim}沒法被包在Table, minipage或framebox中.
      } % End of \item{}

      \item
      {
        如真的要使用那些情況, 那只好自己真的不使用我的語法, 而直接去寫Latex原版的語法.
      } % End of \item{}
    \end{enumerate}
  } % End of \item{}
\end{enumerate}

% ------------------------------------------------

\section{總結}

以上是個人對這份模板的一些個人想法和起源.

如果以上的話都阻擋不了你想使用的話, 那歡迎翻到下一頁開始學習如何使用Latex或本模板, 同時都希望本模板能對你提供到幫助.

% ------------------------------------------------
\EndChapter
% ------------------------------------------------
