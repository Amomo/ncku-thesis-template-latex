% ------------------------------------------------
\StartChapter{Objective}{chapter:objective}
% ------------------------------------------------

\StartSection{起因}

做這個模版的原因其實很簡單:

\begin{enumerate}
  \item
  {
    去投國外paper時, 對方可能會要求使用LaTeX, 所以未來要懂LaTeX是不意外的.
  } % End of \item{}

  \item
  {
    想拿LaTeX來寫畢業論文, 卻發現學校只提供Mircosoft Word模版, 但卻沒有提供LaTeX的, 所以證明本模版對學校是有存在價值的.
  } % End of \item{}

  \item
  {
    因為看到發現台灣科技大學\RefBib{web:latex:template:ntust}, 台灣大學\RefBib{web:latex:template:ntu}, 元智大學\RefBib{web:latex:yzu}都能找到LaTeX的模版, 連大陸那邊都有一些學校有在提供, 更不用說國外的學校.

    那些學校的畢業論文模版不只提供是Mircosoft Word版本(.doc), 是會連LaTex(.tex)版本都有, 而我們學校卻沒有. 唯一我們學校在Google上找到的有提到的卻是數學系系網頁上的功能\RefBib{web:latex:ncku_math_introduction}和建在數學系上的一個討論區\RefBib{web:latex:ncku_math_forum}.
  } % End of \item{}

  \item
  {
    因為學校對Phd跟Master的畢業論文要求是同一個格式, 所以如果完成後對學校任何學生應該都有其好處.

    對大家都有多一個選擇來寫畢業論文, 而不是被限在使用Mircosoft Word來寫.
  } % End of \item{}

  \item
  {
    經過詢問我們資訊工程系(CSIE)的系上一些老師後, 意外發現原來某些實驗室其實已經有各自的版本存在, 但每個版本都有各自的優缺點, 例如:

    \begin{enumerate}

      \item
      {
        新的使用者或接手的人不容易修改或使用.
      } % End of \item{}

      \item
      {
        或是需要安裝的步驟十分麻煩 (e.g cwTeX\RefBib{web:latex:cwtex}).
      } % End of \item{}

      \item
      {
        另外有一些因為是只針對英文版本, 沒有考量在編寫或初稿時會有中英混雜的時候, 故這時候中英文的內容要分開編寫和產生 (學校又要求, 英文內容的論文要同時有中文論文名字等), 所以需要把整個論文分開成不同的檔案.
      } % End of \item{}
    \end{enumerate}
  } % End of \item{}
\end{enumerate}

% ------------------------------------------------

%\newpage
\StartSection{目標}
所以為了解決以上的問題, 這個模版針對了好幾點來處理:

\begin{enumerate}

  \item
  {
    把本模版做到連笨蛋都可以很快懂得使用(所謂的Books for Dummies), 所以只留下使用者要填寫的部份外, 其他都交由模版去負責.
  } % End of \item{}

  \item
  {
    希望做到使用者只讀這份模版, 就會懂得去修改和寫自己所需的內容(所謂的Self-contained. 但其實是不太可能的, 因為LaTex的使用手冊就算寫成一本幾百頁的書, 都可以缺少很多東西), 所以會同時提供很基本使用LaTex的方式, 和填寫本模版步驟.
  } % End of \item{}

  \item
  {
    希望一份模版, 能同時應用在中文或是英文版本, 只要修改內容和一些的設定.
  } % End of \item{}

  \item
  {
    把本模版open source, 讓以後任何的同學們都可以使用和修改, 以合適當時的需求.
  } % End of \item{}

\end{enumerate}

而選擇使用XeLaTex的原因, 是經過分析cwTeX, CJK和XeLaTex後. 發現cwTeX的寫法太糟, 要背多新一種語法, 而且安裝複雜\RefBib{web:latex:cwtex}; 而CJK有一定程度的設定才能在整個論文中自由使用, 感覺設定麻煩而不太能笨蛋化來用, 所以放棄選用; 故最後選用最簡單加一些包裝, 就可以簡單使用中英混合的XeLaTex.

% ------------------------------------------------

\StartSection{缺點}
但是同樣任何東西都會有缺點, 故本模版都不意外:

\begin{enumerate}

  \item
  {
    本模版是以台灣國立成功大學所最新訂下的畢業論文要求(參考: 附錄 - 撰寫論文須知 P.\RefPage{appendix:thesis-spec})來設計, 所以不一定能對非本校的人有用.
  } % End of \item{}

  \item
  {
    對沒有程式基礎, 只會用Mircosoft Word的人來講, 可能會在修改或使用上會十分吃力.
  } % End of \item{}

  \item
  {
    因為我針對某些使用者不用去接觸的部份, 進行了大量的包裝(Wrapping), 所以如果懂得LaTex的人可能會覺得我破壞了LaTex的語法. 但是本模版是針對笨蛋化和全自動, 我相信對不熟LaTex的人來講, 才不管這問題 (如同一般理論派和應用派的差別, 在意的方向完全不一樣).
  } % End of \item{}

%  \newpage
  \item
  {
    某些包裝出來的語法, 可能會在一些情況下會產生衝突而令LaTex不接受, 這時候有2種做法:
    \begin{enumerate}
      \item
      {
        不使用某些寫法, 例如已知的`\verb|\InsertFigure|'沒法被包在Table, minipage或framebox中.
      } % End of \item{}

      \item
      {
        如真的要使用那些情況, 那就不要使用模版提供的語法, 而直接去寫LaTex原版的語法.
      } % End of \item{}
    \end{enumerate}
  } % End of \item{}
\end{enumerate}

% ------------------------------------------------

\StartSection{總結}

以上是個人對這份模版的一些想法和起源, 同時希望本模版能對你提供到一些幫助.

% ------------------------------------------------
\EndChapter
% ------------------------------------------------
