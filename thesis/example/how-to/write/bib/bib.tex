% ------------------------------------------------
\StartSection{文獻引用 Bibliography/Reference}{chapter:how-to:write:bib}
% ------------------------------------------------

\StartSubSection{介紹}

Reference對論文來講十分重要的東西, 所以如果你引用的paper數量不少, 那在整理上會有點麻煩, 所以世界上有不少東西來管理這部份的資料, 如用的Word的話會配合Endnote.\\

而本模版是使用Latex中的BibTex來管理, 你可以在'./content/references'找到3個'.bib'檔, 那正是你可以把你所引用的內容放在裡面.\\

Bib的分類滿多 (參考\RefBib{web:latex:bib_manage}), 但論文主要都是引用'book' (課本, 書籍等), 'misc' (網頁, 任何其他東西), 'inproceedings' (論文類)中的內容, 所以本模版提供的樣板檔案為'book.bib', 'misc.bib' 跟 'paper.bib'.

\StartSubSection{使用方式}

任何放置論文的出版社(如ACM, IEEE, DBLP等), 都會為了方便別人去引用, 都會提供一些資料以給放在論文中引用. Fig \RefTo{fig:write:bib:1} 是以ACM Digital Library例子, 簡單說明如何使用BibTex來管理.

\InsertFigure
  [caption={ACM Digital Library例子},
    label={fig:write:bib:1}, scale=0.5]
  {./example/how-to/write/bib/pic/1.png}

\InsertFigure
  [caption={BibTex的位置},
    label={fig:write:bib:2}, scale=0.4]
  {./example/how-to/write/bib/pic/2.png}

在畫面右方會看到'Export Formats'的位置, 會看到如fig \RefTo{fig:write:bib:2}中一個的BibTex的按鈕.

\InsertFigure
  [caption={BibTex資料},
    label={fig:write:bib:3}, scale=0.5]
  {./example/how-to/write/bib/pic/3.png}

按它後就會出現如fig \RefTo{fig:write:bib:3}這個畫面, 這個就是要填進Bib的資料, 所以把這個東西複製到Bib檔內.

\InsertFigure
  [caption={整理/使用BibTex},
    label={fig:write:bib:4}, scale=0.5]
  {./example/how-to/write/bib/pic/4.png}

但複製完後要改一個東西, 第一行是所謂的label部份(參考Chap \RefTo{chapter:how-to:write:label}), 所以要改成一個自己能記得的label以方便在內容中來引用.

%有什麼問題可以去問Google\cite{website:google}老師. (如果有設定references用的檔案, 即使用了ReferencesFiles, 那必須至少要存在一個cite才不會顯示錯誤.)

% ------------------------------------------------
\EndChapter
% ------------------------------------------------
