% ------------------------------------------------
\StartSubSection{清單 List Structures}

  日常的清單主要有3種:

\begin{itemize}
  \item
  {
    數字

    可以有2種寫法, 使用\verb|\item xxxx|來只寫一行, 或是用\verb|{...}|可把內容包起來.
    \EmptyLine
    \begin{DescriptionFrame}
    \begin{verbatim}
      \begin{enumerate}
      \item Item1

      \item Item2

      \item
      {
        Item3

        Item3's context
      }

      \item
      {
        Item4

        Item4's context
      }
      \end{enumerate}
    \end{verbatim}
    \end{DescriptionFrame}
    \EmptyLine

    效果:
    \begin{enumerate}
      \item Item1

      \item Item2

      \item
      {
        Item3

        Item3's context
      }

      \item
      {
        Item4

        Item4's context
      }
    \end{enumerate}
  } % End of \item{}

  \newpage
  \item
  {
    符號

    \EmptyLine
    \begin{DescriptionFrame}
    \begin{verbatim}
      \begin{itemize}
      \item Item1

      \item Item2

      \item
      {
        Item3

        Item3's context
      }

      \item
      {
        Item4

        Item4's context
      }
      \end{itemize}
    \end{verbatim}
    \end{DescriptionFrame}
    \EmptyLine

    效果:
    \begin{itemize}
      \item Item1

      \item Item2

      \item
      {
        Item3

        Item3's context
      }

      \item
      {
        Item4

        Item4's context
      }
    \end{itemize}
  } % End of \item{}

  \newpage
  \item
  {
    文字

    可以有2種寫法, 使用\verb|\item[xxxx] xxxx|來只寫一行, \\
    或是用\verb|\hfill \\|把內容放到第2行才開始.

    \EmptyLine
    \begin{DescriptionFrame}
    \begin{verbatim}
      \begin{description}
      \item[Item1] Item1's context
      \item[Item2] Item2's context
      \item[Item3] \hfill \\
        Item3's context
      \end{description}
    \end{verbatim}
    \end{DescriptionFrame}
    \EmptyLine

    效果:
    \begin{description}
      \item[Item1] Item1's context
      \item[Item2] Item2's context
      \item[Item3] \hfill \\
      Item3's context
    \end{description}
  } % End of \item{}

  \newpage
  \item
  {
    巢狀表單

    表單應該最多只會用到第4層, 但是其實當你需要用到第3層時, 這時候你應該考慮的不是怎使用表單, 而是要怎換另外一種寫法了.

    \begin{DescriptionFrame}
    \begin{verbatim}
      \begin{enumerate}
        \item
        {
          Level-1 Item 1
          \begin{enumerate}
            \item Nested Item 1

            \item
            {
              Level-2 Item 2

              \begin{enumerate}
              \item
              {
                Level-3 Item 1
                \begin{enumerate}
                  \item Level-4 Item 1
                  \item Level-4 Item 2
                \end{enumerate}
              }
              \item Level-3 Item 2
              \end{enumerate}
            }
          \end{enumerate}
        }
      \end{enumerate}

      \begin{itemize}
        \item
        {
          Level-1 Item 1

          \begin{itemize}
            \item
            {
              Level-2 Item 2
              \begin{itemize}
                \item Level-3 Item 1
                \item Level-3 Item 2
              \end{itemize}
            }
            \item Level-2 Item 2
          \end{itemize}
        }
      \end{itemize}
    \end{verbatim}
    \end{DescriptionFrame}

    效果:
    \begin{enumerate}
      \item
      {
        Level-1 Item 1
        \begin{enumerate}
          \item Nested Item 1

          \item
          {
            Level-2 Item 2

            \begin{enumerate}
              \item
              {
                Level-3 Item 1

                \begin{enumerate}
                  \item Level-4 Item 1
                  \item Level-4 Item 2
                \end{enumerate}
              }

              \item Level-3 Item 2
            \end{enumerate}
          }
        \end{enumerate}
      }
    \end{enumerate}

    \begin{itemize}
      \item
      {
        Level-1 Item 1

        \begin{itemize}
        \item
        {
          Level-2 Item 2

          \begin{itemize}
          \item Level-3 Item 1
          \item Level-3 Item 2
          \end{itemize}
        }

        \item Level-2 Item 2
        \end{itemize}
      }
    \end{itemize}
  } % End of \item{}
\end{itemize}
% ------------------------------------------------
