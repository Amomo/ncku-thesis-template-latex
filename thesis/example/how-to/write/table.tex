% ------------------------------------------------
\StartChapter{表格 Table}{chapter:how-to:write:image}
% ------------------------------------------------
\section{介紹}

表格(Table)在任何情況下都是一個常用的顯示方式, 所以如何設計它都會有大量的玩法. 在正常Mircosoft Word這種有畫面的情況下, 可以慢慢拉出一個比較適合自己的, 但是在Latex中這個過程會是十分的痛苦, 因為你沒法馬上知道修改後的畫面, 故要不斷測試才知道效果, 這樣會大大減低選用table的使用次數.

在一般任何的Latex教學上, 如何編寫一個table出來都會是其中一項, 了解任何一個部份的寫法, 位置, 設定等. 但是由於那些資料十分的巨量 (不同寫法有不同效果), 所以這絕對不是使用本模板的大家想知道的東西, 故本模板不使用過往的方式, 而且直接教大家怎樣使用現有的online tool去處理掉這個問題.

以下的說明都是針對LaTeX Table Generator (\url{http://www.tablesgenerator.com/})\cite{web:latex:table-generator}來進行說明.

LaTeX Table Generator (Fig. \ref{fig:how-to:table:table-generator})的頁面非常明瞭和簡陋, 只要有過Mircosoft Word中的table設計的經驗, 應該要上手這個東西絕對不會很難.

\newpage
\InsertCenterImage
  [scale=0.4,
    caption={LaTeX Table Generator頁面},
    label={fig:how-to:table:table-generator}]
    {./example/how-to/write/pic/table/table-generator.png}

% ------------------------------------------------
\newpage
\section{產生Latex}

  我們使用這工具就是要去產生Latex用在論文當中, 所以這一步比其他的知識更為重要. 記得使用以下的步驟:

  \begin{enumerate}
    \item
    {
      使用畫面來設計table.
      \InsertCenterImage
        {./example/how-to/write/pic/table/table-view.png}
    } % End of \item{}

    \item
    {
      按Generate去產生Latex.
      \InsertCenterImage
        {./example/how-to/write/pic/table/generate.png}
    } % End of \item{}

    \item
    {
      複製Latex放到論文的".tex"中.
      \InsertCenterImage
        {./example/how-to/write/pic/table/latex-code.png}
    } % End of \item{}

    \item
    {
      執行XeLaTeX去產生效果.
    } % End of \item{}
  \end{enumerate}

  第1$\sim$3步會在整個設計table中常常都會使用, 所以會熟能生巧的. 而有經驗的人都知道, 第1步是最需要時間, 而第2$\sim$4步不用幾分鐘就能做完了, 所以只要用心的話, 多漂亮的table都是能弄出來的.

% ------------------------------------------------
\newpage
\section{功能}

要設計一個複雜的table就需要足夠的功能才能慢慢弄, 所以在這邊介紹一些算是非常有用的功能.

\subsection{File}

  在"File"中有幾個很有用的功能.
  \InsertCenterImage
    {./example/how-to/write/pic/table/menu-file.png}

  \begin{enumerate}

    % ------------------------------------------------
    \item
    {
      Import CSV file

      你可以直接upload一個CSV format的檔案之後弄table的外觀.
      \InsertCenterImage
        [scale=0.7]
        {./example/how-to/write/pic/table/csv.png}
    } % End of \item{}

    % ------------------------------------------------
    \newpage
    \item
    {
      Paste table data

      可以把Microsoft Excel的table直接做Copy \& Paste到這一邊來.
      \InsertCenterImage
        [scale=0.45]
        {./example/how-to/write/pic/table/paste.png}

      或是可以直接輸入資料來建立, 但要注意的是它只能接受CSV的寫法, 即是每一筆資料都是以","來分隔. 所以如果使用Fig \ref{fig:csv:enter-example-data}的寫法的話:
      \InsertCenterImage
        [scale=0.65,
          caption={Enter example data},
          label={fig:csv:enter-example-data}]
        {./example/how-to/write/pic/table/paste-data.png}

      會出現Fig \ref{fig:csv:result-example-data}的效果:
      \InsertCenterImage
        [scale=0.65,
          caption={Result of example data},
          label={fig:csv:result-example-data}]
        {./example/how-to/write/pic/table/paste-data-result.png}

    } % End of \item{}

    % ------------------------------------------------
    \newpage
    \item
    {
      Save table

      這online tool有一個十分有用的功能就是能把所做的table save下來, 只要輸入名字後再按download就會得到一個".tgn"檔案.
      \InsertCenterImage
        [scale=0.8]
        {./example/how-to/write/pic/table/save-table.png}

      \InsertCenterImage
        {./example/how-to/write/pic/table/save-tgn.png}

    } % End of \item{}

    % ------------------------------------------------
    \item
    {
      Load table

      在"Save table"中得到的".tgn"檔案就是使用這邊來重新讀取table.
      \InsertCenterImage
        [scale=0.8]
        {./example/how-to/write/pic/table/load-table.png}
    } % End of \item{}
  \end{enumerate}

\newpage
\subsection{Edit}

  在"Edit"中有2個常用的功能

  \InsertCenterImage
    {./example/how-to/write/pic/table/menu-edit.png}

  \begin{enumerate}

    \item
    {
      Undo / Repeat

      很基本的重做上一步/下一步所做過的行為, 故不用解釋什麼.
    } % End of \item{}

    \item
    {
      Autosave

      這功能十分有用, 因為這tool是網頁tool, 所以正常重開網頁時會令到資料不見. 所以如果有把"Autosave"開啟的話, 那table就算接了"F5"都不會不見. (預設上應該會自動有開啟)
      \InsertCenterImage
        {./example/how-to/write/pic/table/edit-autosave.png}
    } % End of \item{}

  \end{enumerate}

% ------------------------------------------------
\newpage
\subsection{Table}

  \begin{enumerate}

    \item
    {
      Set size

      這是table最基本的功能, 在Mircosoft Word時要插入多大的table時, 都要設定table的大小, 這邊正是那一個功能.
      \InsertCenterImage
        {./example/how-to/write/pic/table/table-set-size.png}
    } % End of \item{}

    \item
    {
      Clear table

      如果想把弄出來的table重新清掉所有設定和資料, 就是使用這一個.
      \InsertCenterImage
        {./example/how-to/write/pic/table/table-clear-table.png}
    } % End of \item{}

  \end{enumerate}

% ------------------------------------------------
\newpage
\subsection{Extra options}

  在下方的"Extra options"有幾個基本的功能
  \InsertCenterImage
    [scale=0.5]
    {./example/how-to/write/pic/table/options.png}

\begin{enumerate}

  \item
  {
    Center table horizontally

    把整個table置中在頁面
    \InsertCenterImage
      [scale=0.5]
      {./example/how-to/write/pic/table/options-table-center.png}

  } % End of \item{}

  \newpage
  \item
  {
    Caption above / below, Label

    把圖表的標題要放在上方還是下方

    \InsertCenterImage
      [scale=0.4,
        caption={標題放在上方}]
      {./example/how-to/write/pic/table/caption/above.png}

    \InsertCenterImage
      [scale=0.4,
        caption={標題放在下方}]
      {./example/how-to/write/pic/table/caption/below.png}

    {\bf 注意:} 由於它沒有位置去修改標題和label, 所以要手動把caption和label中的內容修改.
  } % End of \item{}
\end{enumerate}


% ------------------------------------------------
\newpage
\subsection{Style}

  在右邊可以設定table的style.
  \InsertCenterImage
    {./example/how-to/write/pic/table/style/style.png}

   正常在書本, 科學文章(如論文)和新聞中, table都是用三線式的方式, 因為這種的table簡單明瞭. 主要特點為整個table只有三條橫線, 上下兩端的線條較粗, 中間一條較細, 一般不使用分隔號.

  以下2個是2個是使用Latex原版table的顯示方式 (Fig \ref{table:style:default-1}, \ref{table:style:default-2}).

  \InsertCenterImage
    [scale=0.4,
      caption={Default style 1},
      label={table:style:default-1}]
    {./example/how-to/write/pic/table/style/default-1.png}

  \InsertCenterImage
    [scale=0.4,
      caption={Default style 2},
      label={table:style:default-2}]
    {./example/how-to/write/pic/table/style/default-2.png}

  \newpage
  以下2個是2個是使用三線式的顯示方式 (Fig \ref{table:style:booktabs-1}, \ref{table:style:booktabs-2}).

  \InsertCenterImage
    [scale=0.4,
      caption={Booktabs style 1},
      label={table:style:booktabs-1}]
    {./example/how-to/write/pic/table/style/booktabs-1.png}

  \InsertCenterImage
    [scale=0.4,
      caption={Booktabs style 2},
      label={table:style:booktabs-2}]
    {./example/how-to/write/pic/table/style/booktabs-2.png}

  就會發現中間的橫線比較細.

  這些都是一些細節問題, 如果想做簡單明瞭一些, 可以採用三線式表格, 但是不是說只要是表格就必須使用三線式.

% ------------------------------------------------
\newpage
\subsection{其他}

  \begin{enumerate}

    \item
    {
      功能

      其他功能都很好理解的, 只要嘗試過就會明白, 所以不再作詳細解釋.
    } % End of \item{}

    \item
    {
      圖片

      這tool沒法插入圖片, 所以有關圖片的部份要自己加在table中, 請使用\begin{verbatim}\InsertCenterImage和\InsertImage\end{verbatim}, 但是在table中的image是不能加標題和label.
    } % End of \item{}

    \item
    {
      備註

      而在產生出來的Latex中, 可以看到這類的文字(Fig \ref{table:package:comment}). 在注解中所講的, 是指所產生出來的Latex需要使用一些Latex的工具, 但這些工具已被包在本模板中, 所以可以無視的.

      \InsertCenterImage
        [scale=0.7,
          caption={Package remain},
          label={table:package:comment}]
        {./example/how-to/write/pic/table/table-comment.png}
    } % End of \item{}
  \end{enumerate}
% ------------------------------------------------
\EndChapter
% ------------------------------------------------
