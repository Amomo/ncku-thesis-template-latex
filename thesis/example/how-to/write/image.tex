% ------------------------------------------------
\StartChapter{圖片 Image}{chapter:how-to:write:image}
% ------------------------------------------------
\section{介紹}

插入圖片其實有很多的玩法, 但是在畢業論文中, 它的放置位置則是非常固定的, 都是以中間為主, 之後就是插多張圖片. 因為它很固定, 所以我針對了插入單張或多張, 分別提供了以下的指令.

要注意的是, 圖片在畫面看到的大小, 跟真正寫到文件是不一樣的 (因為經過程式的自動縮放), 所以比例正常都要修改的.

%\textbf{注意}: 由於Latex語法的設計上, 在2個'$\lbrace\rbrace$'之間不能斷行或出現空格, 否則會被判斷成其他的語法

% ------------------------------------------------
\newpage
\section{單張}

  \begin{framed}
  \begin{verbatim}
    Path:   圖片位置 (必填)

    Options 設定 (使用','來分隔, 不分先後順序)
      scale:   比例 (選填)
        (1.0/不填: 原大小; 0.x ~ < 1.0: 縮小; > 1.0: 放大)
      caption: 標題 (選填)
      label:   標簽 (選填, 必須要配合Caption使用, 否則無效)
      angle:   角度 (選填)

    插入圖片
    \InsertImage[Options]{Path}

    插入圖片並置中
    \InsertCenterImage[Options]{Path}

    E.g
      \InsertImage
        [caption={這 是 標 題}]
          {./image.png}

      \InsertCenterImage
        [scale=0.5,
          angle=45,
          caption={這 是 標 題},
          label={this:is:label}]
          {./image.png}

      每一項資料可以使用斷行來分隔以保持可讀性.
      caption和label必須要使用'{}'才能有空格的句子.
  \end{verbatim}
  \end{framed}

  \newpage

  {\bf 效果:}
  \begin{enumerate}
    \item
    {
      只填了比例和圖片位置
      \begin{verbatim}
        \InsertImage
          {./image.png}
      \end{verbatim}
      \InsertImage
        {./example/how-to/write/pic/image/Cc-by_new.svg.png}
    } % End of \item{}

    \item
    {
      使用置中的版本
      \begin{verbatim}
        \InsertCenterImage
          {./image.png}
      \end{verbatim}
      \InsertCenterImage
        {./example/how-to/write/pic/image/Cc-by_new.svg.png}
    } % End of \item{}

    \item
    {
      放大比例
      \begin{verbatim}
        \InsertCenterImage
          [scale=1.5]
            {./image.png}
      \end{verbatim}
      \InsertCenterImage
        [scale=1.5]
          {./example/how-to/write/pic/image/Cc-by_new.svg.png}
    } % End of \item{}

    \newpage

    \item
    {
      縮小比例
      \begin{verbatim}
        \InsertCenterImage
          [scale=0.5]
            {./image.png}
      \end{verbatim}
      \InsertCenterImage
        [scale=0.5]
          {./example/how-to/write/pic/image/Cc-by_new.svg.png}
    } % End of \item{}

    \item
    {
      增加標題並去掉比例的數字
      \begin{verbatim}
        \InsertCenterImage
          [caption={Little man}]
            {./image.png}
      \end{verbatim}
      \InsertCenterImage
        [caption={Little man}]
          {./example/how-to/write/pic/image/Cc-by_new.svg.png}
    } % End of \item{}

    \newpage
    \item
    {
      增加標簽
      \begin{verbatim}
        \InsertCenterImage
        [caption={Little man No.1},
          label={fig:little-man-no.1}]
            {./image.png}
      \end{verbatim}

      之後可以使用ref去引用 \verb| \ref{fig:little-man-no.1} |
      \InsertCenterImage
        [caption={Little man No.1},
          label={fig:little-man-no.1}]
          {./example/how-to/write/pic/image/Cc-by_new.svg.png}

      e.g: 文中所指的人物一號 (Fig. \ref{fig:little-man-no.1}).
    } % End of \item{}

%    \newpage
    \item
    {
      使用角度去轉45度
      \begin{verbatim}
        \InsertCenterImage
          [angle=45,
            caption={Little man No.2},
            label={fig:little-man-no.2}]
              {./image.png}
      \end{verbatim}

      使用ref去引用 \verb| \ref{fig:little-man-no.2} |
      \InsertCenterImage
        [angle=45,
          caption={Little man No.2},
          label={fig:little-man-no.2}]
            {./example/how-to/write/pic/image/Cc-by_new.svg.png}

      e.g: 文中所指的人物二號 (Fig. \ref{fig:little-man-no.2}).
    } % End of \item{}

  \end{enumerate}

% ------------------------------------------------
\newpage
\section{多張}

  如果要插入多張的話, 因為要能一頁版面的範圍內, 同時又要能清楚顯示到你圖中的內容和文字, 理論上4張都已經算多的了. 所以多張的話, 分別放同不到頁面會比較好.

  設計上可插入1$\sim$8張的圖片, 同時會自動置中, 而且寫法會跟插入單張相近.

  \begin{framed}
  \begin{verbatim}
  Table:    設定放置多張圖片的表格
    ImagePerRow: 每一列多少張圖片
    Caption:     標題 (可空掉)
    Label:       標簽 (可空掉, 要使用標簽前必須同時使用標題)

  Image 1~8: 各張圖片的設定
    設定方式跟使用\InsertImage和\InsertCenterImage是一樣的

  插入多張圖片
    \InsertMultiImages{Table}{Image1}{Image2}{Image3}{Image4}{Image5}{Image6}{Image7}{Image8}

  同時可接受這種使用方式 ('%'是必須存在的,
  以防止被Latex認為這是新段落, 而且不要有空格在當中)
    \InsertMultiImages%
    {%
      {ImagePerRow}{Caption}{Label}
    }%
    {%
      {Path}{Caption}{Label}{Angle}
    }%
    {%
      ...
    }%
    {%
      {Path}{Caption}{Label}{Angle}
    }%
  \end{verbatim}
  \end{framed}

  \newpage

  {\bf 效果:}
  \begin{enumerate}
    \item
    {
      插入2張圖片, 以1張圖為一列
      \begin{verbatim}
      \InsertMultiImages%
      {%
        {1}{2 images, 1 image per row}{}
      }%
      {%
        {1.0}{./image.png}
      }%
      {%
        {1.0}{./image.png}
      }%
      \end{verbatim}
      \InsertMultiImagesTest
        [perrow = 1,
          caption = {2 images, 1 image per row}]
      {
        [scale = 1.0]{./image-1}
      }
      {
        [scale = 2.0]{./image-2}
      }
    } % End of \item{}

    \newpage

    \item
    {
      插入5張圖片, 以2張圖為一列, 並有2張圖轉變角度, 同時有3張圖片做了標簽
      \begin{verbatim}
      \InsertMultiImages%
      {%
        {2}{5 images, 2 images per row}{fig:example:mi2:fig1}
      }%
      {%
        {1.0}{./image.png}{Image 1}
      }%
      {%
        {1.0}{./image.png}{Image 2}{
          fig:example:mi2:fig2}{-20}
      }%
      {%
        {1.0}{./image.png}{Image 3}
      }%
      {%
        {1.0}{./image.png}{Image 4}{fig:example:mi2:fig3}
      }%
      {%
        {1.0}{./image.png}{Image 5}{}{45}
      }%
      \end{verbatim}
      \InsertMultiImages%
      {%
        {2}{5 images, 2 images per row}{fig:example:mi2:fig1}
      }%
      {%
        {1.0}{./example/how-to/write/pic/image/CC-BY-NC.png}{Image 1}
      }%
      {%
        {1.0}{./example/how-to/write/pic/image/CC-BY-NC-ND.png}{Image 2}{fig:example:mi2:fig2}{-20}
      }%
      {%
        {1.0}{./example/how-to/write/pic/image/CC-BY-NC-SA.png}{Image 3}
      }%
      {%
        {1.0}{./example/how-to/write/pic/image/CC-BY-ND.png}{Image 4}{fig:example:mi2:fig3}
      }%
      {%
        {1.0}{./example/how-to/write/pic/image/CC-BY-SA.png}{Image 5}{}{45}
      }%
    } % End of \item{}

      e.g: 
      引用大圖 (Fig.\ref{fig:example:mi2:fig1}) ,
      引用小圖 (fig.\ref{fig:example:mi2:fig2}, fig.\ref{fig:example:mi2:fig3}).
  \end{enumerate}

% ------------------------------------------------
\EndChapter
% ------------------------------------------------
