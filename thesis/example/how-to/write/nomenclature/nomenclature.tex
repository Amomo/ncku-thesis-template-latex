% ------------------------------------------------
\StartSection{術語 Nomenclature}{chapter:how-to:write:nomenclature}
% ------------------------------------------------

Nomenclature在定義一些在整份論文中所會用到的變數是很常用到的. 它的位置會出現在文章當中或是在Chapter 1之前. 它的設計沒有一個標準答案, 在不同的情況下可能有不同顯示方式, 但它基本上跟一張Table是沒差的. 而它在Latex中是使用一個package名為'nomencl'.\\

但經過研究了一下package 'nomencl'或tabbing這些用來建Nomenclature的方式後, 發現'nomencl'在設計上反而會增加在產生論文時的步驟; 而tabbing要自行定義一個闊度才能弄得比較好看, 但同時內容卻出現沒法置中和設計上等一些問題. 故最後決定直接套用Table來讓同學更能自由的設計不同的Nomenclature table.\\

設計Nomenclature table需要2個知識或工具:\\
1) 設計一張Table, 這邊請參考P. \RefPage{chapter:how-to:write:table}.\\
2) 有關所需要用到的符號, 請參考Equation (P. \RefPage{chapter:how-to:write:equation})中所使用到的工具, Texmarker左邊的工具列, 或看這幾個網頁\RefBib{web:symbols:site1}\RefBib{web:symbols:site2}\RefBib{web:symbols:site3}, 應該已經足夠同學們寫出合適的符號.

% ------------------------------------------------
%\newpage
\StartSubSection{使用方式}

如果是指是在Chapter 1之前的一大張的Nomenclature table, 為Nomenclature Chapter. 
  \begin{verbatim}
  \StartNomChapter{ NAME }{ LABEL }
  \EndNomChapter
  \end{verbatim}
Nomenclature Chapter跟一般Chapter的使用方式是一樣的, 但差別在於不會出現'Chapter'這字眼. 而由於大家的Nomenclature Chapter name可能不一樣, 故跟Chapter一樣可設定自行的name.\\

而如果是在文章當中的Nomenclature table. 基本上就是使用同一個的'\verb|\InsertTable|', 但還可以使用'nomtitle'來設定標題. 'nomtitle'跟'caption'的差別是, 使用'nomtitle'所顯示出來的標題是沒有'Table XX:'為開頭, 同樣都是使用'pos'來控制題目的位置.

  \EmptyLine
  \begin{fmpage}{\textwidth}
  \begin{verbatim}
  Options 設定
    nomtitle:   Nomenclature 標題 (選填)
    ...

  E.g
    \InsertTable
    [nomtitle={這是Nomenclature Table的標題}]
      {
        ...
      }
  \end{verbatim}
  \end{fmpage}
  \EmptyLine

有關這個的用法可參考'example/nomenclature/nomenclature.tex'中的Nomenclature Chapter所demo的例子, 那2個例子只是最簡單的Nomenclature table設計, 應該足夠同學們去弄出合適自己的Nomenclature table的設計.

