% ------------------------------------------------
\StartSection{表格 Table}{chapter:how-to:write:table}
% ------------------------------------------------

\StartSubSection{介紹}

表格(Table)在任何情況下都是一個常用的顯示方式, 所以如何設計它都會有大量的玩法. 在正常Mircosoft Word這種有畫面的情況下, 可以慢慢拉出一個比較適合自己的, 但是在Latex中這個過程會是十分的痛苦, 因為你沒法馬上知道修改後的畫面, 故要不斷測試才知道效果, 這樣會大大減低選用table的使用次數.

在一般任何的Latex教學上, 如何編寫一個table出來都會是其中一項, 了解任何一個部份的寫法, 位置, 設定等. 但是由於那些資料十分的巨量 (不同寫法有不同效果), 所以這絕對不是使用本模板的大家想知道的東西, 故本模板不使用過往的方式, 而且直接教大家怎樣使用現有的online tool去處理掉這個問題.

以下的說明都是針對LaTeX Table Generator\RefBib{web:latex:table-generator}來進行說明.

LaTeX Table Generator (Fig \RefTo{fig:how-to:table:table-generator})的頁面非常明瞭和簡單, 只要有過Mircosoft Word中的table設計的經驗, 應該要上手這個東西絕對不會很難.

% ------------------------------------------------
\newpage
\InsertImage
  [scale=0.4, align=center,
  caption={LaTeX Table Generator頁面},
  label={fig:how-to:table:table-generator}]
  {./example/how-to/write/table/pic/table-generator.png}

% ------------------------------------------------
\newpage
\StartSubSection{產生Latex}

  我們使用這工具就是要去產生Latex用在論文當中, 所以這一步比其他的知識更為重要. 記得使用以下的步驟:

  \begin{enumerate}
  \item
  {
    使用畫面來設計table.
    \InsertImage
      [align=center]
      {./example/how-to/write/table/pic/table-view.png}
  } % End of \item{}

  \item
  {
    按Generate去產生Latex.
    \InsertImage
      [align=center]
      {./example/how-to/write/table/pic/generate.png}
  } % End of \item{}

  \item
  {
    複製Latex放到論文的".tex"中.
    \InsertImage
      [align=center]
      {./example/how-to/write/table/pic/latex-code.png}
  } % End of \item{}

  \item
  {
    執行XeLaTeX去產生效果.
  } % End of \item{}
  \end{enumerate}

  第1$\sim$3步會在整個設計table中常常都會使用, 所以會熟能生巧的. 而有經驗的人都知道, 第1步是最需要時間, 而第2$\sim$4步不用幾分鐘就能做完了, 所以只要用心的話, 多漂亮的table都是能弄出來的.

% ------------------------------------------------
\newpage
\StartSubSection{功能}

要設計一個複雜的table就需要足夠的功能才能慢慢弄, 所以在這邊介紹一些算是非常有用的功能.

\StartSubSection{File}

  在"File"中有幾個很有用的功能.
  \InsertImage
    [align=center]
    {./example/how-to/write/table/pic/menu-file.png}

  \begin{enumerate}

  % ------------------------------------------------
  \item
  {
    Import CSV file

    你可以直接upload一個CSV format的檔案之後弄table的外觀.
    \InsertImage
      [scale=0.7, align=center]
      {./example/how-to/write/table/pic/csv.png}
  } % End of \item{}

  % ------------------------------------------------
  \newpage
  \item
  {
    Paste table data

    可以把Microsoft Excel的table直接做Copy \& Paste到這一邊來.
    \InsertImage
      [scale=0.45, align=center]
      {./example/how-to/write/table/pic/paste.png}

    或是可以直接輸入資料來建立, 但要注意的是它只能接受CSV的寫法, 即是每一筆資料都是以","來分隔. 所以如果使用Fig \RefTo{fig:csv:enter-example-data}的寫法的話:
    \InsertImage
      [scale=0.65, align=center,
        caption={Enter example data},
        label={fig:csv:enter-example-data}]
      {./example/how-to/write/table/pic/paste-data.png}

    會出現Fig \RefTo{fig:csv:result-example-data}的效果:
    \InsertImage
      [scale=0.65, align=center,
        caption={Result of example data},
        label={fig:csv:result-example-data}]
      {./example/how-to/write/table/pic/paste-data-result.png}

  } % End of \item{}

  % ------------------------------------------------
  \newpage
  \item
  {
    Save table

    這online tool有一個十分有用的功能就是能把所做的table save下來, 只要輸入名字後再按download就會得到一個".tgn"檔案.
    \InsertImage
      [scale=0.8, align=center]
      {./example/how-to/write/table/pic/save-table.png}

    \InsertImage
      [align=center]
      {./example/how-to/write/table/pic/save-tgn.png}

  } % End of \item{}

  % ------------------------------------------------
  \item
  {
    Load table

    在"Save table"中得到的".tgn"檔案就是使用這邊來重新讀取table.
    \InsertImage
      [scale=0.8, align=center]
      {./example/how-to/write/table/pic/load-table.png}
  } % End of \item{}
  \end{enumerate}

\newpage
\StartSubSection{Edit}

  在"Edit"中有2個常用的功能

  \InsertImage
    [align=center]
    {./example/how-to/write/table/pic/menu-edit.png}

  \begin{enumerate}

  \item
  {
    Undo / Repeat

    很基本的重做上一步/下一步所做過的行為, 故不用解釋什麼.
  } % End of \item{}

  \item
  {
    Autosave

    這功能十分有用, 因為這tool是網頁tool, 所以正常重開網頁時會令到資料不見. 所以如果有把"Autosave"開啟的話, 那table就算接了"F5"都不會不見. (預設上應該會自動有開啟)
    \InsertImage
      [align=center]
      {./example/how-to/write/table/pic/edit-autosave.png}
  } % End of \item{}

  \end{enumerate}

% ------------------------------------------------
\newpage
\StartSubSection{Table}

  \begin{enumerate}

  \item
  {
    Set size

    這是table最基本的功能, 在Mircosoft Word時要插入多大的table時, 都要設定table的大小, 這邊正是那一個功能.
    \InsertImage
      [align=center]
      {./example/how-to/write/table/pic/table-set-size.png}
  } % End of \item{}

  \item
  {
    Clear table

    如果想把弄出來的table重新清掉所有設定和資料, 就是使用這一個.
    \InsertImage
      [align=center]
      {./example/how-to/write/table/pic/table-clear-table.png}
  } % End of \item{}

  \end{enumerate}

% ------------------------------------------------
\newpage
\StartSubSection{Extra options}

  在下方的"Extra options"有幾個基本的功能
  \InsertImage
    [align=center, scale=0.5]
    {./example/how-to/write/table/pic/options.png}

\begin{enumerate}

  \item
  {
  Center table horizontally

  把整個table置中在頁面
  \InsertImage
    [align=center, scale=0.5]
    {./example/how-to/write/table/pic/options-table-center.png}

  } % End of \item{}

  %\newpage
  \label{chapter:how-to:write:table:label-example}
  \item
  {
  Caption above / below, Label

  把圖表的標題要放在上方還是下方

  \InsertFigures
    [perrow = 2,
      caption = {Option of caption}] %
    {
      [scale=0.4,
      caption={標題放在上方}]
      {./example/how-to/write/table/pic/caption/above.png}
    }%
    {
      [scale=0.4,
      caption={標題放在下方}]
      {./example/how-to/write/table/pic/caption/below.png}
    }

  {\bf 注意:} 由於它沒有位置去修改caption和label, 所以要手動把caption和label中的內容修改.
  } % End of \item{}
\end{enumerate}

% ------------------------------------------------
\newpage
\StartSubSection{Style}

  在右邊可以設定table的style.
  \InsertImage
    [align=center]
    {./example/how-to/write/table/pic/style/style.png}

   正常在書本, 科學文章(如論文)和新聞中, table都是用三線式的方式, 因為這種的table簡單明瞭. 主要特點為整個table只有三條橫線, 上下兩端的線條較粗, 中間一條較細, 一般不使用分隔號.

  Fig \RefTo{table:style:sample-1}是一個例子分別是使用Latex原版的顯示方式(Fig \RefTo{table:style:default-1})或是使用booktabs版的顯示方式(Fig \RefTo{table:style:booktabs-1}).

  \InsertFigures
    [perrow = 2,
      caption = {A sample between Latex style and Booktabs style},
      label={table:style:sample-1}] %
    {
      [scale=0.3,
      caption={Default style},
      label={table:style:default-1}]
      {./example/how-to/write/table/pic/style/default-1.png}
    }%
    {
      [scale=0.2,
      caption={Booktabs style},
      label={table:style:booktabs-1}]
      {./example/how-to/write/table/pic/style/booktabs-1.png}
    }

  %\newpage
  而Fig \RefTo{table:style:sample-2}是2個版本都加上垂直線時候的樣子.

  \InsertFigures
    [perrow = 2,
      caption = {Table with horizontal line},
      label={table:style:sample-2}] %
    {
      [scale=0.3,
      caption={Default style}]
      {./example/how-to/write/table/pic/style/default-2.png}
    }%
    {
      [scale=0.25,
      caption={Booktabs style}]
      {./example/how-to/write/table/pic/style/booktabs-2.png}
    }

  就會發現booktabs版的中間的橫線比較細.

  這些都是一些細節問題, 如果想做簡單明瞭一些, 可以採用三線式表格, 但不是說只要是表格就必須使用三線式.

% ------------------------------------------------
%\newpage
\StartSubSection{其他}

  \begin{enumerate}

  \item
  {
    功能

    其他功能都很好理解的, 只要嘗試過就會明白, 所以不再作詳細解釋.
  } % End of \item{}

  \item
  {
    圖片

    這tool沒法插入圖片, 所以有關圖片的部份要自己加在table中, 請參考P. \RefPage{table:how-to:write:image:insert-image-into-table}, 但是在table中的image是不能加標題和label.
  } % End of \item{}

  \item
  {
    備註

    而在產生出來的Latex中, 可以看到這類的文字(Fig \RefTo{table:package:comment}). 在注解中所講的, 是指所產生出來的Latex需要使用一些Latex的工具, 但這些工具已被包在本模板中, 所以可以無視的.

    \InsertImage
      [scale=0.7, align=center,
        caption={Package meno},
        label={table:package:comment}]
      {./example/how-to/write/table/pic/table-comment.png}
  } % End of \item{}
  \end{enumerate}

% ------------------------------------------------
\EndChapter
% ------------------------------------------------
