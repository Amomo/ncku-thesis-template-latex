% ------------------------------------------------
\StartSection{章節 Chapter/Section}{chapter:how-to:write:chapter-section}
% ------------------------------------------------

\StartSubSection{介紹}

編寫任何的文章, 都會使用不同的章節來把內容進行分區. 例如學校的排版樣子大約:

  \EmptyLine
\begin{DescriptionFrame}
  \centerline{\LARGE Chapter X}
  \vspace{0.2cm}
  \centerline{\LARGE 這是標題}

  \vspace{0.5cm}
  \mbox{\Large X.1 子項目}\\
  \mbox{\hspace{1.2cm}項目內容 ...}

  \vspace{0.3cm}
  \mbox{\large X.1.1 子項目}\\
  \mbox{\hspace{1.2cm}項目內容 ...}
\end{DescriptionFrame}
  \EmptyLine

所以針對這些功能, 本模版提供:

  \EmptyLine
\begin{DescriptionFrame}
  \begin{verbatim}
    主要章節
    Title: 標題 (必填)
    Label: 標簽 (選填)
    \StartChapter{ Title }{ Label }
    \EndChapter % 用來保證你的內容在這Chapter內

    次章節
    Title: 標題 (必填)
    Label: 標簽 (選填)
    \StartSection{ Title }{ Label }

    次章節的子章節
    Title: 標題 (必填)
    Label: 標簽 (選填)
    \StartSubSection{ Title }{ Label }
  \end{verbatim}
\end{DescriptionFrame}
  \EmptyLine

所以針對剛剛的例子, 它的LaTex寫法為:

  \EmptyLine
\begin{DescriptionFrame}
  \begin{verbatim}
    \StartChapter{這是標題}

    \StartSection{子項目}
    項目內容 ...

    \StartSubSection{X.1的子項目}
    項目內容 ...

    \EndChapter
  \end{verbatim}
\end{DescriptionFrame}

