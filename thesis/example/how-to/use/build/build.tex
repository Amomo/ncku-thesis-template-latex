% ------------------------------------------------
\StartSection{產生論文 Generate Thesis}{chapter:how-to:use:generate}
% ------------------------------------------------

\StartSubSection{介紹}
這邊會簡單講解如何安裝基本的程式來產生你的論文.

\StartSubSection{MiKTeX安裝}
我們需要MiKTeX來幫我們來轉Latex成PDF.

  \InsertImage
    [align=center, scale=0.5,
      caption={MiKTeX Logo}]
    {./example/how-to/use/build/pic/miktex/logo.png}

首先去MiKTeX的網頁\RefBib{web:miktex:website}來下載它回來, 它預設在'Recommended Download'是32-bit的, 所以如果你要下載64-bit的話, 就要按'Other Downloads'中的第一個.

  \InsertImage
    [align=center, scale=0.2,
      caption={Download MiKTeX}]
    {./example/how-to/use/build/pic/miktex/mdownload.png}

  \InsertImage
    [align=center, scale=0.35,
      caption={安裝MiKTeX}]
    {./example/how-to/use/build/pic/miktex/minstall.png}

安裝MiKTeX其實沒有什麼需要太在意的東西, 但由獨有一個東西需要設定, 在不停按下一步時, 會出現fig \RefTo{fig:how-to:use:build:package-download}這個畫面, 在這邊推薦選擇'\textbf{Yes}', 因為這邊是用來設定自動幫你下載一些你需要使用的工具來產生論文.

  \InsertImage
    [align=center, scale=0.35,
      caption={Download Package},
      label={fig:how-to:use:build:package-download}]
    {./example/how-to/use/build/pic/miktex/auto-download.png}

最後就要等待安裝, 由於內容滿多, 所以在這邊可能需要等待幾分鐘.

  \InsertImage
    [align=center, scale=0.35,
      caption={等待安裝完成}]
    {./example/how-to/use/build/pic/miktex/installing.png}

% ------------------------------------------------
\newpage
\StartSubSection{Texmaker安裝}

我們需要Texmaker來幫我們處理產生流程和看PDF用.

  \InsertImage
    [align=center, scale=0.5,
      caption={Texmaker Logo}]
    {./example/how-to/use/build/pic/texmaker/logo.png}

首先去Texmaker的網頁\RefBib{web:texmaker:website}來下載它回來, 推薦使用'Executable file for windows', 同時使用'Alternative download link' (因為這個line是使用Google Drive, 所以速度能有保證).

  \InsertImage
    [align=center, scale=0.2,
      caption={Download Texmaker}]
    {./example/how-to/use/build/pic/texmaker/download.png}

安裝Texmaker其實沒有什麼要設定的東西, 不停按下一步就行了.

  \InsertImage
    [align=center, scale=0.35,
      caption={安裝Texmaker}]
    {./example/how-to/use/build/pic/texmaker/install.png}

  \InsertImage
    [align=center, scale=0.35,
      caption={安裝Texmaker}]
    {./example/how-to/use/build/pic/texmaker/install-2.png}

% ------------------------------------------------
\newpage
\StartSubSection{產生論文和書脊}

當安裝完Texmaker和MiKTeX後, 直接點開'thesis.tex', 可以看到這個畫面(Fig\RefTo{fig:how-to:use:build:texmaker:thesis.tex}).

  \InsertImage
    [align=center, scale=0.25,
      caption={Texmaker打開thesis.tex畫面},
      label={fig:how-to:use:build:texmaker:thesis.tex}]
    {./example/how-to/use/build/pic/texmaker/1.png}

產生論文的方式為在上方(Fig\RefTo{fig:how-to:use:build:texmaker:to_xelatex})由'快速編譯'改成'XeLaTeX', 之後按左方的箭頭就可以進行產生的處理(註: 如果是第一次使用, 那這時候背後MiKTeX會自動下載一些工具回來, 所以會等待比較久).

  \InsertImage
    [align=center, scale=0.5,
      caption={改使用XeLaTeX},
      label={fig:how-to:use:build:texmaker:to_xelatex}]
    {./example/how-to/use/build/pic/texmaker/2.png}

\newpage
之後只要等待下方出現一些結果(Fig\RefTo{fig:how-to:use:build:texmaker:gen_message})就是說明已產生完成.

  \InsertImage
    [align=center, scale=0.25,
      caption={處理的結果},
      label={fig:how-to:use:build:texmaker:gen_message}]
    {./example/how-to/use/build/pic/texmaker/2.5.png}

如果PDF產生成功, 那接旁邊'瀏覽PDF'的箭頭, 會出現一個視窗(Fig\RefTo{fig:how-to:use:build:texmaker:new_pdf})來顯示那個PDF檔.

  \InsertImage
    [align=center, scale=0.25,
      caption={瀏覽PDF},
      label={fig:how-to:use:build:texmaker:new_pdf}]
    {./example/how-to/use/build/pic/texmaker/3.png}

\newpage
把以上的步驟用在'spine.tex'上(Fig\RefTo{fig:how-to:use:build:texmaker:spine.tex})就能去產生你的書脊.

  \InsertImage
    [align=center, scale=0.3,
      caption={Texmaker打開spine.tex畫面},
      label={fig:how-to:use:build:texmaker:spine.tex}]
    {./example/how-to/use/build/pic/texmaker/4.png}

當如果你已經把'thesis.tex'和'spine.tex'都產生了PDF, 你的資料夾應該會有這些檔案和資料夾(Fig\RefTo{fig:how-to:use:build:texmaker:dir}), 那2個PDF正是你需要的東西.

  \InsertImage
    [align=center, scale=0.5,
      caption={資料夾內容},
      label={fig:how-to:use:build:texmaker:dir}]
    {./example/how-to/use/build/pic/texmaker/5.png}

% ------------------------------------------------

\newpage
\StartSubSection{產生PDF的流程}

在編譯Latex時成PDF時, 必須注意內部的引用(\verb|\RefBib{}|)情況.

如果只是編寫內容, 引用的內容和號碼不是重要的話, 那直接使用:\\
XeLaTeX -> 瀏覽PDF\\
即可.

但如果你的PDF是最終版本, 那你的流程則需要使用:\\
XeLaTeX -> BibTex -> XeLaTeX -> 瀏覽PDF\\
才對.

因為第一次的XeLaTeX是用來產生'thesis.aux', 而有這個檔案才能對你的內容中Reference的引用來進行連接, 而BibTex正是做這個的處理以產出'thesis.bbl', 而第二次的XeLaTeX會使用'thesis.bbl'來把你的Reference中的號碼在內容中設定.

以上步驟只需用在'thesis.tex'的部份, 而'spine.tex'則不用做這個行為.

% ------------------------------------------------
