% ------------------------------------------------
\StartSection{產生論文 Generate Thesis}{chapter:how-to:use:generate}
% ------------------------------------------------

\StartSubSection{介紹}
這邊會簡單講解如何安裝基本的程式來產生你的論文.

\StartSubSection{MiKTeX安裝}
我們需要MiKTeX來幫我們來轉Latex成PDF.

  \InsertCenterImage
    [scale=0.5,
      caption={MiKTeX Logo}]
    {./example/how-to/use/build/pic/miktex/logo.png}

首先去MiKTeX的網頁\RefBib{web:miktex:website}來下載它回來, 它預設在'Recommended Download'是32-bit的, 所以如果你要下載64-bit的話, 就要按'Other Downloads'中的第一個.

  \InsertCenterImage
    [scale=0.2,
      caption={Download MiKTeX}]
    {./example/how-to/use/build/pic/miktex/mdownload.png}

  \InsertCenterImage
    [scale=0.35,
      caption={安裝MiKTeX}]
    {./example/how-to/use/build/pic/miktex/minstall.png}

安裝MiKTeX其實沒有什麼需要太在意的東西, 但由獨有一個東西需要設定, 在不停按下一步時, 會出現fig \RefTo{fig:how-to:use:build:package-download}這個畫面, 在這邊推薦選擇'\textbf{Yes}', 因為這邊是用來設定自動幫你下載一些你需要使用的工具來產生論文.

  \InsertCenterImage
    [scale=0.35,
      caption={Download Package},
      label={fig:how-to:use:build:package-download}]
    {./example/how-to/use/build/pic/miktex/auto-download.png}

最後就要等待安裝, 由於內容滿多, 所以在這邊可能需要等待幾分鐘.

  \InsertCenterImage
    [scale=0.35,
      caption={等待安裝完成}]
    {./example/how-to/use/build/pic/miktex/installing.png}

% ------------------------------------------------
\newpage
\StartSubSection{Texmaker安裝}

我們需要Texmaker來幫我們處理產生流程和看PDF用.

  \InsertCenterImage
    [scale=0.5,
      caption={Texmaker Logo}]
    {./example/how-to/use/build/pic/texmaker/logo.png}

首先去Texmaker的網頁\RefBib{web:texmaker:website}來下載它回來, 推薦使用'Executable file for windows', 同時使用'Alternative download link' (因為這個line是使用Google Drive, 所以速度能有保證).

  \InsertCenterImage
    [scale=0.2,
      caption={Download Texmaker}]
    {./example/how-to/use/build/pic/texmaker/download.png}

安裝Texmaker其實沒有什麼要設定的東西, 不停按下一步就行了.

  \InsertCenterImage
    [scale=0.35,
      caption={安裝Texmaker}]
    {./example/how-to/use/build/pic/texmaker/install.png}

  \InsertCenterImage
    [scale=0.35,
      caption={安裝Texmaker}]
    {./example/how-to/use/build/pic/texmaker/install-2.png}

% ------------------------------------------------




編譯


xelatex thesis // 第一次基本的編譯
bibtex thesis // 處理論文引用
xelatex thesis // 第二次編譯利用 bibtex 的結果產生論文引用的連結


