% ------------------------------------------------
\StartSection{基本介紹 Introduction}{chapter:how-to:use:intro}
% ------------------------------------------------

如果要使用本模板來寫你的論文, 那你要先拿到3個東西:

% ------------------------------------------------
\StartSubSection{本模板的檔案}

本模板的source code (.tex檔, 即是Mircosoft Word的.doc檔) 已經完整的放在GitHub上\RefBib{web:this-project:github} (Fig \RefTo{fig:how-to:use:intro:github}).

\InsertCenterImage
  [scale=0.5,
    caption={Templete on GitHub},
    label={fig:how-to:use:intro:github}]
  {./example/how-to/use/pic/github.png}

可以使用右方的"Download ZIP"來下載最新版的本模板(Fig \RefTo{fig:how-to:use:intro:github-download}).
\InsertCenterImage
  [scale=0.4,
    caption={Download Icon},
    label={fig:how-to:use:intro:github-download}]
  {./example/how-to/use/pic/github-download.png}

% ------------------------------------------------
\StartSubSection{產生用的工具}

用來讀取Latex來產生你的論文用的工具

以Mircosoft Word來講是Mircosoft Office

% ------------------------------------------------
\StartSubSection{編寫用Editor}

用來編寫你的論文用的Editor
因為要寫Word的話, 就必須使用Mircosoft Office才可以寫. 但如果是寫Latex的話, 就算只是記事本都可以編寫, 所以在這邊你可以去使用你喜好的Editor.

但注意的是, 切勿不要使用一些預設不是針對UTF-8的Editor, 如 Windows內建的記事本(notepad). 因為你在過程中應該都會寫出或留下一些中文, 這時候如果那個Editor自動存成其他編碼, 就會有處理上的問題.

所以使用一些有名的Editor會比較安全, 如Notepad++, gedit等 (當然你都可以使用你熟悉的)



