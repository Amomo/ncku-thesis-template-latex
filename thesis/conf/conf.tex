
% This file is need to encoded in utf-8
%
% Choose or fill in some needed information for this thesis or dissertation
% 選擇或填入你的論文一些需要使用的資料

% ----------------------------------------------------------------------------

% --- 使用的論文內容 ---
% 如果沒有打開\DemoMode
% 就會使用'./context/context.tex'中你所編寫論文內容.
% 否則會使用'./example/context.tex'的模版說明文件內容.

\DemoMode

% ----------------------------------------------------------------------------

% --- 封面上顯示文字的語言 ---
%
% \DisplayCoverInChi:  封面以全中文顯示
% \DisplayCoverInEng:  封面以全英文顯示
% 只能選擇其中一個, 但只有最後設定的一方有效

\DisplayCoverInChi

%--------------------------------------------------

% --- 封面名字顯示方式 ---
%
% 預設在封面上只會顯示中文或英文名字而已.
%
% 不論你是使用\DisplayCoverInChi或\DisplayCoverInEng,
% 把'%'拿掉以設定同時顯示中英文名字.

\CDBothName

% ----------------------------------------------------------------------------

% --- Title 論文題目 ---
% 填寫中文和(或)英文
% 如果題目內有必須以數學模式表示的符號,請用 \mbox{} 包住數學模式
% 如果覺得自動產生出來的題目斷行位置不適合, 可以手動加'\\'來強制斷行
% (圖書館說不管你是編寫中英混合或全英文版, 都必須同時存在中英題目)
%
% 有3種可使用, 可獨立使用, 但只有最後設定的一方有效
% \SetTitle{你的題目}{Your Title}   % 同時設定中英文題目
% \SetChiTitle{你的題目}            % 只設定中文題目
% \SetEngTitle{Your Title}         % 只設定英文題目
%
% e.g:
%
% \SetTitle %
% {國立成功大學碩博士用畢業論文XeLaTex模版} %
% {National Cheng Kung University (NCKU) Thesis/Dissertation Template in XeLaTex}
%
% or
%
% \SetChiTitle{國立成功大學碩博士用畢業論文XeLaTex模版}
% \SetEngTitle{National Cheng Kung University (NCKU)\\ Thesis/Dissertation Template in XeLaTex}

\SetTitle %
{國立成功大學碩博士用畢業論文XeLaTex模版} %
{National Cheng Kung University (NCKU) \\ Thesis/Dissertation Template in XeLaTex}

% ----------------------------------------------------------------------------

% --- Draft 初稿 ---
% 顯示 '(初稿)' (中文版) 和 '(Draft)' (英文版) 在封面
\DisplayDraft

% ----------------------------------------------------------------------------

% --- Degree name 學位 ---
%
% 有2種可選擇, 但只有最後設定的一方有效
% \PhdDegree    % 博士學位
% \MasterDegree % 碩士學位

\PhdDegree

% ----------------------------------------------------------------------------

% --- Your name 你的名字 ---
% 填寫你的中文和(或)英文

% 有3種可使用, 可獨立使用, 但只有最後設定的一方有效
% \SetMyName{你的名字}{Your name}   % 同時設定你的中英文名字
% \SetMyChiName{你的名字}           % 只設定你的中文名字
% \SetMyEngName{Your name}         % 只設定你的英文名字

\SetMyName{你的名字}{Your name}

% ----------------------------------------------------------------------------

% --- Date 日期 ---

% 依本校研究生學位考試細則第十條規定:
%
% 碩士班:
%   論文日期:上學期為〇〇〇年1月;下學期為〇〇〇年6月,以該學期結束日期(一月或六月)為準。
%   (如:在上學期101年9月~102年1月期間口試,
%       不論是在此期間何月份口試,其日期均固定為102年1月).
%   另碩士生如101上學期完成口試,101下學期申請出國,102上學期辦理離校,
%   則論文封面為103年1月
%
% 博士班:
%   以當學期通過學位口試,則論文日期為口試日期(如〇〇〇年〇〇月〇〇日),
%   若論文有修改致延至次學期,則以論文上傳日期為主。
%
% 故本模版會根據你的學位, 來選擇顯示在封面的日期格式.
%

% --- 論文封面上的日期 ---
% 設定西元的年月, 會自動計算出民國的年份, 和英文的月份轉換
% 次序: {年份}{月份}
% \SetCoverDate{2014}{12}

\SetCoverDate{2014}{12}

%--------------------------------------------------

% --- 口試的日期 ---
% 設定西元的年月日, 會自動計算出民國的年份, 和英文的月份轉換
% 次序: {年份}{月份}{日}
% \SetOralDate{2014}{12}{31}

\SetOralDate{2014}{12}{31}

% ----------------------------------------------------------------------------

% --- 系所 Department or Institute ---
%
% 設定你的系所名字, e.g:
% \SetDeptMath 數學系
% \SetDeptCSIE 資訊工程學系

\SetDeptCSIE

% ----------------------------------------------------------------------------

% --- 指導老師 Advisor(s) ---
% 在封面上預算了最多3位的空間
% 中文名字固定以 博士  為結尾
% 英文名字固定以 Dr. 為開頭

% 有3種可使用, 用來設定3位老師的名字
% \SetAdvisorNameX{老師的名字}{Professor's name} % 同時設定中英文名字
% \SetAdvisorChiNameX{老師的名字}                % 只設定中文名字
% \SetAdvisorEngNameX{Professor's name}         % 只設定英文名字
% (NameX為NameA, NameB, NameC)

% 使用\SetAdvisorNameA是必須的, 而如果你的指導教授有2或3位,
% 那只要增加\SetAdvisorNameB和\SetAdvisorNameC則可

\SetAdvisorNameA{A}{A}
\SetAdvisorNameB{B}{B}
\SetAdvisorNameC{C}{C}

% ----------------------------------------------------------------------------

% --- 口試証明文件 Oral presentation document ---
% 使用範例版本 或 使用檔案 只能選擇其中一方

% 使用口試範例版本
\DisplayOralTemplate

% --- 範例版本的語言 ---
% 選擇你需要的範例
% (Only work with \DisplayOralTemplate)
% \DisplayOralChiTemplate    % 顯示中文範例版本
% \DisplayOralEngTemplate    % 顯示英文範例版本

\DisplayOralChiTemplate    % 顯示中文範例版本
\DisplayOralEngTemplate    % 顯示英文範例版本

% --- 口試委員 Committee member(s) ---
% 口試委員數量 (至少2位, 最多9位, 預設為9位)
% (Only work with \DisplayOralTemplate)
% 博士學位考試委員會置委員五人至九人
% 碩士學位考試委員會置委員三人至五人
% 口試委員人數含指導教授
\SetCommitteeSize{9}

%--------------------------------------------------

% 使用口試圖片檔案
% 把你的圖片放在'context/oral'下
% 之後設定中英文版所對應是哪一個檔案
% 就算已啟用\DisplayOralImage,
% 但沒有填寫圖檔檔名的話, 都不會顯示出來.
% (例子用的'example-oral-chi.pdf'和'example-oral-eng.pdf'已放在'context/oral'中)

%\DisplayOralImage                % 顯示圖檔
%\SetOralImageChi{example-oral-chi.pdf}   % 中文口試檔案
%\SetOralImageEng{example-oral-eng.pdf}   % 英文口試檔案

% ----------------------------------------------------------------------------

% --- 關鍵字 Keyword ---
% 最多9個關鍵字
% 為了方便同學自行設定
% 故所產出來的PDF檔案中的關鍵字和內文摘要的關鍵字
% 可獨立個別設定

% \SetKeywords是設定所產出來的PDF中的Keyword項目
% 可同時填寫中英文
% e.g
% \SetKeywords{Keyword A (關鍵字 A)}{Keyword B (關鍵字 B)}{Keyword C (關鍵字 C)}
% 或單純中文或英文
% \SetKeywords{Keyword A}{Keyword B}{Keyword C}
% \SetKeywords{關鍵字 A}{關鍵字 B}{關鍵字 C}

\SetKeywords{NCKU Thesis/Dissertation template}{Graduate}{LaTex/XeLaTex}

% 摘要中的關鍵字
% 為了方便同學們能達到以下情況:
% a. 只寫中文版摘要
% b. 只寫英文版摘要
% c. 同時寫中英文版摘要
% 故中英文版的關鍵字都是可個別設定
% \SetAbstractChiKeywords: 用來設定中文版摘要的關鍵字
% \SetAbstractEngKeywords: 用來設定英文版摘要的關鍵字
% \SetAbstractExtKeywords: 用來設定英文延伸摘要的關鍵字 (只有你要編寫英文延伸摘要才需要設定)
% 所以只要使用你需要寫的版本則可.
% 但如果2個版本都要寫, 則2個都同時使用則可.
% 沒有填寫的話, 則摘要中的關鍵字部份是不會顯示出來.
%
% e.g
% \SetAbstractChiKeywords{關鍵字 A}{關鍵字 B}{關鍵字 C}
% \SetAbstractEngKeywords{Keyword A}{Keyword B}{Keyword C}
% \SetAbstractExtKeywords{Keyword A}{Keyword B}{Keyword C}
% 英文延伸摘要的關鍵字理應會跟英文版摘要的關鍵字是一樣,
% 但為了同學能編寫不同內容和關鍵字, 故可獨立設定.

\SetAbstractChiKeywords{國立成功大學畢業論文模版}{碩博士}{LaTex/XeLaTex}
\SetAbstractEngKeywords{NCKU Thesis/Dissertation Template}{Graduate}{LaTex/XeLaTex}
\SetAbstractExtKeywords{NCKU Thesis/Dissertation Template}{Graduate}{LaTex/XeLaTex}

% ----------------------------------------------------------------------------

% --- 目錄 Index ---
% 設定可獨立使用, 但只有最後設定的一方有效

% 標題文字語言 Language
% 目錄的標題文字使用預設的中文或是英文
% \IndexChiMode:  標題文字為中文
% \IndexEngMode:  標題文字為英文

\IndexEngMode

% 標題文字 Text of title
% 預設的目錄標題為: 目錄 (中文) / Table of Contents (英文)
% 預設的表格目錄標題為: 表格 (中文) / List of Tables (英文)
% 預設的圖片目錄標題為: 圖片 (中文) / List of Figures (英文)
% 如應為預設文字不是你所希望的, 那可以使用這邊去個別設定你所希望的文字, 不分中文英.

% 設定目錄標題
%\SetIndexTitleText{Table of Contents / 目錄}

% 設定表格目錄標題
%\SetTablesIndexTitleText{List of Tables / 表格}

% 設定圖片目錄標題
%\SetFiguresIndexTitleText{List of Figures / 圖片}

% ----------------------------------------------------------------------------

% --- 章節標題的語言 ---
%
% \ChapterSectionTitleInChi:  章節標題以全中文顯示
% 預設上章節標題是以英文來顯示 (如 Chapter 1), 但全中文編寫則需要顯示成中文,
% 故使用'\ChapterSectionTitleInChi'可以顯示成中文 (如 第1章).

%\ChapterSectionTitleInChi

% ----------------------------------------------------------------------------
