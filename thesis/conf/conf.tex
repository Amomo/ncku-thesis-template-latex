
% This file is need to encoded in utf-8
% 填入你的論文一些需要使用的資料

% ----------------------------------------------------------------------------

% --- Title 論文題目 ---
% 填寫中文和(或)英文
% 如果題目內有必須以數學模式表示的符號,請用 \mbox{} 包住數學模式

% 有3種可使用, 可獨立使用, 但只有最後設定的一方有效
% \SetTitle{你的題目}{Your Title}   % 同時設定中英文題目
% \SetChiTitle{你的題目}            % 只設定中文題目
% \SetEngTitle{Your Title}         % 只設定英文題目

\SetTitle
{國立成功大學碩博士用畢業論文XeLaTex模板}
{National Cheng Kung University (NCKU) \\ Thesis/Dissertation Template in XeLaTex}
% \SetChiTitle{國立成功大學碩博士用畢業論文XeLaTex模板}
% \SetEngTitle{National Cheng Kung University (NCKU) \\ Thesis/Dissertation Template in XeLaTex}

% ----------------------------------------------------------------------------

% --- Degree name 學位 ---
% 不用填寫, 只要選擇即可

% 有2種可使用, 但只有最後設定的一方有效
% \PhdDegree    % 博士學位
% \MasterDegree % 碩士學位

\MasterDegree

% ----------------------------------------------------------------------------

% --- Your name 你的名字 ---
% 填寫你的中文和(或)英文

% 有3種可使用, 可獨立使用, 但只有最後設定的一方有效
% \SetMyName{你的名字}{Your name}   % 同時設定你的中英文名字
% \SetMyChiName{你的名字}           % 只設定你的中文名字
% \SetMyEngName{Your name}         % 只設定你的英文名字

\SetMyName{你的名字}{Your name}

% ----------------------------------------------------------------------------

% --- Date 日期 ---

% --- 論文封面上的日期 ---
% 填寫中文日期(年份, 月份)和(或)英文日期(年份, 月份)
% 一次過設定4個數字, 或是只有2個就行了
% {中文年份}{中文月份}{英文年份}{英文月份}

% 有3種可使用, 可獨立使用, 但只有最後設定的一方有效
% \SetThesisDate{100}{1}{100}{1}   % 同時設定中英文日期
% \SetThesisChiDate{100}{1}        % 只設定中文日期
% \SetThesisEngDate{2014}{January} % 只設定英文日期

\SetThesisDate{100}{1}{2014}{January}

%--------------------------------------------------

% ---  口試的日期 (可不用設定, 請參考內文 '口試資料' 中的說明) ---
% 填寫中文日期(年份, 月份)和(或)英文日期(年份, 月份)
% 一次過設定4個數字, 或是只有2個就行了
% {中文年份}{中文月份}{英文年份}{英文月份}

% 有3種可使用, 可獨立使用, 但只有最後設定的一方有效
% \SetOralDate{100}{1}{100}{1}   % 同時設定中英文日期
% \SetOralChiDate{100}{1}        % 只設定中文日期
% \SetOralEngDate{2014}{January} % 只設定英文日期

\SetOralDate{100}{1}{2014}{January}

% ----------------------------------------------------------------------------

% --- 系所 Department or Institute ---
% 填寫你的系所名字

% 有3種可使用, 可獨立使用, 但只有最後設定的一方有效
% \SetDeptName{系所中文名字}{Department english name} % 同時設定中英文系所名字
% \SetDeptChiName{系所中文名字}                       % 只設定系所中文名字
% \SetDeptEngName{Department english name}          % 只設定系所英文名字
\SetDeptName
{資訊工程學系}
{Department of Computer Science and Information Engineering}

% ----------------------------------------------------------------------------

% --- 指導老師 Advisor(s) ---
% 在封面上預算了最多3位的空間
% 英文名字固定以Prof.開頭

% 有3種可使用, 用來設定3位老師的名字
% \SetAdvisorNameX{老師的名字}{Professor's name} % 同時設定中英文名字
% \SetAdvisorChiNameX{老師的名字}                % 只設定中文名字
% \SetAdvisorEngNameX{Professor's name}         % 只設定英文名字
% (請修改NameX為NameA, NameB, NameC)

% 使用\SetAdvisorNameA是必須的, 而如果你的指導教授有2或3位,
% 那只要增加\SetAdvisorNameB和\SetAdvisorNameC則可
\SetAdvisorNameA{X 教授}{Professor X}

% ----------------------------------------------------------------------------

