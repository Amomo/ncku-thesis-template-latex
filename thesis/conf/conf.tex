
% This file is need to encoded in utf-8
%
% Choose or fill in some needed information for this thesis or dissertation
% 選擇或填入你的論文一些需要使用的資料

% ----------------------------------------------------------------------------

% --- 使用的論文內容 ---
% 如果沒有打開\DemoMode
% 就會使用'./context/context.tex'中你所編寫論文內容.
% 否則會使用'./example/context.tex'的模版說明文件內容.

\DemoMode

% ----------------------------------------------------------------------------

% --- 行距 ---
% 同學可自行設定每行的距離, 這邊是以放大縮小方式來使用.
% 所以是輸入 0.1, 0.5, 1, 1.0, 1.5, 2.0, 2 等數字.
% 預設的行距: 1.2

%\SetLineStretch{1.2}

% ----------------------------------------------------------------------------

% --- 封面上語言和名字顯示方式 ---
%
% \DisplayCoverInChi:  封面以全中文顯示
% \DisplayCoverInEng:  封面以全英文顯示
% 只能選擇其中一個, 但只有最後設定的一方有效
% 預設使用\DisplayCoverInEng
% 
% 另外預設在封面上只會顯示中文或英文名字而已.
% 不論你是使用\DisplayCoverInChi或\DisplayCoverInEng,
% 使用\DisplayCoverPeoplesBothNames以設定同時顯示中英文名字.

%\DisplayCoverInChi
\DisplayCoverInEng
\DisplayCoverPeoplesBothNames

% ----------------------------------------------------------------------------

% --- Title 論文題目 ---
% 填寫中文和(或)英文
% 如果題目內有必須以數學模式表示的符號,請用 \mbox{} 包住數學模式
% 如果覺得自動產生出來的題目斷行位置不適合, 可以手動加'\\'來強制斷行
% (圖書館說不管你是編寫中英混合或全英文版, 都必須同時存在中英題目)
%
% 有3種可使用, 可獨立使用, 但只有最後設定的一方有效
% \SetTitle{你的題目}{Your Title}   % 同時設定中英文題目
% \SetChiTitle{你的題目}            % 只設定中文題目
% \SetEngTitle{Your Title}         % 只設定英文題目
%
% e.g:
%
% \SetTitle %
% {國立成功大學碩博士用畢業論文LaTex模版} %
% {National Cheng Kung University (NCKU) Thesis/Dissertation Template in LaTex}
%
% or
%
% \SetChiTitle{國立成功大學碩博士用畢業論文LaTex模版}
% \SetEngTitle{National Cheng Kung University (NCKU) \\Thesis/Dissertation Template in LaTex}

\SetTitle %
{國立成功大學碩博士用畢業論文LaTex模版} %
{National Cheng Kung University (NCKU) \\Thesis/Dissertation Template in LaTex}

% ----------------------------------------------------------------------------

% --- Draft 初稿 ---
% 顯示 '(初稿)' (中文版) 和 '(Draft)' (英文版) 在封面
\DisplayDraft

% ----------------------------------------------------------------------------

% --- Degree name 學位 ---
%
% 有2種可選擇, 但只有最後設定的一方有效
% \PhdDegree    % 博士學位
% \MasterDegree % 碩士學位

\PhdDegree

% ----------------------------------------------------------------------------

% --- Your name 你的名字 ---
% 填寫你的中文和(或)英文

% 有3種可使用, 可獨立使用, 但只有最後設定的一方有效
% \SetMyName{你的名字}{Your name}   % 同時設定你的中英文名字
% \SetMyChiName{你的名字}           % 只設定你的中文名字
% \SetMyEngName{Your name}         % 只設定你的英文名字

\SetMyName{你的名字}{Your name}

% ----------------------------------------------------------------------------

% --- Date 日期 ---

% 依本校研究生學位考試細則第十條規定:
%
% 碩士班:
%   論文日期:上學期為〇〇〇年1月;下學期為〇〇〇年6月,以該學期結束日期(一月或六月)為準。
%   (如:在上學期101年9月~102年1月期間口試,
%       不論是在此期間何月份口試,其日期均固定為102年1月).
%   另碩士生如101上學期完成口試,101下學期申請出國,102上學期辦理離校,
%   則論文封面為103年1月
%
% 博士班:
%   以當學期通過學位口試,則論文日期為口試日期(如〇〇〇年〇〇月〇〇日),
%   若論文有修改致延至次學期,則以論文上傳日期為主。
%
% 故本模版會根據你的學位, 來選擇顯示在封面的日期格式.
%

% --- 論文封面上的日期 ---
% 設定西元的年月, 會自動計算出民國的年份, 和英文的月份轉換
% 次序: {年份}{月份}
% \SetCoverDate{2014}{12}

\SetCoverDate{2014}{12}

%--------------------------------------------------

% --- 口試的日期 ---
% 設定西元的年月日, 會自動計算出民國的年份, 和英文的月份轉換
% 次序: {年份}{月份}{日}
% \SetOralDate{2014}{12}{31}

\SetOralDate{2014}{12}{31}

% ----------------------------------------------------------------------------

% --- 系所 Department or Institute ---
%
% 設定你的系所名字, e.g:
% \SetDeptMath 數學系
% \SetDeptCSIE 資訊工程學系

\SetDeptCSIE

% ----------------------------------------------------------------------------

% --- 指導老師 Advisor(s) ---
% 在封面上預算了最多3位的空間
% 中文名字固定以 博士  為結尾
% 英文名字固定以 Dr. 為開頭

% 有3種可使用, 用來設定3位老師的名字
% \SetAdvisorNameX{老師的名字}{Professor's name} % 同時設定中英文名字
% \SetAdvisorChiNameX{老師的名字}                % 只設定中文名字
% \SetAdvisorEngNameX{Professor's name}         % 只設定英文名字
% (NameX為NameA, NameB, NameC)

% 使用\SetAdvisorNameA是必須的, 而如果你的指導教授有2或3位,
% 那只要增加\SetAdvisorNameB和\SetAdvisorNameC則可

\SetAdvisorNameA{A}{A}
\SetAdvisorNameB{B}{B}
\SetAdvisorNameC{C}{C}

% ----------------------------------------------------------------------------

% --- 口試証明文件 Oral presentation document ---
% 使用範例版本 或 使用檔案 只能選擇其中一方

% 使用口試範例版本
\DisplayOralTemplate

% --- 範例版本的語言 ---
% 選擇你需要的範例
% (Only work with \DisplayOralTemplate)
% \DisplayOralChiTemplate    % 顯示中文範例版本
% \DisplayOralEngTemplate    % 顯示英文範例版本

\DisplayOralChiTemplate    % 顯示中文範例版本
\DisplayOralEngTemplate    % 顯示英文範例版本

% --- 口試委員 Committee member(s) ---
% 口試委員數量 (至少2位, 最多9位, 預設為9位)
% (Only work with \DisplayOralTemplate)
% 博士學位考試委員會置委員五人至九人
% 碩士學位考試委員會置委員三人至五人
% 口試委員人數含指導教授
\SetCommitteeSize{9}

%--------------------------------------------------

% 使用口試圖片檔案
% 把你的圖片放在'context/oral'下
% 之後設定中英文版所對應是哪一個檔案
% 就算已啟用\DisplayOralImage,
% 但沒有填寫圖檔檔名的話, 都不會顯示出來.
% (例子用的'example-oral-chi.pdf'和'example-oral-eng.pdf'已放在'context/oral'中)

%\DisplayOralImage                % 顯示圖檔
%\SetOralImageChi{example-oral-chi.pdf}   % 中文口試檔案
%\SetOralImageEng{example-oral-eng.pdf}   % 英文口試檔案

% ----------------------------------------------------------------------------

% --- 關鍵字 Keyword ---
% 最多9個關鍵字
% 為了方便同學自行設定
% 故所產出來的PDF檔案中的關鍵字和內文摘要的關鍵字
% 可獨立個別設定

% \SetKeywords是設定所產出來的PDF中的Keyword項目
% 可同時填寫中英文
% e.g
% \SetKeywords{Keyword A (關鍵字 A)}{Keyword B (關鍵字 B)}{Keyword C (關鍵字 C)}
% 或單純中文或英文
% \SetKeywords{Keyword A}{Keyword B}{Keyword C}
% \SetKeywords{關鍵字 A}{關鍵字 B}{關鍵字 C}

\SetKeywords{NCKU Thesis/Dissertation template}{Graduate}{LaTex/XeLaTex}

% 摘要中的關鍵字
% 為了方便同學們能達到以下情況:
% a. 只寫中文版摘要
% b. 只寫英文版摘要
% c. 同時寫中英文版摘要
% 故中英文版的關鍵字都是可個別設定
% \SetAbstractChiKeywords: 用來設定中文版摘要的關鍵字
% \SetAbstractEngKeywords: 用來設定英文版摘要的關鍵字
% \SetAbstractExtKeywords: 用來設定英文延伸摘要的關鍵字 (只有你要編寫英文延伸摘要才需要設定)
% 所以只要使用你需要寫的版本則可.
% 但如果2個版本都要寫, 則2個都同時使用則可.
% 沒有填寫的話, 則摘要中的關鍵字部份是不會顯示出來.
%
% e.g
% \SetAbstractChiKeywords{關鍵字 A}{關鍵字 B}{關鍵字 C}
% \SetAbstractEngKeywords{Keyword A}{Keyword B}{Keyword C}
% \SetAbstractExtKeywords{Keyword A}{Keyword B}{Keyword C}
% 英文延伸摘要的關鍵字理應會跟英文版摘要的關鍵字是一樣,
% 但為了同學能編寫不同內容和關鍵字, 故可獨立設定.

\SetAbstractChiKeywords{國立成功大學畢業論文模版}{碩博士}{LaTex/XeLaTex}
\SetAbstractEngKeywords{NCKU Thesis/Dissertation Template}{Graduate}{LaTex/XeLaTex}
\SetAbstractExtKeywords{NCKU Thesis/Dissertation Template}{Graduate}{LaTex/XeLaTex}

% ----------------------------------------------------------------------------

% --- 目錄 Index ---
% 設定可獨立使用, 但只有最後設定的一方有效

% 標題文字語言 Language
% 目錄的標題文字使用預設的中文或是英文
% \IndexChiMode:  標題文字為中文
% \IndexEngMode:  標題文字為英文
% 預設使用\IndexEngMode

%\IndexChiMode
\IndexEngMode

% 標題文字 Text of title
% 預設的目錄標題為: 目錄 (中文) / Table of Contents (英文)
% 預設的表格目錄標題為: 表格 (中文) / List of Tables (英文)
% 預設的圖片目錄標題為: 圖片 (中文) / List of Figures (英文)
% 如應為預設文字不是你所希望的, 那可以使用這邊去個別設定你所希望的文字, 不分中英文.

% 設定目錄標題
%\SetIndexTitleText{Table of Contents / 目錄}

% 設定表格目錄標題
%\SetTablesIndexTitleText{List of Tables / 表格}

% 設定圖片目錄標題
%\SetFiguresIndexTitleText{List of Figures / 圖片}

% ----------------------------------------------------------------------------

% --- 圖片相關的設定 ---
% 預設上每一張圖的名字都是以 'Figure 2.1'
% 假如想使用自定的名字, 如 '圖 2.1'
% 則使用 \SetCustomFigureName{圖} 即可.

%\SetCustomFigureName{Figure}

% ----------------------------------------------------------------------------

% --- 表格相關的設定 ---
% 預設上每一張表的名字都是以 'Table 2.1'
% 假如想使用自定的名字, 如 '表 2.1'
% 則使用 \SetCustomTableName{表} 即可.

%\SetCustomTableName{Table}

% ----------------------------------------------------------------------------

% --- 參考文獻 Reference ---
% 設定可獨立使用, 但只有最後設定的一方有效

% Reference的標題文字使用預設的中文或是英文
% 預設的標題為: 參考文獻 (中文) / References (英文)
% \ChapterReferenceTitleInChi:  標題文字為中文
% \ChapterReferenceTitleInEng:  標題文字為英文
% 預設使用\ChapterReferenceTitleInEng
%
% 如應為預設文字不是你所希望的,
% 則可使用\SetChapterReferenceTitle去設定你所希望的文字, 不分中英文.

%\ChapterReferenceTitleInChi
%\ChapterReferenceTitleInEng
%\SetChapterReferenceTitle{References / 參考文獻}

% ----------------------

% Reference引用時的格式
% 除非有特殊的格式要求, 否則這部份是不用管的.

%
% 使用的格式  | 	作者名稱顯示的格式          |  引用時顯示的例子
%     abbrv       |     H. J. Simpson                  |                [4]
%     plain        |     Homer Jay Simpson     |                [4]
%     alpha      |     Homer Jay Simpson     |            Sim95
%    apacite  |     Homer J. S.                       |        Homer, 1995
% 預設使用plain
%
% 注意: 如果你要轉換使用格式時, 推薦在重新產生論文前, 先把所有除了thesis.tex外的所有
% thesis開頭或以thesis為檔名的檔案全刪掉. 例如'thesis.bbl', 'thesis.aux', 'thesis.lof'等所有檔案.
% 否則有可能在產生論文時遇到錯誤, 如果遇到錯誤, 請不斷重新刪掉和重新產生論文,
% 直到解決問題為止.
% 已知: 由abbrv轉去apacite必定需要刪除檔案才能進行.
%

%\BibStyleUseAbbrv
%\BibStyleUsePlain
%\BibStyleUseAlpha
%\BibStyleUseApacite

% ----------------------------------------------------------------------------

% --- 章節標題的設定 ---

% --- 數字 ---
% 可自行設計你想要的數字和格式
%
% 有以下的數字類型提供
%   1. 阿拉伯數字
%   2. 羅馬數字 (大小寫)
%   3. 英文字母  (大小寫)
%   4. 天干 (如: 甲乙丙丁戊癸)
%   5. 中文數字 (如: 一二三)
%
% 由於是可組合出不同的例子,
% 請同學自行慢慢研究和嘗試去弄出自己想要的樣子.
%
% ----------------------
%
% --- 使用方式 ---
%
% 章節意思
% Chapter (章序號): 1
% Section (節): 1.1
% SubSection (小節): 1.1.1
% SubSubSection (小小節): 1.1.1.1
%
% 使用以下的寫法去設定不同地方數字和格式
%\ChapterTitleNumFormat{ < 格式 >}    % chapter 	(章序號)
%\SectionTitleNumFormat{ < 格式 >}    % section 	(節)
%\SubSectionTitleNumFormat{ < 格式 >}    % subsection 	(小節)
%\SubSubSectionTitleNumFormat{ < 格式 >}    % subsubsection 	(小小節)
%
% ----------------------
%
% --- 數字類型 ---
%
% Chapter (章序號)
%    \StyleCNumChiNum 章序號是使用 '中文數字' 方式
%    \StyleCNumTiangan 章序號是使用 '天干' 方式
%    \StyleCNumArabic 章序號是使用 '阿拉伯數字' 方式
%    \StyleCNumLowerRoman 章序號是使用 '小寫羅馬數字' 方式
%    \StyleCNumUpperRoman 章序號是使用 '大寫羅馬數字' 方式
%    \StyleCNumLowerAlph 章序號是使用 '小寫英文字母' 方式
%    \StyleCNumUpperAlph 章序號是使用 '大寫英文字母' 方式
%
% Section (節)
%    \StyleSNumChiNum 節是使用 '中文數字' 方式
%    \StyleSNumTiangan 節是使用 '天干' 方式
%    \StyleSNumArabic 節是使用 '阿拉伯數字' 方式
%    \StyleSNumLowerRoman 節是使用 '小寫羅馬數字' 方式
%    \StyleSNumUpperRoman 節是使用 '大寫羅馬數字' 方式
%    \StyleSNumLowerAlph 節是使用 '小寫英文字母' 方式
%    \StyleSNumUpperAlph 節是使用 '大寫英文字母' 方式
%
% SubSection (小節)
%    \StyleSSNumChiNum 小節是使用 '中文數字' 方式
%    \StyleSSNumTiangan 小節是使用 '天干' 方式
%    \StyleSSNumArabic 小節是使用 '阿拉伯數字' 方式
%    \StyleSSNumLowerRoman 小節是使用 '小寫羅馬數字' 方式
%    \StyleSSNumUpperRoman 小節是使用 '大寫羅馬數字' 方式
%    \StyleSSNumLowerAlph 小節是使用 '小寫英文字母' 方式
%    \StyleSSNumUpperAlph 小節是使用 '大寫英文字母' 方式
%
% SubSubSection (小小節)
%    \StyleSSSNumChiNum 小小節是使用 '中文數字' 方式
%    \StyleSSSNumTiangan 小小節是使用 '天干' 方式
%    \StyleSSSNumArabic 小小節是使用 '阿拉伯數字' 方式
%    \StyleSSSNumLowerRoman 小小節是使用 '小寫羅馬數字' 方式
%    \StyleSSSNumUpperRoman 小小節是使用 '大寫羅馬數字' 方式
%    \StyleSSSNumLowerAlph 小小節是使用 '小寫英文字母' 方式
%    \StyleSSSNumUpperAlph 小小節是使用 '大寫英文字母' 方式
%
% ----------------------
%
% --- 格式 ---
% < 格式 > 正是由文字和數字類型所組出來
%
% 預設的格式:
% Chapter: Chapter 1
% Section: 1.1
% SubSection: 1.1.1
% SubSubSection: (空白, 只有題目)
%
% 如果 '章' 要由文字改使用為:
%        'Chapter 1' -> '第 1 章'
% 則使用
%        \ChapterTitleNumFormat{第\StyleCNumArabic 章}
% (注意在'章'字前必須有一個空白, 否則會被當成\StyleCNumArabic的字元之一, 
 %  這是基於LaTex的寫法, 所以請慢慢嘗試和注意Style前後字元能不能連在一起)
%
% 如果 '章' 要由數字改使用為:
%        '1' -> '-A-'
% 則使用
%        \ChapterTitleNumFormat{Chapter -\StyleCNumUpperAlph-}
%
% 如果 '節' 要由數字改使用為:
%        '1.2' -> '一 -乙-'
% 則使用
%        \SectionTitleNumFormat{\StyleCNumChiNum -\StyleCNumTiangan-}
%
% 如果 '節' 不想看到 '章' 的數字:
%        '1.2' -> '2'
% 則使用
%        \SectionTitleNumFormat{\StyleSNumArabic}
% 直接不提供 '章' 的數字在格式中則可
%
% ----------------------
%
% --- 請在這邊設定你要的樣子 ---
%

%\ChapterTitleNumFormat{Chapter \StyleCNumArabic}
%\SectionTitleNumFormat{%
%  \StyleCNumArabic.\StyleSNumArabic}
%\SubSectionTitleNumFormat{%
%  \StyleCNumArabic.\StyleSNumArabic.\StyleSSNumArabic}
%\SubSubSectionTitleNumFormat{}

% ----------------------------------------------------------------------------

% --- 附錄標題的設定 ---
% 請先參考 章節標題的設定 的使用說明, 使用方式幾乎一樣的.
% 但為了分開2邊的使用方式, 故使用不同的名稱.
%
% ----------------------
%
% --- 數字類型 ---
%
% Chapter (章序號)
%    \StyleAppixCNumChiNum 章序號是使用 '中文數字' 方式
%    \StyleAppixCNumTiangan 章序號是使用 '天干' 方式
%    \StyleAppixCNumArabic 章序號是使用 '阿拉伯數字' 方式
%    \StyleAppixCNumLowerRoman 章序號是使用 '小寫羅馬數字' 方式
%    \StyleAppixCNumUpperRoman 章序號是使用 '大寫羅馬數字' 方式
%    \StyleAppixCNumLowerAlph 章序號是使用 '小寫英文字母' 方式
%    \StyleAppixCNumUpperAlph 章序號是使用 '大寫英文字母' 方式
%
% Section (節)
%    \StyleAppixSNumChiNum 節是使用 '中文數字' 方式
%    \StyleAppixSNumTiangan 節是使用 '天干' 方式
%    \StyleAppixSNumArabic 節是使用 '阿拉伯數字' 方式
%    \StyleAppixSNumLowerRoman 節是使用 '小寫羅馬數字' 方式
%    \StyleAppixSNumUpperRoman 節是使用 '大寫羅馬數字' 方式
%    \StyleAppixSNumLowerAlph 節是使用 '小寫英文字母' 方式
%    \StyleAppixSNumUpperAlph 節是使用 '大寫英文字母' 方式
%
% SubSection (小節)
%    \StyleAppixSSNumChiNum 小節是使用 '中文數字' 方式
%    \StyleAppixSSNumTiangan 小節是使用 '天干' 方式
%    \StyleAppixSSNumArabic 小節是使用 '阿拉伯數字' 方式
%    \StyleAppixSSNumLowerRoman 小節是使用 '小寫羅馬數字' 方式
%    \StyleAppixSSNumUpperRoman 小節是使用 '大寫羅馬數字' 方式
%    \StyleAppixSSNumLowerAlph 小節是使用 '小寫英文字母' 方式
%    \StyleAppixSSNumUpperAlph 小節是使用 '大寫英文字母' 方式
%
% SubSubSection (小小節)
%    \StyleAppixSSSNumChiNum 小小節是使用 '中文數字' 方式
%    \StyleAppixSSSNumTiangan 小小節是使用 '天干' 方式
%    \StyleAppixSSSNumArabic 小小節是使用 '阿拉伯數字' 方式
%    \StyleAppixSSSNumLowerRoman 小小節是使用 '小寫羅馬數字' 方式
%    \StyleAppixSSSNumUpperRoman 小小節是使用 '大寫羅馬數字' 方式
%    \StyleAppixSSSNumLowerAlph 小小節是使用 '小寫英文字母' 方式
%    \StyleAppixSSSNumUpperAlph 小小節是使用 '大寫英文字母' 方式
%
% ----------------------
%
% --- 格式 ---
% < 格式 > 正是由文字和數字類型所組出來
%
% 預設的格式:
% Chapter: Appendix A
% Section: A.1
% SubSection: A.1.1
% SubSubSection: (空白, 只有題目)
%
% 如果 '章' 要由文字改使用為:
%        'Appendix A' -> '附錄 A'
% 則使用
%        \AppendixChapterTitleNumFormat{附錄 \StyleAppixCNumUpperAlph}
%
% 其他的使用方式跟 章節標題的設定 相同.
%
% ----------------------
%
% --- 請在這邊設定你要的樣子 ---
%

%\AppendixChapterTitleNumFormat{Appendix \StyleAppixCNumUpperAlph}
%\AppendixSectionTitleNumFormat{%
%  \StyleAppixCNumUpperAlph.\StyleAppixSNumArabic}
%\AppendixSubSectionTitleNumFormat{%
%  \StyleAppixCNumUpperAlph.\StyleAppixSNumArabic.\StyleAppixSSNumArabic}
%\AppendixSubSubSectionTitleNumFormat{}

% ----------------------------------------------------------------------------
