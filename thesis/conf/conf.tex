
% This file is need to encoded in utf-8
% 填入你的論文一些需要使用的資料

% ----------------------------------------------------------------------------
%\LangChi  % For Taiwan students
%\LangEng  % For foreign students
\LangDemo % default
% ----------------------------------------------------------------------------

% --- Title 論文題目 ---
% 填寫中文和(或)英文
% 如果題目內有必須以數學模式表示的符號,請用 \mbox{} 包住數學模式
% 如果覺得自動產生出來的題目斷行位置不適合, 可以手動加'\\'來強制斷行
%
% 有3種可使用, 可獨立使用, 但只有最後設定的一方有效
% \SetTitle{你的題目}{Your Title}   % 同時設定中英文題目
% \SetChiTitle{你的題目}            % 只設定中文題目
% \SetEngTitle{Your Title}         % 只設定英文題目

\SetTitle
{國立成功大學碩博士用畢業論文XeLaTex模板}
{National Cheng Kung University (NCKU) Thesis/Dissertation Template in XeLaTex}
% \SetChiTitle{國立成功大學碩博士用畢業論文XeLaTex模板}
% \SetEngTitle{National Cheng Kung University (NCKU)\\
% Thesis/Dissertation Template in XeLaTex}

% ----------------------------------------------------------------------------

% --- Degree name 學位 ---
% 不用填寫, 只要選擇即可

% 有2種可使用, 但只有最後設定的一方有效
% \PhdDegree    % 博士學位
% \MasterDegree % 碩士學位

%\MasterDegree
\PhdDegree

% ----------------------------------------------------------------------------

% --- Your name 你的名字 ---
% 填寫你的中文和(或)英文

% 有3種可使用, 可獨立使用, 但只有最後設定的一方有效
% \SetMyName{你的名字}{Your name}   % 同時設定你的中英文名字
% \SetMyChiName{你的名字}           % 只設定你的中文名字
% \SetMyEngName{Your name}         % 只設定你的英文名字

\SetMyName{你的名字}{Your name}

% ----------------------------------------------------------------------------

% --- Date 日期 ---

% --- 論文封面上的日期 ---
% 設定西元的年月日, 會自動計算出民國的年份, 和英文的月份轉換
% 次序: {年份}{月份}

\SetThesisDate{2014}{12}

%--------------------------------------------------

% ---  口試的日期 (可不用設定, 請參考內文 '口試資料' 中的說明) ---
% 設定西元的年月日, 會自動計算出民國的年份, 和英文的月份轉換
% 次序: {年份}{月份}{日}
% \SetOralDate{2014}{12}{31}

\SetOralDate{2014}{12}{31}

% ----------------------------------------------------------------------------

% --- 系所 Department or Institute ---
% 填寫你的系所名字

% 有3種可使用, 可獨立使用, 但只有最後設定的一方有效
% \SetDeptName{系所中文名字}{Department english name} % 同時設定中英文系所名字
% \SetDeptChiName{系所中文名字}                       % 只設定系所中文名字
% \SetDeptEngName{Department english name}          % 只設定系所英文名字

\SetDeptName
{資訊工程學系}
{Insitute of Computer Science and Information Engineering}

% ----------------------------------------------------------------------------

% --- 指導老師 Advisor(s) ---
% 在封面上預算了最多3位的空間
% 中文名字固定以 教授  為結尾
% 英文名字固定以 Prof. 為開頭

% 有3種可使用, 用來設定3位老師的名字
% \SetAdvisorNameX{老師的名字}{Professor's name} % 同時設定中英文名字
% \SetAdvisorChiNameX{老師的名字}                % 只設定中文名字
% \SetAdvisorEngNameX{Professor's name}         % 只設定英文名字
% (請修改NameX為NameA, NameB, NameC)

% 使用\SetAdvisorNameA是必須的, 而如果你的指導教授有2或3位,
% 那只要增加\SetAdvisorNameB和\SetAdvisorNameC則可

\SetAdvisorNameA{A}{A}
%\SetAdvisorNameB{B}{B}
%\SetAdvisorNameC{C}{C}

% ----------------------------------------------------------------------------

% --- 口試証明文件 Document of oral presentation ---

% 要顯示 圖片 / 範例
\DisplayOralTemplete % 顯示範例
%\DisplayOralImage    % 顯示圖片
%\SetOralImageChi{}
%\SetOralImageEng{}

% --- 口試委員 Committee member(s) ---
% 口試委員數量 (至少4位, 最多8位, 預設為8位)
% (Only work with \DisplayOralTemplete)
\SetCommitteeSize{8}

% ----------------------------------------------------------------------------

% 關鍵字 Keyword

%\SetKeywords{Keyword A}{Keyword B}{Keyword C}{Keyword D}{Keyword E}
\SetKeywords{NCKU Thesis/Dissertation templete}{Graduate}{Latex/XeLaTex}

% ----------------------------------------------------------------------------

% 書脊 Spine

% Title (For \LangChi only)
% Default using the english title
% Use the to using chinese as title
%\SpineTitleChi

% ----------------------------------------------------------------------------

