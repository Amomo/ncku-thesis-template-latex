%
% This file is part of the project of
% National Cheng Kung University (NCKU) Thesis/Dissertation Template in LaTex.
% This project is hold at
%     <https://github.com/wengan-li/ncku-thesis-template-latex>
% by Wen-Gan Li.
%
% This project is distributed in the hope of usefuling to someone,
% you can redistribute it and/or modify it under the terms of the
% Attribution-NonCommercial-ShareAlike 4.0 International.
%
% You should have received a copy of the
% Attribution-NonCommercial-ShareAlike 4.0 International
% along with this project.
% If not, see <http://creativecommons.org/licenses/by-nc-sa/4.0/legalcode.txt>.
%
% Please feel free to fork it, modify it, and try it.
% Have fun !!!
%

% ----------------------------------------------------------------------------

% Reference from:
%   <https://www.sharelatex.com/learn/Theorems_and_proofs>

%    definition       (定義)
%    condition        (條件)
%    theorem          (定理)
%    lemma            (引理)
%    example          (例子)
%    corollary        (推論)
%    proposition      (主張)
%    remark           (備註)
%    proof            (證明)
%    conjectures      (猜想)
%    note             (附註)
%    annotation       (註解)
%    claim            (主張)
%    case             (情況)
%    acknowledgment   (確認)
%    conclusion       (結論)

% ----------------------------------------------------------------------------

% Shared functions

\pgfkeys
{
  /InsertTheoremOptions/.is family, /InsertTheoremOptions,
  default/.style =
  {
    title = \empty,
    label = \empty,
  },
  title/.estore in = \TmpValueTitle,
  label/.estore in = \TmpValueLabel,
} % End of \pgfkeys{}

\newcommand{\SetTheoremContentLabel}[1][\empty]
{%
  \ifthenelse{\equal{#1}{\empty}}{}{\IfNoValueF{#1}{\label{#1}}}%
} % End of \DeclareDocumentCommand{}

\newcommand{\InsertTheoremContent}[3][\empty]
{%
  % Parse the input
  \pgfkeys{/InsertTheoremOptions, default, #1}%
  %
  \ifthenelse{\equal{\TmpValueTitle}{\empty}}%
  {%
    \begin{#2}%
    \SetTheoremContentLabel{\TmpValueLabel}%
    #3%
    \end{#2}%
  }%
  {%
    \begin{#2}[\TmpValueTitle]%
    \SetTheoremContentLabel{\TmpValueLabel}%
    #3%
    \end{#2}%
  }%
} % End of \newcommand{}

% ---------------------------------------------------------

\pgfkeys
{
  /TheoremTheoremFormat/.is family, /TheoremTheoremFormat,
  default/.style =
  {
    EnvironmentName = {EnvTheorem},
    ShowText = {Theorem},
    FollowCounter = section,
  },
  EnvironmentName/.estore in = \GetTheoremTheoremFormatEnvironmentName,
  ShowText/.estore in = \GetTheoremTheoremFormatShowText,
  FollowCounter/.estore in = \GetTheoremTheoremFormatFollowCounter,
} % End of \pgfkeys{}

\newcommand{\InsertTheorem}[2][\empty]
{%
  \InsertTheoremContent[#1]{\GetTheoremTheoremFormatEnvironmentName}{#2}%
} % End of \newcommand{}

\newcommand{\InitTheoremTheoremFormat}
{%
  \ifthenelse{\equal{\GetTheoremTheoremFormatFollowCounter}{\empty}}%
  {%
    \newtheorem{%
      \GetTheoremTheoremFormatEnvironmentName}{%
      \GetTheoremTheoremFormatShowText}%
  }%
  {%
    \newtheorem{%
      \GetTheoremTheoremFormatEnvironmentName}{%
      \GetTheoremTheoremFormatShowText}[%
      \GetTheoremTheoremFormatFollowCounter]%
  }%
} % End of \newcommand{}

% ---------------------------------------------------------

\pgfkeys
{
  /TheoremDefinitionFormat/.is family, /TheoremDefinitionFormat,
  default/.style =
  {
    EnvironmentName = {EnvDefinition},
    ShowText = {Definition},
    FollowCounter = section,
  },
  EnvironmentName/.estore in = \GetTheoremDefinitionFormatEnvironmentName,
  ShowText/.estore in = \GetTheoremDefinitionFormatShowText,
  FollowCounter/.estore in = \GetTheoremDefinitionFormatFollowCounter,
} % End of \pgfkeys{}

\newcommand{\InsertDefinition}[2][\empty]
{%
  \InsertTheoremContent[#1]{\GetTheoremDefinitionFormatEnvironmentName}{#2}%
} % End of \newcommand{}

\newcommand{\InitTheoremDefinitionFormat}
{%
  \ifthenelse{\equal{\GetTheoremDefinitionFormatFollowCounter}{\empty}}%
  {%
    \newtheorem{%
      \GetTheoremDefinitionFormatEnvironmentName}{%
      \GetTheoremDefinitionFormatShowText}%
  }%
  {%
    \newtheorem{%
      \GetTheoremDefinitionFormatEnvironmentName}{%
      \GetTheoremDefinitionFormatShowText}[%
      \GetTheoremDefinitionFormatFollowCounter]%
  }%
} % End of \newcommand{}

% ---------------------------------------------------------

\pgfkeys
{
  /TheoremConditionFormat/.is family, /TheoremConditionFormat,
  default/.style =
  {
    EnvironmentName = {EnvCondition},
    ShowText = {Condition},
    FollowCounter = section,
  },
  EnvironmentName/.estore in = \GetTheoremConditionFormatEnvironmentName,
  ShowText/.estore in = \GetTheoremConditionFormatShowText,
  FollowCounter/.estore in = \GetTheoremConditionFormatFollowCounter,
} % End of \pgfkeys{}

\newcommand{\InsertCondition}[2][\empty]
{%
  \InsertTheoremContent[#1]{\GetTheoremConditionFormatEnvironmentName}{#2}%
} % End of \newcommand{}

\newcommand{\InitTheoremConditionFormat}
{%
  \ifthenelse{\equal{\GetTheoremConditionFormatFollowCounter}{\empty}}%
  {%
    \newtheorem{%
      \GetTheoremConditionFormatEnvironmentName}{%
      \GetTheoremConditionFormatShowText}%
  }%
  {%
    \newtheorem{%
      \GetTheoremConditionFormatEnvironmentName}{%
      \GetTheoremConditionFormatShowText}[%
      \GetTheoremConditionFormatFollowCounter]%
  }%
} % End of \newcommand{}

% ---------------------------------------------------------

\pgfkeys
{
  /TheoremProblemFormat/.is family, /TheoremProblemFormat,
  default/.style =
  {
    EnvironmentName = {EnvProblem},
    ShowText = {Problem},
    FollowCounter = section,
  },
  EnvironmentName/.estore in = \GetTheoremProblemFormatEnvironmentName,
  ShowText/.estore in = \GetTheoremProblemFormatShowText,
  FollowCounter/.estore in = \GetTheoremProblemFormatFollowCounter,
} % End of \pgfkeys{}

\newcommand{\InsertProblem}[2][\empty]
{%
  \InsertTheoremContent[#1]{\GetTheoremProblemFormatEnvironmentName}{#2}%
} % End of \newcommand{}

\newcommand{\InitTheoremProblemFormat}
{%
  \ifthenelse{\equal{\GetTheoremProblemFormatFollowCounter}{\empty}}%
  {%
    \newtheorem{%
      \GetTheoremProblemFormatEnvironmentName}{%
      \GetTheoremProblemFormatShowText}%
  }%
  {%
    \newtheorem{%
      \GetTheoremProblemFormatEnvironmentName}{%
      \GetTheoremProblemFormatShowText}[%
      \GetTheoremProblemFormatFollowCounter]%
  }%
} % End of \newcommand{}

% ---------------------------------------------------------

\pgfkeys
{
  /TheoremExampleFormat/.is family, /TheoremExampleFormat,
  default/.style =
  {
    EnvironmentName = {EnvExample},
    ShowText = {Example},
    FollowCounter = section,
  },
  EnvironmentName/.estore in = \GetTheoremExampleFormatEnvironmentName,
  ShowText/.estore in = \GetTheoremExampleFormatShowText,
  FollowCounter/.estore in = \GetTheoremExampleFormatFollowCounter,
} % End of \pgfkeys{}

\newcommand{\InsertExample}[2][\empty]
{%
  \InsertTheoremContent[#1]{\GetTheoremExampleFormatEnvironmentName}{#2}%
} % End of \newcommand{}

\newcommand{\InitTheoremExampleFormat}
{%
  \ifthenelse{\equal{\GetTheoremExampleFormatFollowCounter}{\empty}}%
  {%
    \newtheorem{%
      \GetTheoremExampleFormatEnvironmentName}{%
      \GetTheoremExampleFormatShowText}%
  }%
  {%
    \newtheorem{%
      \GetTheoremExampleFormatEnvironmentName}{%
      \GetTheoremExampleFormatShowText}[%
      \GetTheoremExampleFormatFollowCounter]%
  }%
} % End of \newcommand{}

% ---------------------------------------------------------

\pgfkeys
{
  /TheoremLemmaFormat/.is family, /TheoremLemmaFormat,
  default/.style =
  {
    EnvironmentName = {EnvLemma},
    ShowText = {Lemma},
    FollowCounter = section,
  },
  EnvironmentName/.estore in = \GetTheoremLemmaFormatEnvironmentName,
  ShowText/.estore in = \GetTheoremLemmaFormatShowText,
  FollowCounter/.estore in = \GetTheoremLemmaFormatFollowCounter,
} % End of \pgfkeys{}

\newcommand{\InsertLemma}[2][\empty]
{%
  \InsertTheoremContent[#1]{\GetTheoremLemmaFormatEnvironmentName}{#2}%
} % End of \newcommand{}

\newcommand{\InitTheoremLemmaFormat}
{%
  \ifthenelse{\equal{\GetTheoremLemmaFormatFollowCounter}{\empty}}%
  {%
    \newtheorem{%
      \GetTheoremLemmaFormatEnvironmentName}{%
      \GetTheoremLemmaFormatShowText}%
  }%
  {%
    \newtheorem{%
      \GetTheoremLemmaFormatEnvironmentName}{%
      \GetTheoremLemmaFormatShowText}[%
      \GetTheoremLemmaFormatFollowCounter]%
  }%
} % End of \newcommand{}

% ---------------------------------------------------------

\pgfkeys
{
  /TheoremCorollarieFormat/.is family, /TheoremCorollarieFormat,
  default/.style =
  {
    EnvironmentName = {EnvCorollarie},
    ShowText = {Corollarie},
    FollowCounter = section,
  },
  EnvironmentName/.estore in = \GetTheoremCorollarieFormatEnvironmentName,
  ShowText/.estore in = \GetTheoremCorollarieFormatShowText,
  FollowCounter/.estore in = \GetTheoremCorollarieFormatFollowCounter,
} % End of \pgfkeys{}

\newcommand{\InsertCorollarie}[2][\empty]
{%
  \InsertTheoremContent[#1]{\GetTheoremCorollarieFormatEnvironmentName}{#2}%
} % End of \newcommand{}

\newcommand{\InitTheoremCorollarieFormat}
{%
  \ifthenelse{\equal{\GetTheoremCorollarieFormatFollowCounter}{\empty}}%
  {%
    \newtheorem{%
      \GetTheoremCorollarieFormatEnvironmentName}{%
      \GetTheoremCorollarieFormatShowText}%
  }%
  {%
    \newtheorem{%
      \GetTheoremCorollarieFormatEnvironmentName}{%
      \GetTheoremCorollarieFormatShowText}[%
      \GetTheoremCorollarieFormatFollowCounter]%
  }%
} % End of \newcommand{}

% ---------------------------------------------------------

\pgfkeys
{
  /TheoremPropositionFormat/.is family, /TheoremPropositionFormat,
  default/.style =
  {
    EnvironmentName = {EnvProposition},
    ShowText = {Proposition},
    FollowCounter = section,
  },
  EnvironmentName/.estore in = \GetTheoremPropositionFormatEnvironmentName,
  ShowText/.estore in = \GetTheoremPropositionFormatShowText,
  FollowCounter/.estore in = \GetTheoremPropositionFormatFollowCounter,
} % End of \pgfkeys{}

\newcommand{\InsertProposition}[2][\empty]
{%
  \InsertTheoremContent[#1]{\GetTheoremPropositionFormatEnvironmentName}{#2}%
} % End of \newcommand{}

\newcommand{\InitTheoremPropositionFormat}
{%
  \ifthenelse{\equal{\GetTheoremPropositionFormatFollowCounter}{\empty}}%
  {%
    \newtheorem{%
      \GetTheoremPropositionFormatEnvironmentName}{%
      \GetTheoremPropositionFormatShowText}%
  }%
  {%
    \newtheorem{%
      \GetTheoremPropositionFormatEnvironmentName}{%
      \GetTheoremPropositionFormatShowText}[%
      \GetTheoremPropositionFormatFollowCounter]%
  }%
} % End of \newcommand{}

% ---------------------------------------------------------

\pgfkeys
{
  /TheoremConjectureFormat/.is family, /TheoremConjectureFormat,
  default/.style =
  {
    EnvironmentName = {EnvConjecture},
    ShowText = {Conjecture},
    FollowCounter = section,
  },
  EnvironmentName/.estore in = \GetTheoremConjectureFormatEnvironmentName,
  ShowText/.estore in = \GetTheoremConjectureFormatShowText,
  FollowCounter/.estore in = \GetTheoremConjectureFormatFollowCounter,
} % End of \pgfkeys{}

\newcommand{\InsertConjecture}[2][\empty]
{%
  \InsertTheoremContent[#1]{\GetTheoremConjectureFormatEnvironmentName}{#2}%
} % End of \newcommand{}

\newcommand{\InitTheoremConjectureFormat}
{%
  \ifthenelse{\equal{\GetTheoremConjectureFormatFollowCounter}{\empty}}%
  {%
    \newtheorem{%
      \GetTheoremConjectureFormatEnvironmentName}{%
      \GetTheoremConjectureFormatShowText}%
  }%
  {%
    \newtheorem{%
      \GetTheoremConjectureFormatEnvironmentName}{%
      \GetTheoremConjectureFormatShowText}[%
      \GetTheoremConjectureFormatFollowCounter]%
  }%
} % End of \newcommand{}

% ---------------------------------------------------------

\pgfkeys
{
  /TheoremProofFormat/.is family, /TheoremProofFormat,
  default/.style =
  {
    EnvironmentName = {EnvProof},
    ShowText = {Proof},
    FollowCounter = \empty,
  },
  EnvironmentName/.estore in = \GetTheoremProofFormatEnvironmentName,
  ShowText/.estore in = \GetTheoremProofFormatShowText,
  FollowCounter/.estore in = \GetTheoremProofFormatFollowCounter,
} % End of \pgfkeys{}

\newcommand{\InsertProof}[1]
{%
  \InsertTheoremContent[\empty]{%
    \GetTheoremProofFormatEnvironmentName}{#1}%
} % End of \newcommand{}

\newcommand{\InitTheoremProofFormat}
{%
  \ifthenelse{\equal{\GetTheoremProofFormatFollowCounter}{\empty}}%
  {%
    \newtheorem*{%
      \GetTheoremProofFormatEnvironmentName}{%
      \GetTheoremProofFormatShowText}
  }%
  {%
    \newtheorem{%
      \GetTheoremProofFormatEnvironmentName}{%
      \GetTheoremProofFormatShowText}[%
      \GetTheoremProofFormatFollowCounter]%
  }%
} % End of \newcommand{}

% ---------------------------------------------------------

\pgfkeys
{
  /TheoremNoteFormat/.is family, /TheoremNoteFormat,
  default/.style =
  {
    EnvironmentName = {EnvNote},
    ShowText = {Note},
    FollowCounter = \empty,
  },
  EnvironmentName/.estore in = \GetTheoremNoteFormatEnvironmentName,
  ShowText/.estore in = \GetTheoremNoteFormatShowText,
  FollowCounter/.estore in = \GetTheoremNoteFormatFollowCounter,
} % End of \pgfkeys{}

\newcommand{\InsertNote}[1]
{%
  \InsertTheoremContent[\empty]{%
    \GetTheoremNoteFormatEnvironmentName}{#1}%
} % End of \newcommand{}

\newcommand{\InitTheoremNoteFormat}
{%
  \ifthenelse{\equal{\GetTheoremNoteFormatFollowCounter}{\empty}}%
  {%
    \newtheorem*{%
      \GetTheoremNoteFormatEnvironmentName}{%
      \GetTheoremNoteFormatShowText}
  }%
  {%
    \newtheorem{%
      \GetTheoremNoteFormatEnvironmentName}{%
      \GetTheoremNoteFormatShowText}[%
      \GetTheoremNoteFormatFollowCounter]%
  }%
} % End of \newcommand{}

% ---------------------------------------------------------

\pgfkeys
{
  /TheoremAnnotationFormat/.is family, /TheoremAnnotationFormat,
  default/.style =
  {
    EnvironmentName = {EnvAnnotation},
    ShowText = {Annotation},
    FollowCounter = \empty,
  },
  EnvironmentName/.estore in = \GetTheoremAnnotationFormatEnvironmentName,
  ShowText/.estore in = \GetTheoremAnnotationFormatShowText,
  FollowCounter/.estore in = \GetTheoremAnnotationFormatFollowCounter,
} % End of \pgfkeys{}

\newcommand{\InsertAnnotation}[1]
{%
  \InsertTheoremContent[\empty]{%
    \GetTheoremAnnotationFormatEnvironmentName}{#1}%
} % End of \newcommand{}

\newcommand{\InitTheoremAnnotationFormat}
{%
  \ifthenelse{\equal{\GetTheoremAnnotationFormatFollowCounter}{\empty}}%
  {%
    \newtheorem*{%
      \GetTheoremAnnotationFormatEnvironmentName}{%
      \GetTheoremAnnotationFormatShowText}
  }%
  {%
    \newtheorem{%
      \GetTheoremAnnotationFormatEnvironmentName}{%
      \GetTheoremAnnotationFormatShowText}[%
      \GetTheoremAnnotationFormatFollowCounter]%
  }%
} % End of \newcommand{}

% ---------------------------------------------------------

\pgfkeys
{
  /TheoremClaimFormat/.is family, /TheoremClaimFormat,
  default/.style =
  {
    EnvironmentName = {EnvClaim},
    ShowText = {Claim},
    FollowCounter = \empty,
  },
  EnvironmentName/.estore in = \GetTheoremClaimFormatEnvironmentName,
  ShowText/.estore in = \GetTheoremClaimFormatShowText,
  FollowCounter/.estore in = \GetTheoremClaimFormatFollowCounter,
} % End of \pgfkeys{}

\newcommand{\InsertClaim}[1]
{%
  \InsertTheoremContent[\empty]{%
    \GetTheoremClaimFormatEnvironmentName}{#1}%
} % End of \newcommand{}

\newcommand{\InitTheoremClaimFormat}
{%
  \ifthenelse{\equal{\GetTheoremClaimFormatFollowCounter}{\empty}}%
  {%
    \newtheorem*{%
      \GetTheoremClaimFormatEnvironmentName}{%
      \GetTheoremClaimFormatShowText}
  }%
  {%
    \newtheorem{%
      \GetTheoremClaimFormatEnvironmentName}{%
      \GetTheoremClaimFormatShowText}[%
      \GetTheoremClaimFormatFollowCounter]%
  }%
} % End of \newcommand{}

% ---------------------------------------------------------

\pgfkeys
{
  /TheoremCaseFormat/.is family, /TheoremCaseFormat,
  default/.style =
  {
    EnvironmentName = {EnvCase},
    ShowText = {Case},
    FollowCounter = \empty,
  },
  EnvironmentName/.estore in = \GetTheoremCaseFormatEnvironmentName,
  ShowText/.estore in = \GetTheoremCaseFormatShowText,
  FollowCounter/.estore in = \GetTheoremCaseFormatFollowCounter,
} % End of \pgfkeys{}

\newcommand{\InsertCase}[1]
{%
  \InsertTheoremContent[\empty]{%
    \GetTheoremCaseFormatEnvironmentName}{#1}%
} % End of \newcommand{}

\newcommand{\InitTheoremCaseFormat}
{%
  \ifthenelse{\equal{\GetTheoremCaseFormatFollowCounter}{\empty}}%
  {%
    \newtheorem*{%
      \GetTheoremCaseFormatEnvironmentName}{%
      \GetTheoremCaseFormatShowText}
  }%
  {%
    \newtheorem{%
      \GetTheoremCaseFormatEnvironmentName}{%
      \GetTheoremCaseFormatShowText}[%
      \GetTheoremCaseFormatFollowCounter]%
  }%
} % End of \newcommand{}

% ---------------------------------------------------------

\pgfkeys
{
  /TheoremAcknowledgmentFormat/.is family, /TheoremAcknowledgmentFormat,
  default/.style =
  {
    EnvironmentName = {EnvAcknowledgment},
    ShowText = {Acknowledgment},
    FollowCounter = \empty,
  },
  EnvironmentName/.estore in = \GetTheoremAcknowledgmentFormatEnvironmentName,
  ShowText/.estore in = \GetTheoremAcknowledgmentFormatShowText,
  FollowCounter/.estore in = \GetTheoremAcknowledgmentFormatFollowCounter,
} % End of \pgfkeys{}

\newcommand{\InsertAcknowledgment}[1]
{%
  \InsertTheoremContent[\empty]{%
    \GetTheoremAcknowledgmentFormatEnvironmentName}{#1}%
} % End of \newcommand{}

\newcommand{\InitTheoremAcknowledgmentFormat}
{%
  \ifthenelse{\equal{\GetTheoremAcknowledgmentFormatFollowCounter}{\empty}}%
  {%
    \newtheorem*{%
      \GetTheoremAcknowledgmentFormatEnvironmentName}{%
      \GetTheoremAcknowledgmentFormatShowText}
  }%
  {%
    \newtheorem{%
      \GetTheoremAcknowledgmentFormatEnvironmentName}{%
      \GetTheoremAcknowledgmentFormatShowText}[%
      \GetTheoremAcknowledgmentFormatFollowCounter]%
  }%
} % End of \newcommand{}

% ---------------------------------------------------------

\pgfkeys
{
  /TheoremConclusionFormat/.is family, /TheoremConclusionFormat,
  default/.style =
  {
    EnvironmentName = {EnvConclusion},
    ShowText = {Conclusion},
    FollowCounter = \empty,
  },
  EnvironmentName/.estore in = \GetTheoremConclusionFormatEnvironmentName,
  ShowText/.estore in = \GetTheoremConclusionFormatShowText,
  FollowCounter/.estore in = \GetTheoremConclusionFormatFollowCounter,
} % End of \pgfkeys{}

\newcommand{\InsertConclusion}[1]
{%
  \InsertTheoremContent[\empty]{%
    \GetTheoremConclusionFormatEnvironmentName}{#1}%
} % End of \newcommand{}

\newcommand{\InitTheoremConclusionFormat}
{%
  \ifthenelse{\equal{\GetTheoremConclusionFormatFollowCounter}{\empty}}%
  {%
    \newtheorem*{%
      \GetTheoremConclusionFormatEnvironmentName}{%
      \GetTheoremConclusionFormatShowText}
  }%
  {%
    \newtheorem{%
      \GetTheoremConclusionFormatEnvironmentName}{%
      \GetTheoremConclusionFormatShowText}[%
      \GetTheoremConclusionFormatFollowCounter]%
  }%
} % End of \newcommand{}

% ---------------------------------------------------------

\newcommand{\InitTheoremFormats}
{%
  \InitTheoremDefinitionFormat%
  \InitTheoremConditionFormat%
  \InitTheoremProblemFormat%
  \InitTheoremExampleFormat%
  \InitTheoremTheoremFormat%
  \InitTheoremLemmaFormat%
  \InitTheoremCorollarieFormat%
  \InitTheoremPropositionFormat%
  \InitTheoremConjectureFormat%
  \InitTheoremProofFormat%
  \InitTheoremNoteFormat%
  \InitTheoremAnnotationFormat%
  \InitTheoremClaimFormat%
  \InitTheoremCaseFormat%
  \InitTheoremAcknowledgmentFormat%
  \InitTheoremConclusionFormat%
} % End of \newcommand{}

\newcommand{\SetTheoremFormat}[2][\empty]
{%
  \ifthenelse{\equal{#1}{Definition}}%
  {%
    \pgfkeys{/TheoremDefinitionFormat, default, #2}%
  }{}%
  %
  \ifthenelse{\equal{#1}{Condition}}%
  {%
    \pgfkeys{/TheoremConditionFormat, default, #2}%
  }{}%
  %
  \ifthenelse{\equal{#1}{Problem}}%
  {%
    \pgfkeys{/TheoremProblemFormat, default, #2}%
  }{}%
  %
  \ifthenelse{\equal{#1}{Example}}%
  {%
    \pgfkeys{/TheoremExampleFormat, default, #2}%
  }{}%
  %
  \ifthenelse{\equal{#1}{Theorem}}%
  {%
    \pgfkeys{/TheoremTheoremFormat, default, #2}%
  }{}%
  %
  \ifthenelse{\equal{#1}{Lemma}}%
  {%
    \pgfkeys{/TheoremLemmaFormat, default, #2}%
  }{}%
  %
  \ifthenelse{\equal{#1}{Corollarie}}%
  {%
    \pgfkeys{/TheoremCorollarieFormat, default, #2}%
  }{}%
  %
  \ifthenelse{\equal{#1}{Proposition}}%
  {%
    \pgfkeys{/TheoremPropositionFormat, default, #2}%
  }{}%
  %
  \ifthenelse{\equal{#1}{Conjecture}}%
  {%
    \pgfkeys{/TheoremConjectureFormat, default, #2}%
  }{}%
  %
  \ifthenelse{\equal{#1}{Proof}}%
  {%
    \pgfkeys{/TheoremProofFormat, default, #2}%
  }{}%
  %
  \ifthenelse{\equal{#1}{Note}}%
  {%
    \pgfkeys{/TheoremNoteFormat, default, #2}%
  }{}%
  %
  \ifthenelse{\equal{#1}{Annotation}}%
  {%
    \pgfkeys{/TheoremAnnotationFormat, default, #2}%
  }{}%
  %
  \ifthenelse{\equal{#1}{Claim}}%
  {%
    \pgfkeys{/TheoremClaimFormat, default, #2}%
  }{}%
  %
  \ifthenelse{\equal{#1}{Case}}%
  {%
    \pgfkeys{/TheoremCaseFormat, default, #2}%
  }{}%
  %
  \ifthenelse{\equal{#1}{Acknowledgment}}%
  {%
    \pgfkeys{/TheoremAcknowledgmentFormat, default, #2}%
  }{}%
  %
  \ifthenelse{\equal{#1}{Conclusion}}%
  {%
    \pgfkeys{/TheoremConclusionFormat, default, #2}%
  }{}%
  %
} % End of \newcommand{}

% ---------------------------------------------------------

\SetTheoremFormat[Definition]{ShowText = {Definition}}%
\SetTheoremFormat[Condition]{ShowText = {Condition}}%
\SetTheoremFormat[Problem]{ShowText = {Problem}}%
\SetTheoremFormat[Example]{ShowText = {Example}}%
\SetTheoremFormat[Theorem]{ShowText = {Theorem}}%
\SetTheoremFormat[Lemma]{ShowText = {Lemma}}%
\SetTheoremFormat[Corollarie]{ShowText = {Corollarie}}%
\SetTheoremFormat[Proposition]{ShowText = {Proposition}}%
\SetTheoremFormat[Conjecture]{ShowText = {Conjecture}}%
\SetTheoremFormat[Proof]{ShowText = {Proof}}%
\SetTheoremFormat[Note]{ShowText = {Note}}%
\SetTheoremFormat[Annotation]{ShowText = {Annotation}}%
\SetTheoremFormat[Claim]{ShowText = {Claim}}%
\SetTheoremFormat[Case]{ShowText = {Case}}%
\SetTheoremFormat[Acknowledgment]{ShowText = {Acknowledgment}}%
\SetTheoremFormat[Conclusion]{ShowText = {Conclusion}}%

% ----------------------------------------------------------------------------
