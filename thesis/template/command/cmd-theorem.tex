%
% This file is part of the project of
% National Cheng Kung University (NCKU) Thesis/Dissertation Template in LaTex.
% This project is hold at
%     <https://github.com/wengan-li/ncku-thesis-template-latex>
% by Wen-Gan Li.
%
% This project is distributed in the hope of usefuling to someone,
% you can redistribute it and/or modify it under the terms of the
% Attribution-NonCommercial-ShareAlike 4.0 International.
%
% You should have received a copy of the
% Attribution-NonCommercial-ShareAlike 4.0 International
% along with this project.
% If not, see <http://creativecommons.org/licenses/by-nc-sa/4.0/legalcode.txt>.
%
% Please feel free to fork it, modify it, and try it.
% Have fun !!!
%

% Reference from:
%   <https://www.sharelatex.com/learn/Theorems_and_proofs>

% ----------------------------------------------------------------------------

  %    definition       (定義)
  %    condition        (條件)
  %    theorem          (定理)
  %    lemma            (引理)
  %    example          (例子)
  %    corollary        (推論)
  %    proposition      (主張)
  %    remark           (備註)
  %    proof            (證明)
  %    conjectures      (猜想)
  %    note             (附註)
  %    annotation       (註解)
  %    claim            (主張)
  %    case             (情況)
  %    acknowledgment   (確認)
  %    conclusion       (結論)

  % ----------------------------------------------------------------------------

\pgfkeys
{
  /InsertTheoremOptions/.is family, /InsertTheoremOptions,
  default/.style =
  {
    title = \empty,
    label = \empty,
  },
  title/.estore in = \TmpValueTitle,
  label/.estore in = \TmpValueLabel,
} % End of \pgfkeys{}

\newcommand{\SetTheoremContentLabel}[1][\empty]
{
  \ifthenelse{\equal{#1}{\empty}}{}{\IfNoValueF{#1}{\label{#1}}}
} % End of \DeclareDocumentCommand{}

\newcommand{\InsertTheoremContent}[3][\empty]
{%
  % Parse the input
  \pgfkeys{/InsertTheoremOptions, default, #1}%
  %
  \ifthenelse{\equal{\TmpValueTitle}{\empty}}
  {%
    \begin{#2}%
    \SetTheoremContentLabel{\TmpValueLabel}%
    #3%
    \end{#2}%
  }%
  {%
    \begin{#2}[\TmpValueTitle]%
    \SetTheoremContentLabel{\TmpValueLabel}%
    #3%
    \end{#2}%
  }%
} % End of \newcommand{}

% ---------------------------------------------------------

\pgfkeys
{
  /TheoremFormat/.is family, /TheoremFormat,
  default/.style =
  {
    EnvironmentName = {EnvTheorem},
    ShowText = {Theorem},
    FollowCounter = section,
  },
  EnvironmentName/.estore in = \GetTheoremFormatEnvironmentName,
  ShowText/.estore in = \GetTheoremFormatShowText,
  FollowCounter/.estore in = \GetTheoremFormatFollowCounter,
} % End of \pgfkeys{}

\newcommand{\InsertTheorem}[2][\empty]
{%
  \InsertTheoremContent[#1]{\GetTheoremFormatEnvironmentName}{#2}%
} % End of \newcommand{}

\newcommand{\InitTheoremFormat}
{%
  \ifthenelse{\equal{\GetTheoremFormatFollowCounter}{\empty}}
  {%
    \newtheorem{%
      \GetTheoremFormatEnvironmentName}{%
      \GetTheoremFormatShowText}%
  }%
  {%
    \newtheorem{%
      \GetTheoremFormatEnvironmentName}{%
      \GetTheoremFormatShowText}[\GetTheoremFormatFollowCounter]%
  }%
} % End of \newcommand{}

% ---------------------------------------------------------

\pgfkeys
{
  /ProofFormat/.is family, /ProofFormat,
  default/.style =
  {
    EnvironmentName = {EnvProof},
    ShowText = {Proof},
    FollowCounter = \empty,
  },
  EnvironmentName/.estore in = \GetProofFormatEnvironmentName,
  ShowText/.estore in = \GetProofFormatShowText,
  FollowCounter/.estore in = \GetProofFormatFollowCounter,
} % End of \pgfkeys{}

\newcommand{\InsertProof}[1]
{%
  \InsertTheoremContent[\empty]{\GetProofFormatEnvironmentName}{#1}%
} % End of \newcommand{}

\newcommand{\InitProofFormat}
{%
  \ifthenelse{\equal{\GetProofFormatFollowCounter}{\empty}}
  {%
    \newtheorem*{\GetProofFormatEnvironmentName}{\GetProofFormatShowText}
  }%
  {%
    \newtheorem{%
      \GetProofFormatEnvironmentName}{%
      \GetProofFormatShowText}[\GetProofFormatFollowCounter]%
  }%
} % End of \newcommand{}

% ---------------------------------------------------------

\newcommand{\InitTheoremFormats}
{%
  \InitTheoremFormat%
  \InitProofFormat%
} % End of \newcommand{}

\newcommand{\SetTheoremFormat}[2][\empty]
{%
  \ifthenelse{\equal{#1}{Theorem}}
  {%
    \pgfkeys{/TheoremFormat, default, #2}%
  }{}%
  %
  \ifthenelse{\equal{#1}{Proof}}
  {%
    \pgfkeys{/ProofFormat, default, #2}%
  }{}%
  %
} % End of \newcommand{}

% ---------------------------------------------------------

\SetTheoremFormat[Theorem]{ShowText = {Theorem}}%
\SetTheoremFormat[Proof]{ShowText = {Proof}}%

% ----------------------------------------------------------------------------
