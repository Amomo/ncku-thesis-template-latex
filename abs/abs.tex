\begin{center}
\large \textbf{Bus-Pin-Aware Bus-Driven Floorplanning} \\[15mm]
\normalsize \textbf{Student: Po-Hsun Wu \hspace{5mm} Advisor: Dr. Tsung-Yi Ho\\[7mm]}
\normalsize \textbf{Department of Computer Science and\\
                    Information Engineering \\
                    National Cheng Kung University\\
                    Tainan, Taiwan, R.O.C.\\[7mm]}
\large \textbf{Abstract}
\end{center}
%%%%%%%%%%%%%%%%%%%%%%%%%%
\label{abs}
%%%%%%%%%%%%%%%%%%%%%%%%%%

\baselineskip=26pt


As the number of buses increase substantially in multi-core SoC
designs, the bus planning problem has become the dominant factor
in determining the performance and power consumption of SoC
designs. To cope with the bus planning problem, it is desirable to
consider this issue in early floorplanning stage. Recently,
bus-driven floorplanning problem has attracted much attention in
the literature. However, current algorithms adopt an
over-simplified formulation ignoring the position and
orientation of the bus pins, the chip performance may be deteriorated.
In this paper, we propose the bus-driven
floorplanning algorithm that fully considers the impacts of the bus
pins. By fully utilizing the position and orientation of the bus pins,
bus bendings are not restricted to occur at the modules on the bus,
then it has more flexibility during bus routing. With more
flexibility on the bus shape, the size of the solution
space is increased and a better bus-driven floorplanning solution
can be obtained. Compared with the bus-driven
floorplanner \cite{Ma08}, the experimental results show that our
algorithm performs better in runtime by 3.5$\times$, success rate
by 1.2$\times$, wirelength by 1.8$\times$, and reduced the
deadspace by 1.2$\times$.


\begin{itemize}
\item {\bf Keywords:} Floorplanning; Bus planning

\end{itemize}
