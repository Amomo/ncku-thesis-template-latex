
% ----------------------------------------------------------------------------
%                               English abstract
%                                   英文摘要
% ----------------------------------------------------------------------------

% Set the line spacing
\baselineskip = 26pt

% ------------------------------------------------

% Page start
\newpage
\phantomsection

% Add to "Table of Contents"
\addcontentsline{toc}{chapter}{Abstract}

% ------------------------------------------------

\begin{center}
\large \textbf{Abstract}
%\label{abstract}
\end{center}

% ------------------------------------------------

%\minisec{Abstract}
%In big data, the meaning of owning one petabyte or ten petabyte are the same --- meaningless. The information is hiding inside the data, this means how fast can the data retrieval from database and process it to become an information we need that is the major question in this field, and normally the bottleneck is the query operation that between the database \cite{paper:nodb} and the program.\\

%To lower the searching time in database, people are start using non-relational database from relational database to seek for less searching time. But this kind of changing may meet some problem, such as re-design the schema of the database. Because non-relational database are using key-value pair design, if you want to retrieve a value, you need to know its key, but you can't using the value to find the key or using conditional searching (like the 'WHERE Clause' in SQL) to search the key or value. This will cause some feature which basic on the query operation un-functional in program, this may need to modify tons of code to become useable. Also if these kind of features are very important and use widely which may need to use MapReduce to distribute the searching into many servers for decreasing the searching times. This is highly increased the work for programmer and system designer, also increased the cost of maintenance the servers.\\

%So this paper is to design a indexing algorithm custom for key-value stores --- Li's Hash, using the original key-value usage to implement a indexing to provide the basic query feature in relational database such as the searching and comparison operation, also the time complexity of these feature is $O(1)$ or $O(b)$ ($b$ is the data's byte length store in Li's Hash).\\

%Because Li's Hash is design as a dynamic indexing algorithm which means the indexing are doing when input a key-value data, rather than normal indexing which needed a freeze state. By using modularized design for swappable the back-end database, we have implemented a prototype to provide as a library \cite{web:lishash:home-page} that let the user can use relational or non-relational APIs with different back-end non-relational database, just only need to modify the code for connecting to our library, then they don't need to modified any schema in original design, and keep the relational database concept to use non-relational database without need to learn any new concept about the back-end database.\\

In age of big data, owning one petabyte or ten petabyte is meaningless, because the storage no longer as primary issue.\\

The information is hiding inside the data, this means how fast can the data retrieval from database and process it to become an information we needed that is the major issue.\\

People are start using non-relational database from relation database to seek for less searching time. But non-relational database are using key-value pair design, if you want to retrieve a value, you need to know its key, but you can't using the value to find the key or using conditional searching (like the 'WHERE Clause' in SQL) to search the key or value. This will cause some feature which basic on the query operation un-functional, also if these kind of features are very important and use widely which may need to use MapReduce to distribute the searching. This will highly increased the work for programmer and cost of maintenance the servers.\\

So this paper is to design a indexing algorithm custom for key-value stores --- Li's Hash, using the original key-value usage to implement a indexing to provide the query feature in relational database such as the searching and comparison operation, also the time complexity of these feature is $O(1)$ or $O(b)$ ($b$ is the data's byte length store in Li's Hash).\\

This paper have implemented a prototype to build up a library in order to provide relational or non-relational APIs with different back-end non-relational database, user only need to modify the code for connecting to this library, then they don't need to modified any schema in original design, and keep the relational database concept to use non-relational database without need to learn any new concept about the back-end database.\\

% ------------------------------------------------

\begin{itemize}
\item {\bf Keywords:} Floorplanning, Bus planning
\end{itemize}

% ------------------------------------------------

% End of page

% ------------------------------------------------
